\section{Векторный алгоритм Кленшоу}
\emph {Векторный алгоритм Кленшоу} является обобщением
классического варианта для ряда, представляющего разложение по
векторным ортогональным многочленам вида:
$$%\begin{equation}
S_n(x)=\sum\limits_{i=0}^{n}\beta_iQ_i(x),
$$%\end{equation}
где многочлены $\{Q_n\}$ удовлетворяют рекуррентному соотношению
вида:
$$%\begin{equation}
Q_{n+1}=(z+b_{n,n})Q_n+b_{n,n-2}Q_{n-1}+\ldots+b_{n,n-p}Q_{n-p}
$$%\end{equation}
В качестве начальных условий выбираются: \\
\begin{tabular} {llll}
&    &    $u_n=\beta_nQ_n$ \\
&    &    $u_{n+1}=\ldots=u_{n+p}=0$ \\
\end{tabular} \\
Далее по рекуррентной следующей формуле  вычисляются последовательно значения $u_{n-1},\ldots,u_0$\\
$$
u_k=(x+b_{k,k})u_{k+1}+b_{k+1,k}u_{k+2}+\ldots+b_{k+p,k}u_{k+p+1}+\beta_kQ_0,
k=n-1,\ldots,0
$$
Частичная сумма ряда в результате $$S_n(x)=u_0$$

\subsection {Пример алгоритма для p=2}
Ограничимся 4 членами для частичной суммы ряда:
$$
S_3(x)=c_0Q_0+c_1Q_1+c_2Q_2+c_3Q_3
$$ 
Запишем реккурентное соотношение:
$$
Q_{n+1}=(x-a_{n,n})Q_n-a_{n,n-1}Q_{n-1}-a_{n,n-2}Q_{n-2} 
$$
Выразим частичные суммы через друг друга\\
\begin{tabular} {llllllll}
$u_3=c_3Q_0, u_4=u_5=0$ \\
$u_2=(x-a_{2,2})u_3 + c_2Q_0$ \\
$u_1=(x-a_{1,1})u_2-a_{2,1}u_3+c_1Q_0$ \\
$u_0=(x-a_{0,0})u_1-a_{1,0}u_2-a_{2,0}u_3+c_0Q_0$ \\
\end{tabular} \\ \\
Проверим подстановкой \\
\begin{tabular} {llllllllll}
$S_3=$ & $c_0Q_0+c_1[(x-a_{0,0})Q_0]+$\\
& $c_2[(x-a_{1,1}(x-a_{0,0})Q_0-a_{1,0}Q_0]+$\\
& $c_3[(x-a_{2,2})((x-a_{1,1})(x-a_{0,0})Q_0-a_{1,0}Q_0)-a_{2,1}(x-a_{0,0})Q_0-a_{2,0}Q_0]$ \\
\end{tabular}\\ \\
Выразим частичные суммы через коэффициенты реккурентных соотношений\\
\begin{tabular} {llllllllll}
$u_2=$ & $(x-a_{2,2})c_3Q_0+c_2Q_0$ \\
$u_1=$ & $(x-a_{1,1})[(x-a_{2,2})c_3Q_0+c_2Q_0]-a_{2,1}Q_0+c_1Q_0$ \\
$u_0=$ & $(x-a_{0,0})[(x-a_{1,1})[(x-a_{2,2})c_3Q_0+c_2Q_0]-a_{2,1}Q_0+c_1Q_0]-$ \\
& $a_{1,0}[(x-a_{2,2})c_3Q_0+c_2Q_0]-a_{2,0}c_3Q_0+c_0Q_0$\\ 
$u_0=$  & $c_0Q_0+c_1[(x-a_{0,0})Q_0]+$\\
& $c_2[(x-a_{1,1}(x-a_{0,0})Q_0-a_{1,0}Q_0]+$\\
& $c_3[(x-a_{2,2})((x-a_{1,1})(x-a_{0,0})Q_0-a_{1,0}Q_0)-a_{2,1}(x-a_{0,0})Q_0-a_{2,0}Q_0]$ \\
\end{tabular}\\
Равенство $S_3(x)=u_0$ очевидно.