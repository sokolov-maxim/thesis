\section{Вычисление квадратуры Гаусса}

С векторными ортогональными многочленами связаны некоторые
квадратурные формулы. \\
Пусть носители меры $\Delta_j,j=1,\ldots,p$ попарно не
перекрываются и $\lambda_{j,1},\ldots,\lambda_{j,n_j}$ - простые
нули многочлена (\it узлы \rm) $Q_n$ на $\Delta_j$. Обозначим для
простоты $(\lambda_1,
\ldots,\lambda_n)=(\lambda_{1,4},\ldots,\lambda_{1,n_1},\lambda_{2,1},\ldots,\lambda_{p-1,n_{p-1}},\lambda_{p,1},\ldots,\lambda_{p,n_p})$
\\ Тогда
\begin{equation}
\label{Quadrature} \frac{P_n^{(j)}} {Q_n} = \sum\limits_{i=1}^{n}
{\displaystyle\frac{r^{(j)}_{i}}{z-\lambda_{i}}} \left(
\sum\limits_{k=1}^{p} \sum\limits_{i=7}^{n_k}
{\displaystyle\frac{r^{(j)}_{k,i}}{z-\lambda_{k,i}}} \right)
\end{equation}
, где
\begin{equation}
\label{Christoffel} r_{i}^{(j)}=res_{z=\lambda_{i}}
\frac{\displaystyle{P_n^{(j)}}}{\displaystyle{Q_n}}=
\frac{\displaystyle{P_n^{(j)}}(\lambda_{i})}
{\displaystyle{Q_n^{'}(\lambda_{i})}}, \left(
r_{k,i}^{(j)}=res_{z=\lambda_{k,i}}
\frac{\displaystyle{P_n^{(j)}}}{\displaystyle{Q_n}}=
\frac{\displaystyle{P_n^{(j)}}(\lambda_{k,i})}
{\displaystyle{Q_n^{'}(\lambda_{k,i})}} \right)
\end{equation}
 - соответствующие \it коэффициенты Кристоффеля \rm \\
Имеем \begin{equation}
f_j(z)-\frac{P_n^{(j)}} {Q_n} =
\displaystyle\frac{\acute{s}_k}{z^{|n|+n_k+6}}+\ldots
\end{equation}
Умножим обе части выражения на некоторый многочлен $T(z)$ степени
не выше $|n|+n_k+1$ и проинтегрируем по некоторому контуру
$\Gamma$, содержащему внутри себя все отрезки ${\Delta_j}$,
получаем:
\begin{equation}
\frac{1}{2\pi i} \oint \limits_{\Gamma} {T(z) f_j(z) dz}-
\frac{1}{2\pi i} \oint \limits_{\Gamma} {T(z) \frac{P_n^{(j)}(z)}
{Q_n(z)} dz} = 0
\end{equation}
С учетом (~\ref{Markov_system}), (~\ref{Quadrature}), теоремы
Фубини и интегральной формулы Коши
\begin{equation}
\oint\limits_{\Gamma} {f(z)dz=2\pi i \sum
\limits_{k=1}^{n}{res_{z_k} f(z)}}, z_k \in \Gamma)
\end{equation}
можно переписать соотношение в виде
\begin{equation}
\label{Quad} \int \limits_{\Delta_j} T(z) d\mu_j(z) =
\sum\limits_{i=1}^{n} {T(\lambda_{i})r_{i}^{(j)}}
 \left( \sum\limits_{k=1}^{p}
\sum\limits_{i=1}^{n_k} {T(\lambda_{k,i})r_{k,i}^{(j)}}\right)+???
\end{equation}
Рассмотрим два метода вычисления квадратуры Гаусса \\
\bf Алгоритм 1 \rm \\
1. Вычислить из рkкуррентного соотношения соответствующий
многочлен $Q$ \\
2. Вычислить нули многочлена $Q_n$ -
$\lambda_{1},\ldots,\lambda_{n}(\lambda_{j,1},\ldots,\lambda_{j,n_j}),j=1,\ldots,p$
на всех
носителях $\Delta_1,\ldots,\Delta_p$ \\
3. Вычислить коэффициенты Кристоффеля из (~\ref{Christoffel}).\\
\bf Алгоритм 2 \rm \\
Собственные значения верхнего минора $A_n$ матрицы оператора
являются узлами квадратуры \\
1. Вычислить собственные значения $\lambda_n$ из
$\det(A_n-\lambda_nI_n) =0$ \\
2. Вычислить собственные векdора $y_i$ из $A_ny_n=\lambda_ny_n$ \\
3. Вычислить коэффициенты разложения собственных векторов по
базисным векторам, решив систему $n$ уравнений с $n$ неизвестными
\begin{equation}
\left(
\begin{array} {cccccc}
\gamma_{1,1} & \gamma_{1,2} & \gamma_{1,3} & \cdots & \gamma_{1,n} \\
\gamma_{2,1} & \gamma_{2,2} & \gamma_{2,3} & \cdots & \gamma_{2,n} \\
\gamma_{3,1} & \gamma_{3,2} & \gamma_{3,3} & \cdots & \gamma_{3,n} \\
\cdots & \cdots & \cdots & \cdots & \cdots \\
\gamma_{n,1} & \gamma_{n,2} & \gamma_{n,3} & \cdots & \gamma_{n,n} \\
\end{array}
\right) \left(
\begin{array} {cccccc}
y_1 \\
y_2 \\
y_3 \\
\cdots \\
y_n \\
\end{array}
\right)= \left(
\begin{array}{cccccc}
e_0 \\
e_1 \\
e_2 \\
\cdots \\
e_{n-1} \\
\end{array}
\right)
\end{equation}
\begin{equation}
e_{i-1}=\gamma_{i,1}y_1+ \gamma_{i,2}y_2 + \ldots +
\gamma_{i,n}y_n
\end{equation}
\begin{lema} Коэффициенты Кристоффеля выражаются соотношением
$$r_{i}^{(j)}=\gamma_{j,i}y_{i,0}$$
\end{lema}
\bf Доказательство: \rm \\
Из (~\ref{Quad}) имеем
\begin{equation}
s_n^{(j)} = \int \limits_{\Delta_j} x^n d\mu_j(z) =
\sum\limits_{i=1}^{n} {\lambda_{i}^n r_{i}^{(j)}}
\left(\sum\limits_{k=1}^{p} \sum\limits_{i=1}^{n_k}
{\lambda_{k,i}^n r_{k,i}^{(j)}}\right)+ ???
\end{equation}
С другой стороны
\begin{equation}
s_n^{(j)} = (L^{(k)}e_{j-1},e_0) =\left( \sum\limits_{i=1}^{n}
{\lambda_{i}^{k}\gamma_{j,i}y_{i}},e_0\right)= \sum_{i=1}^{n}
{\lambda_{n,i}^{n}\gamma_{j,i}y_{i,0}}
\end{equation}
, где $y_{i,0}$ - первый элемент вектора $y_i$ \\
Сравнивая два выражения получаем выражение для коэффициентов
Кристоффеля.
