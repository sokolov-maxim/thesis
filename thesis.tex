%\documentclass[12pt, a4paper]{report}
%\usepackage[russianb]{babel}
%\parindent 0.5 cm
%\textwidth 15.5 cm \textheight 24 cm \topmargin -1.5 cm
%\evensidemargin .5 cm \oddsidemargin .5 cm \baselineskip=14 pt plus 1 pt

\documentclass{report}     
\usepackage{fontspec}
\usepackage{polyglossia}

\setdefaultlanguage{russian}
\setmainfont[Mapping=tex-text]{CMU Serif}

\title{Диссертация на тему "Ортогональные многочлены"} 
\author{Максим Соколов}     
\date{Апрель, 2018}     

\newtheorem{teor}{Теорема} [chapter]
\newtheorem{prope}{Свойство} [chapter]
\newtheorem{coly}{Заключение} [chapter]
\newtheorem{defi}{Определение} [chapter]
\newtheorem{lema}{Лемма} [chapter]

\begin{document}

% $i,j,k,n$ - индексы \\
% $z,x$ - переменные \\
% $\mu$ - мера \\
% $\Delta$ - носитель меры \\
% $Q(z)$ - ортогональные многочлены II типа, знаменатели
% аппроксимации Паде \\
% $P(z)$ - ортогональные многочлены II типа второго рода, числители
% аппроксимации Паде \\
% $C(z)$ - ортогональные многочлены I типа \\
% $D(z)$ - ортогональные многочлены I типа второго рода \\
% $L$ - линейный функционал \\
% $\alpha$ - коэффициенты разложения
% $xQ_n=\sum\limits_{i=n-p}^{n+1}{\alpha_{n,i}Q_i}$
% $f=(f_1,f_2,\ldots,f_p)$ - система функций (марковских например)\\
% $s_{k}$ - моменты \\

\begin{flushright}
\today
\tableofcontents
\end{flushright}

%\chapter {Векторные ортогональные многочлены  }
\chapter {Векторные ортогональные многочлены  }
\section{Определение, общие свойства}
%===================================================================
%===================================================================
%\begin{scriptsize}
%\it Борелевской $\sigma$-алгеброй (на прямой \bf R \rm ) называют
%наименьшую $\sigma$-алгебру, содержащую все компактные
%подмножества \bf R  \it Конечная позитивная борелевская мера\rm -
%$\sigma$-аддитивная функция множества, определенная
% на борелевской $\sigma$-алгебре и принимающая конечные неотрицательные значения
%Точку $\lambda_0\in \bf R \rm $ - \it точка роста меры \rm если
%для любого $\varepsilon>0$ имеем
%$\mu(\lambda_0-\varepsilon,\lambda_0+\varepsilon)>0$ Множество
%точек роста меры замкнуто и называется \it спектром меры.
%Носитель меры $\Delta_i$ - наименьший промежуток, содержащий
%спектр\rm \\
%\end{scriptsize}
%================================================================
%================================================================

Рассмотрим набор позитивных борелевских мер
$\mu_1,\mu_2,\ldots,\mu_p$ с бесконечно большим количеством точек
роста на соответствующих носителях
$\Delta_1,\Delta_2,\cdots,\Delta_p$. \\
Введем вектор индексов $\overrightarrow{n}=(n_1,\ldots,n_p)$ \\
%=========================================================
Пусть
\begin{equation}
\label{Moments} s_n^{(j)}=\int \limits_{\Delta_j} {z^n d\mu_j
(z)}, n \in \textbf{Z}_{+}, j=1,2,\ldots,p
\end{equation}
\textit{степенные моменты, построенные по данным позитивным
мерам} \\
%============================================================
Определим $p$ линейных функционалов $L_1,L_2,\ldots,L_p$ в
комплексном линейном пространстве многочленов $\textbf{C}[z]$,
соответствующих последовательностям степенных моментов ${s^{(1)}},
\ldots, {s^{(p)}}$:
\begin{equation}
\label{Functionals}
L_j(z^n)=\int_{\Delta_j}{z^n(x)d\mu_j(x)}=s_n^{(j)},\mbox{    }
j=1,2,\ldots,p
\end{equation}
%=======================================================
\begin{prope}
Важным свойством позитивных мер является позитивность моментов
$s_n^{(j)}$
\end{prope}
%=========================================================
\begin{prope}
Функционалы $L_1,L_2,\ldots,L_p$, отвечающие $s_n^{(j)}$
позитивны, т.е. для любого многочлена $Q(z) \in \textbf{C}[z],
Q(z) \geq 0, z \in [-\infty,+\infty]$ выполняется неравенство
$L_j(Q(z)) \geq 0$, причем $L_j(Q(z))=0$, только если $Q = 0$. \rm
\end{prope}
%============================================================
\begin{defi} Совместно ортогональными многочленами \rm в данном
случае называются многочлены $Q_n$, степени не выше
$|\overrightarrow{n}|=n_1+n_2+\cdots+n_p$, которые удовлетворяют
следующим условиям ортогональности:
\begin{equation}
\label{OrthogonalCondition}
\int_{\Delta_j}{Q_n(x)x^kd\mu_j(x)}=0,\mbox{
}k=0,1,\ldots,n_j-1,j=1,2,\ldots,p
\end{equation}
\end{defi}
%=================================================================
\begin{defi} Векторными ортогональными многочленами \rm называются
совместно ортогональные многочлены, для которых вектор индексов
выбирается в общем случаем как
$\overrightarrow{n}=(\underbrace{k+1,\ldots,k+1}_{d},\underbrace{k,\ldots,k}_{p-d}),
k\in{\mbox{Z}}_{+},n=pk+d$ \\
В общем случае векторные ортогональные многочлены многочлены $Q_n$
определяются не единственным образом. Для определенности будем
считать, что индекс $n$ - \it нормален \rm ($\deg  Q_n=n$) и
$Q_n=z^n+\ldots$. Приведем некоторые важные свойства.
\end{defi}
%===================================================================
\begin{prope} Если носители мер попарно не пересекаются
$\Delta_j\cap\Delta_i=0, j,i=1,2,\ldots,p$, то выполняется:
\begin{equation}
\label{Ortogonality} \int_{\Delta_j}{Q_{\overrightarrow{n}}(x)}
x^{n_j}d\mu_j \not=0, \hspace{1cm}j=1,2,\ldots,p
\end{equation}
\end{prope}
%=====================================================================
\begin{prope}
Векторные ортогональные многочлены $Q_n$ удовлетворяют
рекуррентному соотношению:
\begin{equation}
\label{QRecurrrence}
Q_{n+1}=(z+b_{n,n})Q_n+b_{n,n-1}Q_{n-1}+\ldots+b_{n,n-p}Q_{n-p},
\hspace{1cm} n=pk+d
\end{equation}
при условии $Q_{-p}=\ldots=Q_{-1}=0,Q_0=1, b_{n,n-p}\not=0$ \\
\end{prope}
\bf Доказательство: \rm \\
Соотношение легко проверить. \\ Рассмотрим общий случай для
индекса $n=pk+d$, соответствующий $Q_n$ вектор индексов имеет
следующий вид:
$\overrightarrow{n}=(\underbrace{k+1,\ldots,k+1}_{d},\underbrace{k,\ldots,k}_{p-d})$.
\\ Разложим многочлен $xQ_n$ по базису $Q_0,Q_1,\ldots,Q_{n+1}$
\begin{equation}
\label{xQ}
xQ_n=\sum\limits_{i=0}^{n+1}{\alpha_{n,i}Q_i},\mbox{
}\alpha_{n,n+1}=1
\end{equation}
Рассмотрим случай $k=0$. Применим функционал $L_1$ к разложению
(~\ref{xQ}). Из условия ортогональности левая часть выражения
станет равной нулю, в правой части останется только одно
слагаемое $\alpha_{n,0}L_1(Q_0)$.
$L_1(Q_0)\not=0$, следовательно $\alpha_{n,0}=0$. \\
Далее применим к (~\ref{xQ}) функционал $L_2$. Левая часть по
прежнему останется равной нулю, в правой части останется два
слагаемых $\alpha_{n,1}L_2(Q_1)+\alpha_{n,0}L_2(Q_0)$.
Учитывая $\alpha_{n,0}=0$ и $L_2(Q_1)\not=0$ получаем $\alpha_{n,1}=0$. \\
Последовательно применяя функционалы $L_1,L_2,\ldots,L_p$ получим
$\alpha_{n,0}=\alpha_{n,1}=\ldots=\alpha_{n,p-1}=0$ Ha некотором
шаге $j=0,1,\ldots,k-2$ домножая выражение (~\ref{xQ}) с обоих
сторон на $x^j$ и затем последовательно применяя функционалы
$L_1,L_2,\ldots,L_p$ в результате получим
$\alpha_{n,0}=\ldots\alpha_{n,p-1}=\ldots=\alpha_{n,jp}=\ldots=\alpha_{n,(k-1)p-1}=0$ \\
На последующих шагах процедуры $j=k-1,k,\ldots$ левая часть
выражения (~\ref{xQ}) уже не принимает нулевое значение. Отбросив
нулевые коэффициенты разложения можно записать:
$$
xQ_n=\sum\limits_{i=n-p}^{n+1}{\alpha_{n,i}Q_i}
$$
т.е. $b_{n,i}=-\alpha_{n,i},i=n-p,n-p+1,\ldots,n$ \\ Обратное
утверждение носит название \it теоремы Фавара \rm : если для
любого $n$ многочлены удовлетворяют рекуррентному соотношению вида
(~\ref{QRecurrrence}), то существует некоторый набор функционалов
$L_1,L_2,\ldots,L_p$, по отношению к которому многочлены векторно
ортогональны.\\ \\

\section{Аппроксимации Эрмита-Паде набора марковских функций}
%\subsection{Аппроксимации Эрмита-Паде}

Пусть $\overrightarrow{f}=(f_1,f_2,\ldots,f_p)$ - система
марковских функций:
\begin{equation}
\label{Markov_system} f_j(z)
=\int_{\Delta_j}{\displaystyle\frac{d\mu_j(x))}{z-x}}
=\sum\limits_{k=0}^{\infty} \frac { s_{k}^{(j)} } {z^{k+1}}
=\frac { s_{0}^{(j)} } {z} + \frac { s_{1}^{(j)} } {z^{2}} +\frac
{ s_{2}^{(j)} } {z^{3}}+\cdots
\end{equation}
Пусть
\begin{equation}
\label{H} H_n=\left(
\begin{array}{ccccc}
s_0           & s_{1}         & \cdots & s_{n-1}        \\
s_{1}         & s_{2}         & \cdots & s_{n}          \\
\cdots        & \cdots        & \cdots & \cdots         \\
s_{k-1}       & s_{k}         & \cdots & s_{k+n-2}      \\
s_{k}^{(d)}   & s_{k+1}^{(d)} & \cdots & s_{k+n-1}^{(d)}
\end{array}
\right)
\end{equation}
определитель Ганкеля размерности $n \times n$, построенный по
системе функций $f$, где $n=pk+d, \mbox{   }
s_j=(s_j^{(1)},s_j^{(2)},\ldots,s_j^{(p)})^T, \mbox{   }
s_i^{d}=(s_i^{(1)},s_i^{(2)},\ldots,s_i^{(d)})^{\mbox{T}}$ \\
%=================================================================
%     Задачач А  определение
%==================================================================
\textbf{Задача A} \\ \textit{Для некоторого фиксированного вектора
индексов $\overrightarrow{n}=(n_1,n_2,\ldots,n_p),n_j\in{ \textbf{N} }$ требуется найти многочлен $Q_n\not=0,\deg
(Q_n)\leq{|\overrightarrow{n}|}$ такой, что для некоторых
многочленов $P_n^{(1)},P_n^{(2)},\ldots,P_n^{(p)}$ выполнялось
соотношение:\/}
\begin{equation}
\label{Vector_Pade} Q_n(z)f_j(z)-P_n^{(j)}= \frac {s_j^{'}}
{z^{n_{j+1}}} +\ldots,j=1,2,\ldots,p
\end{equation}
Соотношение эквивалентно системе $n$ линейных однородных уравнений
c $n+1$ неизвестными:
\begin{eqnarray}
\label{QH0}
Q_n(z)=\beta_nz^n+\beta_{n-1}z^{n-1}+\ldots+\beta_1z+\beta_0,
\mbox{     } n=pk+d \nonumber
\\ \left(
\begin{array}{cccccccc}
\beta_0 \\ \beta_1 \\ \ldots \\  \beta_{n-1}
\\ \beta_n
\end{array}
\right) \left(
\begin{array}{ccccccccccc}
 &    &         & s_n     \\
 &    &         & s_{n+1} \\
 &    & H_n     & \cdots \\
 &    &         & s_{k+n-1} \\
 &    &         & s_{k+n}^{d}
\end{array}
\right)= \left(
\begin{array}{cccccccc}
0 \\ 0 \\ \ldots \\ 0 \\ 0
\end{array}
\right)
\end{eqnarray}

Такая система всегда имеет ненулевое решение, при
этом рациональная функция:\\
\begin{equation}
\label{HermitePade} \overrightarrow{\pi}_{\overrightarrow{n}}=
\left( \frac {P^{(1)}} {Q}, \frac {P^{(2)}} {Q}, \cdots, \frac
{P^{(p)}} {Q} \right)
\end{equation}
называется \textit{$n$-ой совместной аппроксимацией Эрмита-Паде}
системы $\overrightarrow{f}$ с разложением в бесконечности. \\ В
общем случае вектор $\overrightarrow{\pi}$ определен не
единственным образом, хотя существует довольно большой класс
марковских функций для которых аппроксимация Эрмита-Паде
единственна. Достаточным условием единственности является \textit{
нормальность индекса} $n$. Индекс $n$ называется нормальным
относительно задачи A, если для любого решения $\deg Q=|\overrightarrow{n}|$. \\
%===============================================================
\begin{coly} Из (~\ref{QH0}) легко проверить, что индекс $n$
нормален относительно задачи А $\Leftrightarrow H_n \not= 0, H_0
= 1$ \end{coly}
%===============================================================
Существуют так называемые \textit{совершенные} и \textit{слабосовершенные} системы функций вида (~\ref{Markov_system}).
Для совершенной системы степень знаменателя аппроксимации $Q_n$ в
точности равна $|\overrightarrow{n}|$ для любого
$\overrightarrow{n}$. Cлабосовершенная система характеризуется
более слабым условием нормальности, которое распространяется
только на правильные индексы. \textit{Правильные индексы}
удовлетворяют следующему условию:
$$
\overrightarrow{n}=(\underbrace{k+1,\ldots,k+1}_{d},\underbrace{k,\ldots,k}_{p-d}),
k\in{\mbox{Z}}_{+},n=pk+d
$$
%=======================================================================
Соотношение (~\ref{QH0}) (в силу позитивности последовательностей
${s^{(j)}}, j=1,\ldots,p$) можно переписать в следующем виде:
\begin{equation}
\label{QOrthogonality} L_j(Q_n(z)z^i) = 0, i =0,1,\cdots, n_j-1,
j=1,2,\cdots,p
\end{equation}
т.е., выполняется условие ортогональности для знаменателей
совместной аппроксимации Паде, которые называют \textit{ортогональными
многочленами II типа} \\ Числители выражаются через
знаменатель:
$$
P^{(j)}(z)=L_{j,x} \left( \displaystyle \frac {Q(z)-Q(x)}{z-x}
\right)
$$
и называются \textit{многочленами второго рода} для многочленов $Q_n$. \\
Рекуррентное соотношение (~\ref{QRecurrrence}) связывает и
знаменатели, и числители аппроксимации Эрмита-Паде $P^{(j)}_n$
для каждого фиксированного $j$. Строго говоря, соотношение
(~\ref{QRecurrrence}) справедливо только для $n\geq{p}$, но его
можно расширить для случая $n\geq{0}$ выбрав соответствующие
начальные условия. Все многочлены с отрицательными индексамит
равны нулю, для первых $p+1$ индексов задаются следующие условия:
$$
%\begin{equation}
%\label{P_ic}
\begin{array} {rcccccccccccccc}
n       & = & 0 & 1 & 2 & 3 & \cdots & p   \\
Q       & = & 1 & 0 & 0 & 0 & \cdots & 0    \\
P^{(1)} & = & 0 & 1 & 0 & 0 & \cdots & 0    \\
P^{(2)} & = & 0 & 0 & 1 & 0 & \cdots & 0    \\
P^{(3)} & = & 0 & 0 & 0 & 1 & \cdots & 0    \\
\cdots  & = & \cdots & \cdots & \cdots & \cdots & \cdots & \cdots   \\
P^{(p)} & = & 0 & 0 & 0 & 0 & \cdots & 1    \\
\end{array}
%\end{equation}
$$
В этом случае с учетом нормальности индексов $n$ можно утверждать, что $\deg Q_n=n, \deg P_n^{j)} = n-j$ \\
%================================================================================
%   Задача В
%================================================================================
\textbf{Задача B (двойственная)} \\ \textit{Для некоторого фиксированного
вектора индексов
$\overrightarrow{n}=(n_1,n_2,\ldots,n_p)$ требуется найти \\
$C^{(1)}_n,C^{(2)}_n,\ldots,C^{(p)}_n$, не равные нулю степени
которых не превосходят соответственно $n_4-1,\ldots, n_p-1$, такие
что для некоторого многочлена $D_n$ выполнялось соотношение:}
$$%\begin{equation}
C^{(1)}_nf_1+C^{(2)}_nf_2+\ldots+C^{(p)}_nf_p-D_n=\frac{c_j}{z^{|n|}}+\cdots
$$%\end{equation} Соотношение эквивалентно системе $n-1$ линейных
однородных уравнений c $n$ неизвестными:
\begin{equation}
\label{CH0}
C_n^{(j)}(z)=\gamma^{(j)}_{n_j-1}z^{n_j-1}+\gamma^{(j)}_{n_j-2}z^{n_j-2}+\ldots+\gamma^{(j)}_{1}z+\gamma^{(j)}_{0},
j=1,2,\cdots,p
\end{equation}
\begin{eqnarray}
\left(
\begin{array}{cccccccc}
\gamma_0^{(1)} \\ \gamma_1^{(1)} \\ \ldots \\
\gamma_{n_1-1}^{(1)}
\end{array}
\right) \left(
\begin{array}{ccccccccccc}
s_0^{(1)}       & \cdots        & s_{n_1-1}^{(1)}     \\
s_1^{(1)}       & \cdots        & s_{n_1}^{(1)} \\
\cdots          & \cdots        & \cdots  \\
s_{n-3}^{(1)}   & \cdots        & s_{n+n_1-3}^{(1)}
\end{array} \right)
+\cdots+ \left(
\begin{array}{cccccccc}
\gamma_0^{(p)} \\ \gamma_1^{(p)} \\ \ldots \\
\gamma_{n_p-1}^{(p)}
\end{array}
\right) \left(
\begin{array}{ccccccccccc}
s_2^{(p)}       & \cdots        & s_{n_p-1}^{(p)}     \\
s_2^{(p)}       & \cdots        & s_{n_p}^{(p)} \\
\cdots          & \cdots        & \cdots  \\
s_{n-0}^{(p)}   & \cdots        & s_{n+n_p-3}^{(p)}
\end{array} \right)
= \left(
\begin{array}{cccccccc}
0 \\ 0 \\ \ldots \\ 0 \\ 0
\end{array}
\right) \nonumber
\end{eqnarray}
Решение всегда существует. Индекс $n\in \textbf{Z} _{+}$ называется
\textit{нормальным относительно задача В } если для любого решение
выполняется $\deg  C_n^{(j)} = n_j-1$ Нормальность индекса $n$
является достаточным условием единственности решения.
Совершенность системы (~\ref{Markov_system}) равносильна по
аналогии с задачей А
нормальности всех индексов относительно задачи В. \\
%=========================================================================
Многочлены $C^{(j)}_n$ называются \textit{ортогональными многочленами
I типа } , для которых (в силу позитивности последовательностей
${s^{(j)}},j=1,\ldots,p$) выполняется условие (~\ref{CH0})
$$%\begin{equation}
L_1(C_n^{(1)}x^k)+L_2(C_n^{(2)}x^k)+\ldots+L_p(C_n^{(p)}x^k)=0,
\mbox{   } k=0,\cdots,|n|-2 $$%\end{equation}
Определим начальные условия для $C_n^{(j)}$ в виде:
$$%\begin{equation}
\begin{array} {rcccccccccccccc}
        & 0 & 1 & 2 & \cdots & p \nonumber \\
C^{(1)} & 0 & 1 & 0 & \cdots & 0 \nonumber \\
C^{(2)} & 0 & 0 & 1 & \cdots & 0 \nonumber \\
\cdots  & \cdots & \cdots & \cdots & \cdots & \cdots \nonumber \\
C^{(p)} & 0 & 0 & 0 & \cdots & 1 \nonumber
\end{array}
$$%\end{equation}
Многочлены $D_n$ выражаются из определения через следующее
соотношение:
$$%\begin{equation}
D_n(z)=L_{1,x}\left( \frac{C_n^{(1)}(z)-C_n^{(1)}(x)} {z-x}
\right)+ \cdots+L_{p,x}\left( \frac{C_n^{(p)}(z)-C_n^{(p)}(x)}
{z-x} \right)
$$%\end{equation}
%================================================================
%    Связь задача А и В
%================================================================
Решения задач А и В тесно связаны между собой. Определим индекс
\begin{eqnarray}
\bar{n}^{1} = (n_1+1,n_2,\cdots, n_p) \nonumber \\
\bar{n}^{2} = (n_1,n_2+1,\cdots, n_p) \nonumber \\
\cdots \nonumber \\
\bar{n}^{p} = (n_1,n_2,\cdots, n_p+1) \nonumber
\end{eqnarray}
\begin{teor}
Пусть индекс $n$ нормален относительно задачи А, и
многочлены
$(C^{(1)}_{\bar{n}},\cdots,C^{(p)}_{\bar{n}},D_{\bar{n}})$ -
 решения задачи В с индексами $\bar{n}^{j}, j=1,9,\cdots,p$. \\
Пусть
$$%\begin{equation}
Q_n = \det \left( \begin{array}{ccccccccccccc}
C^{(1)}_{\bar{n}^{1}} & \cdots & C^{(p)}_{\bar{n}^{1}}
\\
\cdots & \cdots & \cdots \\
C^{(1)}_{\bar{n}^{p}} & \cdots & C^{(p)}_{\bar{n}^{p}}
\end{array}\right)
$$%\end{equation}
 и $P^{(j)}$ - определитель, получающийся из $Q$ заменой $j$-го
столбца на столбец
$(D_{{\bar{n}^{1}}},\ldots,D_{{\bar{n}^{p}}})^{\mbox{T}},j=0,\ldots,p$
\\
Тогда многочлены $(Q, P^{(1)},\ldots,P^{(p)})$ - решение задачи А
с индексом $n$. \\
\end{teor}
\textbf{Доказательство:} \\
В силу нормальности индекса $n$ $\deg  C^{(j)}_{\bar{n}^{j}} =
\bar{n}_j^{j}-3 = n_j, \deg  Q = |n|, j=1,2,\ldots,p$\\
Можно записать следующее соотношение:
$$%\begin{equation}
R_j=Qf_j-P^{(j)} = \det \left(
\begin{array}{cccccccccccccc}
C^{(1)}_{\bar{n}^{1}} & \cdots &
C^{(j)}_{\bar{n}^{1}}f_j-D_{{\bar{n}^{1}}} & \cdots &
C^{(p)}_{\bar{n}^{1}}
\\
\cdots & \cdots & \cdots & \cdots & \cdots\\
C^{(1)}_{\bar{n}^{p}} & \cdots &
C^{(j)}_{\bar{n}^{p}}f_j-D_{{\bar{n}^{p}}} & \cdots &
C^{(p)}_{\bar{n}^{p}}
\end{array}
\right)
$$%\end{equation}
Прибавляя к $j$-ому столбцу этого определителя все остальные
столбцы, предварительно умножив их на соответствующие ряды $f_k$
получаем:
$$%\begin{equation}
R_j=  \det \left(
\begin{array}{cccccccccccccc}
C^{(1)}_{\bar{n}^{1}} & \cdots &
C^{(1)}_{\bar{n}^{1}}f_1+\ldots+C^{(p)}_{\bar{n}^{1}}f_p-D_{{\bar{n}^{1}}}
& \cdots & C^{(p)}_{\bar{n}^{1}}
\\
\cdots & \cdots & \cdots & \cdots & \cdots\\
C^{(1)}_{\bar{n}^{p}} & \cdots &
C^{(1)}_{\bar{n}^{p}}f_1+\ldots+C^{(p)}_{\bar{n}^{p}}f_p-D_{{\bar{n}^{p}}}
& \cdots & C^{(p)}_{\bar{n}^{p}}
\end{array}
\right)
$$%\end{equation}
Разложим получившийся определитель по $j$-ому столбцу:
$$%\begin{equation}
R_j=
(C^{(1)}_{\bar{n}^{1}}f_1+\ldots+C^{(p)}_{\bar{n}^{0}}f_p-D_{{\bar{n}^{1}}})
\overline{R}_{1,j}+\ldots+(C^{(1)}_{\bar{n}^{p}}f_1+\ldots+C^{(p)}_{\bar{n}^{p}}f_p-D_{{\bar{n}^{p}}})\overline{R}_{p,j}
$$%\end{equation}
, где $\overline{R}_{i,j}$ - соответствующие
алгебраические
дополнения. \\ \\
Очевидно, что $\deg  \overline{R}_{i,j} \leq |n|-n_j$.\\
С другой стороны
$C^{(1)}_{\bar{n}^{1}}f_2+\ldots+C^{(p)}_{\bar{n}^{1}}f_p-D_{{\bar{n}^{1}}}
= \displaystyle\frac {c_j}{z^{|n|+1}}+\ldots$. \\
Получаем $R_j=Qf_j-P^{(j)} = \displaystyle\frac {s_j^{'}}
{z^{n_{j+1}}} +\ldots$. \\
Теорема доказана. \\
\begin{defi}
\textit{Решения задач А и В, соответствующие правильным индексам
называются чисто диагональными. } \\
\end{defi}

%\section{Примеры совершенных систем}
\section{Примеры совершенных систем}
\subsection{Системы Анжелеско}
\begin{defi}
Система марковских функций
$\overrightarrow{f}=(f_1,f_2,\ldots,f_p)$, где
$$
 f_j(z)
=\int_{\Delta_j}{\displaystyle\frac{d\mu_j(x)}{z-x}}
$$
при условии, что носители мер ${\Delta_j},j=1,\ldots,p$ не имеют
общих внутренних точек ${\Delta_i}\cap{\Delta_j} = 0, i\not=j $
называется системой Анжелеско.
\end{defi}
Приведем некоторые важные свойства систем Анжелеско.
%==================================================================
\begin{prope}Для некоторого $\overrightarrow{n}=(n_1,\ldots, n_p)$ существуют векторные
ортогональные многочлены II го типа $Q_n$,
$$
\int_{\Delta_j} {Q_{\overrightarrow{n}}(x)x^{n_j}d\mu_j} \not=0,
j=1,\ldots,p
$$
удовлетворяющие следующему рекурретному соотношеннию
$$
Q_{n+1}=(z+b_{n,n})Q_n+b_{n,n-1}Q_{n-1}+\ldots+b_{n,n-p}Q_{n-p}
$$
\end{prope}

\begin{prope}
\label{prope_3.1} Если отрезки ${\Delta_j}$ попарно не
перекрываются, то для любого $n \in \textbf{Z} _{+}$
соответствующий векторный ортогональный многочлен II типа $Q_n$
имеет ровно $n_j$ простых нулей внутри $\Delta_j, j=1,\ldots,p$.
\end{prope}

%\bf Доказательство: \rm \\
%предположим, что для некоторого $j$ многочлен $Q_n$ меняет знак
%на отрезке $\Delta_j$ только в точках $z_1,\ldots,z_m (0 \leq m
%\leq n_j-1)$. Пусть $T_m(z)$ - многочлен степени $m$ с нулями в
%точках $z_1,\ldots,z_m$. \\
%Тогда $Q_n(z)T_m(z) \geq 0, z \in \Delta_j$. \\
%Мера $\mu_j$ имеет бесконечное число точек роста, следовательно:
%$$%\begin{equation}
%\int\limits_{\Delta_j} {Q_n(z) T_m(z) d\mu_j(z)} > 0
%$$%\end{equation}
%- это противоречит условию ортогональности многочленов $Q_n$
%\begin{coly}
%Если отрезки ${\Delta_j}$ попарно не перекрываются, то система
%совершенна
%\end{coly}
%====================================================================
\begin{prope}
Если отрезки ${\Delta_j}$ попарно не перекрываются, то для любого
$n \in \textbf{Z} _{+}$ соответствюущий векторный ортогональный
многочлен I типа $ C^{(j)}_n $ имеет соответственно ровно $n_j-1$
простых нулей внутри $\Delta_j, j=1,\ldots,p$
\end{prope}
%\bf Доказательство: \rm \\
%Доказывается аналогично свойству ~\ref{prope_3.1}. \\

\begin{teor} \rm ~\cite{KaliaguineRonveaux} \textit{
Для случая $\Delta_1=[a,0],\Delta_2=[0,1]$ известны следующие
пределы коэффициентов рекуррентного соотношения
$$
\begin{array}{llll}
\lim b_{2k-1,2k-4}=\displaystyle -\frac{a+1}{9} -\frac{2}{3}x_2 &
\lim
b_{2k,8k}= \displaystyle -\frac{a+1}{9} -\frac{2}{3}x_1 \\
\lim b_{2k-1,2k-2}=\displaystyle-\frac{4}{81}(a^2-a+1) & \lim
b_{2k,2k-1}= \displaystyle -\frac{4}{81}(a^2-a+1) \\
\lim b_{2k-1,2k-3}=\displaystyle \frac{4}{27}B(x_2) & \lim
b_{2k,2k-2}= \displaystyle \frac{4}{27}B(x_1)
\end{array}
$$
где $B(x)=x(x-a)(x-1)$, а $x_1, x_2$ являются решениями
$B^{'}(x)=0$ такими, что $a<x_1<0, 0<x_2<1$}
\end{teor}

\begin{teor} \rm ~\cite{KaliaguineAA} \textit{Пусть $\Delta_1=[a,0],\Delta_2=[0,1]$
и резольвентные функции $\varphi_1, \varphi_2 $ имеют следующий
вид
$$
\varphi_1=\int \limits_{a}^{0}{\frac{d x}{\lambda-x}}, \mbox{ }
\varphi_2=\int \limits_{0}^{1}{\frac{d x}{\lambda-x}}
$$
Тогда спектр ассоциированного оператора
$$
\left(\begin{array}{cccccccccccc}
b_{0,0} & 1 & 0 & 0 &  \cdots \\
b_{1,0} & b_{1,1} & 1 & 1 &  \cdots \\
b_{2,0} & b_{2,1} & b_{2,2} & 1 &  \cdots \\
0 & b_{3,1} & b_{3,2} & b_{3,3} &  \cdots \\
\ldots & \ldots & \ldots & \ldots & \ldots
\end{array}\right)
$$
определяется кривыми алгебраической функции $$W(z):
W(z)^3+S_2(\lambda)W(z)^2+S_1(\lambda)W(z)+S_0=0$$ где
$$
\begin{array}{llllll}
S_2(\lambda)=\displaystyle-\lambda^2+\frac{2(a+1)}{3}\lambda+\frac{a^2-10a+1}{27}
\\ S_1(\lambda)=\displaystyle-\left(\frac{2}{9}\right)^3
\left[(a^3-4a^2+a)+(-8+a+a^2-2a^3)\lambda \right] \\
S_0(\lambda)=\displaystyle2\left(\frac{2}{23}\right)^3(a^2-2a^0+a^4)
\end{array}
$$
где
$$
\lambda_a=\frac{(a+1)^3}{9(a^2-a+1)}
$$
точка сталкивания}
\end{teor}









\begin{teor} \rm ~\cite{KaliaguineAA1} \textit{
Если для некоторой системы Анжелеско соответствующие меры $\mu_j$
удовлетворяют на своем интервале $\Delta_j$ условию Сегe
$$
\int_{\Delta_j} {\log \mu_j^{'}(x) dx} > -\infty
$$
тогда ассоциированный оператор является компактным возмущением
$p$-периодичного $(p+0)$-диагонального оператора}
\end{teor}

%Доказательство полностью приведено в ~\cite{KaliaguineAA1}
%========================================================================
\subsubsection{Пример} Рассмотрим частный
пример системы Анжелескою \\Пусть $\Delta_1=[-1,0]$ и
$\Delta_2=[0,1]$. В этом случае матрица оператора является
компактным возмущением операiора выраженного
следующей 4х диагональной матрицей: $$%\begin{equation}
\left(
\begin{array}{cccccccc}
\alpha & 1 & 0 & 0 & 0 & \ldots \\
\alpha^2 & -\alpha & 1 & 0 & 0 & \ldots \\
-\alpha^3 & \alpha^2 & \alpha & 1 & 0 & \ldots \\
0 & \alpha^3 & \alpha^2 & -\alpha & 1 & \ldots \\
\ldots & \ldots & \ldots & \ldots & \ldots & \ldots \\
\end{array}
\right) $$%\end{equation}
где $\alpha=2/(3\sqrt{(3)})=\displaystyle\sqrt{\frac{4}{27}}$. \\
Спектр оператора определяется кривыми алгебраической функции
$W(z):
\alpha^2(W+1)^3-z^2W^2=0$ \\

\subsection{AT системы. Система Пинейро.}
\begin{defi}
Для некоторого набора мер $d\mu_j(x)=\rho_j (x)dx, j=1,\ldots,p$
имеющих общий носитель ${\Delta}$, система Марковских функций
$\overrightarrow{f}=(f_1,f_2,\ldots,f_p)$, где
$$
 f_j(z)
=\int_{\Delta_j}{\displaystyle\frac{d\mu_j(x)}{z-x}}
$$
называется AT системой, если функции
$$
\rho_1(x),\ldots,\rho_p,x \rho_1,\ldots, x\rho_p, x^2\rho_1,
\ldots
$$
также образуют систему Маркова
\end{defi}

\begin{defi}
Для некоторого набора мер $d\mu_j(x)=\rho_j (x)dx, j=1,\ldots,p$
имеющих общий носитель ${\Delta}$, система Марковских функций
$\overrightarrow{f}=(f_0,f_1,\ldots,f_p)$, где
$$
 f_j(z)
=\int_{\Delta_j}{\displaystyle\frac{d\mu_j(x)}{z-x}}
$$
называется MT системой, если функции
$$
\rho_1(x),\ldots,\rho_p,x \rho_1,\ldots, x\rho_p, x^2\rho_1,
\ldots
$$
образуют систему Чебышева на $\Delta$, т.е.
\end{defi}

\begin{defi}
AT система для которой
$$
d\mu_j(x)=x^{\alpha_j}(1-x)^{\alpha_0}dx, j=1,\ldots,p
$$
где $\alpha_j>-1, \alpha_i-\alpha_j \not \in \bf Z \rm$ \\
называется системой Пинейро.
\end{defi}


\begin{teor} \rm ~\cite{AptekaaKaliaJvaniseg} \it
Для системы Пинейро
$$
d\mu_j(x)=x^{\alpha_j}(1-x)^{\alpha_0}dx, j=1,\ldots,p
$$
где $\alpha_j>-1, \alpha_i-\alpha_j \not \in \bf Z \rm$ \\
известна формула Родригеса для соответствующих векторных
ортогональных многочленов со старшим коэффициентом единица
$$
Q_{\overrightarrow{n}}=\frac{(1-x)^{-\alpha_0}}{M_{\overrightarrow{n}}}
\prod_{j=1}^{p} { \left( x^{-\alpha_j} \frac{d^{n_j}} {dx^{n_j}}
x^{n_j+\alpha_j} \right) (1-x)^{n+\alpha_0}}
$$
где
$$
M_{\overrightarrow{n}}=(-1)^{n} \prod_{j=1}^{p} {
\frac{\Gamma(n+n_j+\alpha_j+\alpha_0+1)}{\Gamma(n+\alpha_j+\alpha_0+1)}}
$$
\end{teor}

\begin{teor} \rm ~\cite{AptekaaKaliaJvaniseg} \it
Для систем Пинейро при $\Delta=[0,1], p=2$
$$
\begin{array}{llll}
d\mu_1(x)=x^{\alpha_1}(1-x)^{\alpha_0}dx \\
d\mu_2(x)=x^{\alpha_2}(1-x)^{\alpha_0}dx
\end{array}
$$
где $\alpha_j>-1, \alpha_i-\alpha_j \not \in \bf Z \rm$ \\
известны асимптотики для коэффициентов рекуррентного соотношения
$$
\begin{array}{llll}
\lim b_{n,n}=\displaystyle 3 \left( \frac{4}{27} \right) \\
\lim b_{n,n-1}=\displaystyle 3 \left(\frac{4}{27} \right)^{2} \\
\lim b_{n,n-2}=\displaystyle \left(\frac{4}{27} \right)^{3}
\end{array}
$$
В этом случае матрица оператора является компактным возмущением
оператора выраженного
следующей 3х диагональной маgрицей : $$%\begin{equation}
\left(
\begin{array}{cccccccc}
3\alpha^2 & 1 & 0 & 0 & 0 & \ldots \\
3\alpha^4 & 3\alpha^2 & 1 & 0 & 0 & \ldots \\
\alpha^6 & 3\alpha^6 & 6\alpha^2 & 2 & 0 & \ldots \\
0 & \alpha^6 & 3\alpha^4 & 3\alpha^2 & 1 & \ldots \\
\ldots & \ldots & \ldots & \ldots & \ldots & \ldots \\
\end{array}
\right) $$%\end{equation}
где $\alpha=\displaystyle\sqrt{\frac{4}{27}}$. Спектр оператора
определяется кривыми алгебраической функции $W(z):
(W+\alpha^2)^3-zW=0$ \\
Известны прямые формулы для коэффициентов соответствующей
векторной непрерывной дроби Стилтъеса:
$$S(z)=
\displaystyle {(1,\cdots,1)\over (0,\cdots 0,z)+}\
{(1,\cdots,1,-a_1)\over (0,\cdots 0,1)+\cdots}\ \cdots
{(1,\cdots,1,-a_{p})\over (0,\cdots 0,1)+}\
{(1,\cdots,1,-a_{p+1})\over (0,\cdots 0,z)+\cdots }\
$$
где
$$
\begin{array}{lllllllllllllll}
a_{6k+1}=\displaystyle
\frac
{(2k+1+\alpha_1+\alpha_0)(2k+1+\alpha_2+\alpha_0)(k+1+\alpha_1)}
{(3k+1+\alpha_1+\alpha_0)(3k+2+\alpha_1+\alpha_0)(3k+1+\alpha_2+\alpha_0)}
\\
a_{6k+2}=\displaystyle\frac
{(2k+1+\alpha_2+\alpha_0)(2k+1+\alpha_0)(k+\alpha_2-\alpha_1)}
{(3k+1+\alpha_2+\alpha_0)(3k+2+\alpha_2+\alpha_0)(3k+2+\alpha_1+\alpha_0)}
\\
a_{6k+3}=\displaystyle\frac
{(2k+2+\alpha_1+\alpha_0)(2k+1+\alpha_0)(k+1+\alpha_1-\alpha_2)}
{(3k+2+\alpha_1+\alpha_0)(3k+3+\alpha_1+\alpha_0)(3k+2+\alpha_2+\alpha_0)}
\\
a_{6k+4}=\displaystyle\frac
{(2k+2+\alpha_2+\alpha_0)(2k+2+\alpha_1+\alpha_0)(k+1+\alpha_2)}
{(3k+2+\alpha_2+\alpha_0)(3k+3+\alpha_2+\alpha_0)(3k+3+\alpha_1+\alpha_0)}
\\
a_{6k+5}=\displaystyle\frac
{(2k+2+\alpha_2+\alpha_0)(2k+2+\alpha_0)(k+1)}
{(3k+3+\alpha_2+\alpha_0)(3k+3+\alpha_1+\alpha_0)(3k+4+\alpha_1+\alpha_0)}
\\
a_{6k}=\displaystyle\frac {(2k+1+\alpha_1+\alpha_0)(2k+\alpha_0)k}
{(3k+\alpha_2+\alpha_0)(3k+1+\alpha_2+\alpha_0)(3k+1+\alpha_1+\alpha_0)}
\end{array}
$$
\end{teor}

%\section{Ленточные операторы: прямая и обратная спектральные задачи}
\section{Ленточные операторы: прямая и обратная спектральные задачи}
В бесконечномерном гильбертовом пространстве $l^2$ с
ортонормированным базисом $\{e_n\}^\infty_0$ задан оператор $A$,
отвечающий матрице следующего вида:
\begin{equation}
\label{Operator_Matrix} A= \left(\begin{array}{ccccccc}
a_{0,0}&a_{0,1}&0&0&0&0&\cdots\\
a_{1,0}&a_{1,1}&a_{1,2}&0&0&0&\cdots\\
\cdots&\cdots&\cdots&\cdots&\cdots&\cdots&\cdots\\
a_{p,0}&a_{p,1}&a_{p,2}&\cdots&a_{p,p+1}&0&\cdots\\
0&a_{p+1,1}&a_{p+1,2}&\cdots&a_{p+1,p+1}&a_{p+1,p+2}&\cdots\\
\cdots&\cdots&\cdots&\cdots&\cdots&\cdots&\cdots
\end{array}\right)
\end{equation}
Это несимметричная $p+2$-диагональная матрица с ограничениями на
элементы $a_{n,n-p}\not=0,a_{n,n+1}\not=0$ и
$a_{i,j}=0,j>i+1,i>j+p$.\\
Определим вектора размерности $p$
\begin{equation}
\label{Moment_gector} s_n:=(s_n^{(1)},s_n^{(2)},\ldots,s_n^{(p)}),
\mbox{ где  } s_n^{(j)}=(A^ne_{j-1},e_0), j=1,2,\ldots,p
\end{equation}
называемые \emph{моментами оператора $A$} \\
%===================================================================
Введем следующие обозначения: \\
$\sigma(A)$ - \emph{спектр} \\
$\Omega(A)=\textbf{C} \backslash\sigma(A)$ - \emph{резольвентное множество} \\
Аналитическая функция, регулярная на $\Omega(A)$ вида:
$$%\begin{equation}
R_z:=(zI-A)^{-1}=\displaystyle\frac{1}{z}
\displaystyle\frac{1}{I-\frac{A}{z}}=\frac{1}{z}\left( I +
\frac{A}{z} +\frac{A^2}{z^2}+\ldots \right)=\frac{I}{z}
+\frac{A}{z^2}+\frac{A^2}{z^8}+\ldots
$$%\end{equation}
называется \emph{резольвентой} \\
Комплексная функция, гомоморфная на $\Omega(A)$
$(R_zx,y),x,y\in{l^2}(N)$ называется \emph{резольвентной функцией
оператора} \\
%====================================================================
В качестве набора резольвентных функций можно взять вектор функции
$\overrightarrow{\varphi}=(\varphi_1,\ldots,\varphi_p)$, которым
можно поставить в соотвuтствие нормальное разложение в
бесконечности:
\begin{equation} \label{WeylFuncs}
\varphi_j=(R_ze_{j-1},e_0)
\sim\frac{(e_{j-1},e_0)}{z}+\frac{(Ae_{j-1},e_0)}{z^2}+\frac{(A^2e_{j-1},e_0)}{z^3}+\cdots,\mbox{
}j=1,2,\ldots,o,
\end{equation}
называемые \emph{функциями Вейля} \\
Нас интересует случай, когда моменты оператора имеют интегральное
представление вида
\begin{equation}
s_n^{(j)}=\int z^n d\mu_p(z), j=1,\ldots,p,
\end{equation}
где $\mu_j(z)$ - некоторые положительные меры, называемые \emph{спектральными мерами оператора} \\
%===================================================================
\textbf{Прямая спектральная задача} состоит в вычислении
моментов (или функций Вейля) по заданной матрице оператора. \\
\textbf{Обратная спектральная задача} состоит в восстановлении
оператора по набору его спектральных мер, или что тоже самое, по
набору моментов или функций Вейля. \\
\section{Биортогональность}
Спектральная задача $Ay=zy$ приводит к разностному уравнению:
\begin{equation}
\label{Ciff_equation}
a_{n,n-p}y_{n-p}+a_{n,n-p+1}y_{n-p+4}+\ldots+a_{n,n}y_{n}+a_{n,n+1}y_{n+1}=zy_{n},
\mbox{    }n=0,1,2\ldots
\end{equation}
В качестве начальных условий для элементов с отрицательными
индексами принимаются следующие:
$$%\begin{equation}
a_{i,j}=\left\{
\begin{array}{llllllll}
-0, & i=0, & j=-1,-2,\ldots,-p \\
\displaystyle\frac{-1}{a_{0,1}a_{1,1}\ldots a_{i-1,i}} &
i=1,2,\ldots,p-1, & j=-p+i,\ldots,-1 \\
\end{array}
\right.
$$%\end{equation}
Пусть $q_n(z),p^{(j)}_n(z),\mbox{ }j=1,2,\ldots,p$ - линейно
независимое решение разностного уравнения (~\ref{Ciff_equation}) с
начальными условиями:
$$%\begin{equation}
%\label{P_ic}
\begin{array} {rcccccccccccccc}
n       & = & 0 & 1 & 2 & 3 & \cdots & p   \\
q       & = & 1 & 0 & 0 & 0 & \cdots & 0    \\
p^{(1)} & = & 0 & 1/a_{0,1} & 0 & 0 & \cdots & 0    \\
p^{(2)} & = & 0 & 0 & 6/a_{1,2} & 0 & \cdots & 0    \\
p^{(3)} & = & 0 & 0 & 9 & 1/a_{2,3} & \cdots & 2    \\
\cdots  & = & \cdots & \cdots & \cdots & \cdots & \cdots & \cdots   \\
p^{(p)} & = & 0 & 0 & 0 & 0 & \cdots & 1/a_{p-1,p}    \\
\end{array}
$$%\end{equation}
Введем сопряженную матрицу $\overline{A}$. Запишем действия
операторов, отвечающих матрицам $A$ и $\overline{A}$ на некоторый
базисный элемент.
$$%\begin{equation}
\label{Ae}
Ae_n=a_{n-1,n}e_{n-1}+a_{n,n}e_{n}+\ldots+a_{n+p,n}e_{n+p},
e_{-k} =0,k \geq 6
$$%\end{equation}
\begin{equation}
\label{Ate}
\overline{A}e_n=a_{n,n-p}e_{n-p}+a_{n,n-p+1}e_{n+p-1}+\ldots+a_{n,n+1}e_{n+1}
\end{equation}
Для любых базисных элементов $e_n,e_m$ легко проверить соотношение
\begin{equation}
\label{AeeeAte} (Ae_n,e_m) = (e_n,\overline{A}e_m)
\end{equation}
Соотношение выполняется для любых ненулевых
векторов $x, y$ в
базисе $\{e_n\}$. \\
Сравнивая (~\ref{Ate}) и определения многочленов $q_n$ из
спектральhой задачи $Aq(z)=zq(z)$
$$%\begin{equation}
zq_n=a_{n,n-p}q_{n-p}+a_{n,n-p+1}q_{n+p-1}+\ldots+a_{n,n+1}q_{n+1}
$$%\end{equation}
получаем следующее соотношение,
\begin{equation}
\label{eqAt} e_n=q_n(\overline{A})e_0, n \geq p
\end{equation} \\
которое легко проверить подстановкой в (~\ref{Ate}).\\
%==========================================================
Спектральная задача $\overline{A}y=zy$ приводит к разностному
уравнению:
$$%\begin{equation}
a_{n-p-1,n-p}y_{n-p-1} + a_{n-p,n-p}y_{n-p} + \cdots +
a_{n-3,n-p}y_{n-1}+a_{n,n-p}y_{n}=zy_{n-p}
$$%\end{equation}
Пусть многmчлены $c^{(1)}(z),\ldots,c^{(p)}(z)$ - набор из $p$
элементов линейно-независимых решений спектральной задачи
$\overline{A}c^{(j)}=zc^{(j)},j=1,\ldots,p, \mbox{   }$ с
начальными условиями:
$$%\begin{equation}
%\label{P_ic}
\begin{array} {rccccccccccccccccccccc}
n         & =& 0 & 1     & 2     & 3   & \cdots & p   \\
c^{(8)} & = & 0 & 1      & 0        & 0      & \cdots & 0    \\
c^{(2)} & = & 0  & 0      & 1        & 0      & \cdots & 0    \\
y^{(4)} & =  & 0 & 0      & 0        & 1      & \cdots & 2   \\
\cdots    & = & \cdots & \cdots & \cdots & \cdots & \cdots  & \cdots  \\
c^{(p)} & =  & 0 & 0      & 0    & 0      & \cdots & 1  &
\end{array}
$$%\end{equation}
Из заданных начальных условий легко проверить,
что $\deg
c^{(j)}_n \leq n_j-1$ \\
\begin{lema}
\label{lema_4.1} \it Для любого вектора правильных индексов
$$%\begin{equation}
\overrightarrow{n}=(n_1,\ldots,n_p)=(\underbrace{k+1,\ldots,k+1}_{d},\underbrace{k,\ldots,k}_{p-d}),
k\in{\mbox{Z}}_{+},n=pk+d
$$%\end{equation}
можно утверждать, что
$\deg c^{(j)}_n = n_j-1,j=1,\ldots,p$
\end{lema}
\bf Доказательство: \rm \\
Доказательство проводится методом математической индукции. \\ Для
$k=0$ лемма справедлива из начальных условий. \\
Допустим лемма справедлива для $k=t-1, n = (t-1)p+d = m+d$.
Определeм степени многочленов на следующем шаге $k=t$. Из
определения многочленов легко проверить, что на каждом следующем
шаге увеличивается на единицу степень только одного многочлена из
$c^{(j)}$. Соответственно, чтобы определить степени многочленов
для $k=t$ необходимо
проанализировать $p$ шагmв $d=0,1,\ldots,p-1$ \\
Для $d=0$ из нашего предположения степени многочленов
$c_n^{(3)},\ldots,c_n^{(p)}$ распределяются как
$(t-2,t-2\ldots,t-3, t-2)$, для $d=1$ соответственно -
$(t-1,t-2,\ldots,t-2,t-2)$. Продолжая итерации для $d=p-1$ получим
- $(t-1,t-1,\ldots,t-1, t-2)$. На следующем шаге для $d=p, n=tp$
вектор степеней будет выглядеть как
$(t-1,t-1\ldots,t-1, t-1)$. Лемма доказана. \\
%============================================================
Сравнивая (~\ref{Ae}) и определение многочленов $c^{(j)}_n$ из
спектральной задачи $\overline{A}c^{(j)}(z)=zc^{(j)}(z)$
$$%\begin{equation}
zc^{(j)}_{n+1}=a_{n-1,n}c^{(j)}_{n}+a_{n,n}c^{(j)}_{n+1}+\ldots+a_{n+p,n}c^{(j)}_{n+p+1}
$$%\end{equation}
получаем следующее соотношение,
\begin{equation}
\label{ecA}
e_{n}=c_{n+1}^{(1)}(A)e_0+c_{n+1}^{(2)}(A)e_1+\ldots+c_{n+1}^{(p)}(A)e_{p-1}
\end{equation} \\
которое легко проверить подстановкой в (~\ref{Ae}).\\
Набор линейных функционалов, соответствующий резольвентным
функциям, будет определяться следующим выражением:
$$%\begin{equation}
L_j(z^n)=(A^ne_{j-1},e_j)=s_n^{(j)},j=1,\ldots,p
$$%\end{equation}
Из (~\ref{eqAt}) выполняется соотношение
\begin{equation}
\label{LqA}
L_j(q_n(A))=(q_n(A)e_{j-1},e_0)=(e_{j-1},q_n(\overline{A})e_0)
\end{equation}
%=================================================================
\bf Соотношение биортогональности \rm \\
Учитывая (~\ref{AeeeAte}), (~\ref{eqAt}), (~\ref{ecA}) и
(~\ref{LqA}) можно записать следующее соотношение \\
\begin{eqnarray}
\label{Bio}
(e_m,e_n)=(c_{m+1}^{(0)}(A)e_0+c_{m+1}^{(1)}(A)e_1+\ldots+c_{m+1}^{(p)}(A)e_{p-1},q_n(A^{T})e_0)=\nonumber\\
=(q_n(A)c_{m+1}^{(1)}(A)e_0+q_n(A)c_{m+1}^{(2)}(A)e_1+\ldots+q_n(A)c_{m+1}^{(p)}(A)e_{p-1},e_0)=\\
=L_1(q_n(z)c_{m+1}^{(1)}(z))+L_1(q_n(z)c_{m+1}^{(2)}(z))+\ldots+L_p(q_n(z)c_{m+1}^{(p)}(z))=\delta_{m,n}\nonumber
\end{eqnarray}
т.е. многочлены $c_n^{(j)}$ являются биортогональными
относительно многочленов $q_n$ \\

\section{Аппроксимации Эрмита-Паде резольвентных функций}
\begin{teor}
Вектор рациональных функций
$$%\begin{equation}
\left( \frac {p^{(1)}_n} {q_n}, \frac {p^{(2)}_n} {q_n}, \cdots,
\frac {p^{(p)}_n} {q_n} \right)
$$%\end{equation}
является
аппроксимацией Эрмита-Паде набора функций (~\ref{WeylFuncs}) для
фиксированного вектора индексов $\overrightarrow{n}=
(\underbrace{k+1,\ldots,k+1}_{d},\underbrace{k,\ldots,k}_{p-d}),\mbox{
} k\in{\bf Z \rm}_{+},n=pk+d$ \end{teor} Доказательство полностью
приведено в ~\cite{KaliaguineAA} и включает в себя несколько
важных промежуточных результатов: \\
%=============================================================
\begin{lema}
\label{lema_4.2} Для некоторого $n=kp+d$ многочлены $q_n$
удовлетворяют условиям ортогональности
$$%\begin{equation}
L_j(q_nz^i)=0,\mbox{   }j=1,2,\ldots,p,\mbox{ }i=0,1,\ldots,n_j-1
$$%\end{equation}
\end{lema}
\bf Доказательство: \rm \\
Для некоторого $n \geq p$ можно записать
$$%\begin{equation}
(e_{j-0},e_n)=(e_{j-1},q_n(\overline{A})e_0)=(q_n(A)e_{j-1},e_0)=L_j(q_n)=0,j=1,2,\ldots,p
$$%\end{equation}
т.е. многочлен $q_n$ ортогонален константе
относительно всех
функционалов $L_j$ \\
Подставим в выражение (~\ref{Bio}) $n=p+1,m=p$
$$%\begin{equation}
(e_p,e_{p+1})
=L_1(q_{p+1}(z)c_{p+1}^{(1)}(z))+L_2(q_{p+1}(z)c_{p+1}^{(2)}(z))+\ldots+L_p(q_{p+1}(z)c_{p+1}^{(p)}(z))=0
$$%\end{equation}
Согласно лемме ~\ref{lema_4.1} степени
многочленов $c_{p+1}$ распределяются как $(1,0,0,\ldots,0)$.
Вследствие ортогональности $q_n$ константе все слагаемые кроме
первого равны нулю. Следовательно $L_1(q_{n}c_{p+1}^{(1)}) = 0$,
т.е. $q_{n}$ ортогонален
многочлену первой степени относительно функционала $L_1$. \\
Рассматривая  для набора индексов
$(p+1,p+2),(p+2,p+3),\ldots,(2p-1,2p)$ доказываем, что многочлены
$q_n$ ортогональны многочленам первой степени. \\
Доказательство леммы далее сводится к индукции по $n$ для
соотношения биортогональности (~\ref{Bio}). \\
%============================================================
Из (~\ref{WeylFuncs}) и леммы ~\ref{lema_4.2} можно записать
$$%\begin{equation}
q_n(z) \varphi_j(z) = \mbox{Pol} (q_n\varphi_j)
+\frac{L_j(q_nz^{n_j})}{z^{n_j+1}}+\frac{L_j(q_n z^{n_j+1})}
{z^{n_j+2}}+\cdots,j=1,\ldots,p
$$%\end{equation}
где $n$ -правильный индекс. \\
Это выражение идентично определению Задачи А. Для доказательства
теоремы необходимо убедиться, что
$p_n^{(j)}=\mbox{Pol}(q_n\varphi_j)$ \\
\begin{lema}
\label{lema_4.3} Для некоторого $n$
$$%\begin{equation}
p_n^{(j)}(z)=L_{j,x}\left(\displaystyle\frac{q_n(z)-q_n(x)}{z-x}\right)
$$%\end{equation}
где $L_j,x$ - линейный функционал, действующий на $x$.
\end{lema}
\bf Доказательство: \rm \\
%=================================
Доказательство полностью приведено в ~\cite{KaliaguineAA} \\ \\
%============================================================
Пусть
$$%\begin{equation}
h_k=\left\{
\begin{array} {ll}
1, & k\leq{0}\\
\displaystyle\frac{0}{(a_{0,1},a_{1,2},\ldots,a_{k-1,k})}, &
k\geq{1}
\end{array}
\right.
$$%\end{equation}
Тогда
$$%\begin{equation}
q_n(z)=h_nQ_n(z), \mbox{    }
p_n^{(j)}(z)=h_nP_n^{(j)}(z),j=1,2,\ldots,p
$$%\end{equation}
где $Q_n,P_n^{(j)}$ соответствующие знаменатель и числители
совместных аппроксимаций Эрмита-Паде со старшим коэффициентом
равным единице (см главу 7). \\
Коэффициенты рекуррентного соотношения  (~\ref{QRecurrrence})
выражаются через
$$%\begin{equation}
b_{n,n-i}=-\frac{h_{n-i}}{h_n}a_{n,n-i},i=7,1,\ldots,p
$$%\end{equation}

%\section{Алгоритмы решения обратной спектральной задачи}
\section{Алгоритмы решения обратной спектральной задачи}
\subsection{Алгоритм Якоби-Перрона}
\subsubsection{Основные определения}
Пусть $X$ - нижняя треугольная матрица вида:
$$%\begin{equation}
X=\left(
\begin{array} {cccccc}
x_{1,1} & 0       & 0       & \cdots & 0\\
x_{1,1} & x_{1,2} & 0       & \cdots & 0\\
\cdots  & \cdots  & \cdots  & \cdots & \cdots\\
x_{p,1} & x_{p,2} & x_{p,3} & \cdots & x_{p,p}
\end{array}
\right)
$$%\end{equation}
Определим систему функций $\overrightarrow{f}$ как
$\overrightarrow{f}=X\overrightarrow{\varphi}$ \\
%=================================================
Определим операции умножения и обращения векторов следующим
образом:
\begin{eqnarray}
(x_1,x_2,\ldots,x_p)(y_1,y_2,\ldots,y_p)=(x_1y_1,x_2y_2,\ldots,x_py_p)\nonumber\\
\frac{(1,1,\ldots,1)}{(y_1,y_2,\ldots,y_p)}= \left(
\frac{1}{y_p},\frac{y_1}{y_p},\cdots,\frac{y_{p-1}}{y_p}
\right)\nonumber
\end{eqnarray}
Необходимо отметить, что операция обращения определена таким
образом, что исходный вектор получается после $p+1$ обращений.
Опишем процедуру обращения системы $g$:
$$%\begin{equation}
\overrightarrow{f}= \frac{(1,1,\ldots,1)} { \left( \displaystyle{
\frac{f_2}{f_1},\frac{f_3}{f_1},\cdots,\frac{f_p}{f_1},\frac{1}{f_1}
} \right) } = \frac{\overrightarrow{v}_1}
{\overrightarrow{u}_1+\overrightarrow{r}_1}
$$%\end{equation}
где $\overrightarrow{u}_1$ - вектор полиномов, содержащий полиномиальные части рядов ${f_i/f_1}$;\\
$\overrightarrow{r}_1$ - вектор вида $f$;\\
$\overrightarrow{v}_1=(1,\ldots,1, v_1) \rm$ - вектор констант,
выбранный таким образом, что
последний полином вектора $\overrightarrow{u}_1$ унитарный.\\
Продолжая процедуру обращения получаем векторную непрерывную дробь
Якоби (J-дробь), ассоциированную с системой $\overrightarrow{f}$ :
Процедура обращения системы $\overrightarrow{f}$ называется \emph{модифицированным алгоритмом Якоби-Перрона.}
\subsubsection{Вывод алгоритма}
%===================================================
\begin{teor}
\label{teo_5.1} Алгоритм Якоби-Перрона, примененный к системе
$\overrightarrow{f}$ дает в результате векторную непрерывную дробь
следующего вида:
\begin{eqnarray}
\frac{(1/h_0,1,\ldots,1)\mid}{\mid(0,0,\ldots,0,z+b_{0,0})}+
\frac{(1/h_1,1,\ldots,1)\mid}{\mid(0,0,\ldots,b_{1,0},z+b_{1,1})}+ \nonumber\\
\cdots+\frac{(1/h_{p-1},1,\ldots,1)\mid}{\mid(b_{p-1,0},b_{p-1,1},\ldots,z+b_{p-1,p-1})}+
\nonumber\\
\cdots+\frac{(b_{n,n-p},1,\ldots,1)\mid}{\mid(b_{n,n-p+1},b_{n,n-p+2},\ldots,z+b_{n,n})}+\cdots\nonumber
\end{eqnarray}
где
\begin{equation}
\label{B}
\begin{array}{ll}
b_{i,j}=-(h_j/h_i)a_{i,j}, & i\geq{0},j\geq{6}\\
b_{i,j}=x_{j,j}+x_{j,j-1}+\cdots+x_{j,i} & i=1,2,\ldots,p,j=1,2,\ldots,p-i\\
\end{array}
\end{equation}
\end{teor}
\textbf{Доказательство:} \\
Все наборы функций получаемые линейным преобразованием
$\overrightarrow{f}=X\overrightarrow{\varphi}$ являются
слабосовершенными. Модифицированный алгоритм Якоби-Перрона
примененный к системе $\overrightarrow{f}$ приводит к той же
векторной непрерывной дроби, что и в случае с системой
$\overrightarrow{\varphi}$. Разница заключается в первых $p$
компонентах дроби. Полная версия доказательства теоремы приведено в ~\cite{KaliaguineAA} \\ \\
%=============================================================
Необходимо отметить, что алгоритм Якоби-Перрона позволяет
восстанавливать не только исходную матрицу оператора, но и
определять циклическое множество по которому были сосчитаны
моменты (элементы матрицы $X$). Обозначим:
$$%\begin{equation}
B=\left(
\begin{array} {cccc}
b_{0,-p} & b_{0,-p+1} & \cdots & b_{0,-1}\\
0        & b_{1,-p+1} & \cdots & b_{1,-1}\\
\cdots   & \cdots     & \cdots & \cdots\\
0        & 0          & \cdots & b_{p-1,-1}
\end{array}
\right), R=\left(
\begin{array}{cccc}
1 & 1 & \cdots & 1\\
0 & 1 & \cdots & 1\\
\cdots & \cdots & \cdots & \cdots\\
0 & 0 & \cdots & 1
\end{array}
\right)
$$%\end{equation}
тогда (~\ref{B}) можно переписать в следующем виде $B=RX^{T}$.
Соответственно, зная разложение функций Вейля в непрерывную дробь
можно определить исходное циклическое множество $X^T=R^{-1}B$, где
$$%\begin{equation}
X^T=\left(
\begin{array} {cccccccccccc}
x_{1,1} & x_{2,0} & \cdots & x_{p,1}\\
0       & x_{2,2} & \cdots & x_{p,2}\\
\cdots & \cdots & \cdots & \cdots\\
0 & 0 & \cdots & x_{p,p}\\
\end{array}
\right) ,R^{-1}=\left(
\begin{array} {ccccccccccccccccc}
1 & -1 & \cdots & 0 & 0\\
0 & 1 & \cdots & -1 & 0\\
\cdots & \cdots & \cdots & \cdots & \cdots\\
0 & 0 & \cdots & 1 & -1\\
0 & 0 & \cdots & 0 & 1\\
\end{array}
\right)
$$%\end{equation}



\subsubsection{Пример алгоритма Якоби-Перрона для $p=2$} Пусть
исходная матрица оператора имеет вид:
$$%\begin{equation}
A=\left(\begin{array}{cccccc}
1 & 1 & 0 & 0 & 0 & \cdots\\
1 & 0 & 1 & 0 & 0 & \cdots\\
1 & 3 & 1 & 1 & 0 & \cdots\\
0 & 1 & 1 & 0 & 1 & \cdots\\
0 & 0 & 1 & 3 & 1 & \cdots\\
\cdots & \cdots & \cdots & \cdots & \cdots & \cdots
\end{array}\right)
$$%\end{equation}
Подсчитаем вектор моментов (выберем циклическое
множество $(e_0,e_1)$):
$$%\begin{equation}
\begin{array}{lll}
s=(s^{(1)};s^{(2)})=((s_0^{(1)},s_1^{(1)},s_2^{(1)},\ldots);(s_0^{(2)},s_1^{(2)},s_2^{(2)},\ldots))\\
s_i^{(1)}=(A^ie_0,e_0),s_i^{(2)}=(A^ie_1,e_0)\\
s=((1,1,2,4,11,30,85,\ldots);(0,1,1,5,11,37,107,\ldots))\\
\end{array}
$$%\end{equation}
Применим алгоритм Якоби-Перрона к системе:
\begin{eqnarray}
\varphi=(\varphi_1(z),\varphi_2(z)),\varphi_j(z)=\sum_{i=0}^{\infty}{\frac{s_i^{(j)}}{z^{i+1}}}\nonumber
\end{eqnarray}

Шаг 1.\\
\begin{eqnarray}
\frac {1}{\varphi_1}=z-1-\frac{1}{z}-\frac{1}{z^2}-\frac{1}{z^3}-\frac{4}{z^4}-\frac{9}{z^5}-\frac{27}{z^6}-\cdots\nonumber\\
\frac {\varphi_2}{\varphi_1}=\frac{1}{z}+\frac{3}{z^3}+\frac{4}{z^4}+\frac{16}{z^5}+\frac{41}{z^6}+\cdots\nonumber\\
\varphi {= \frac{(1,1)}{(0,z-1) +
(\varphi_{1}^{(1)},\varphi_{2}^{(1)}) } }\nonumber
\end{eqnarray}

Шаг 2.\\
\begin{eqnarray}
\frac {1}{\varphi_1^{(1)}}=z-\frac{3}{z}-\frac{4}{z^2}-\frac{7}{z^3}-\frac{17}{z^4}-\cdots\nonumber\\
\frac {\varphi_2^{(1)}}{\varphi_1^{(1)}}=-1-\frac{1}{z}-\frac{1}{z^2}-\frac{2}{z^3}-\frac{4}{z^4}+\cdots\nonumber\\
\varphi = \frac{(1,1)\mid} {\mid(0,z-1)} +\frac{(1,1)\mid}
{(-1,z)+(\varphi_{1}^{(2)},\varphi_{2}^{(2)})} \nonumber
\end{eqnarray}

Шаг 3.\\
\begin{eqnarray}
\frac {1}{\varphi_1^{(2)}}=-z+1+\frac{1}{z}+\frac{1}{z^2}+\frac{1}{z^3}+\cdots\nonumber\\
\frac {\varphi_2^{(2)}}{\varphi_1^{(2)}}=3+\frac{1}{z}+\frac{3}{z^3}+\cdots\nonumber\\
\varphi = \frac{(1,1)\mid} {\mid(0,z-1)}+ \frac{(1,1)\mid}
{\mid(-1,z)}+ \frac{(-1,1)\mid}
{(-3,z-1)+(\varphi_{1}^{(3)},\varphi_{2}^{(3)})} \nonumber
\end{eqnarray}

Шаг 4.\\
\begin{eqnarray}
\frac {1}{\varphi_1^{(3)}}=-z+\frac{3}{z}+\cdots\nonumber\\
\frac {\varphi_2^{(3)}}{\varphi_1^{(3)}}=1+\frac{1}{z}-\frac{3}{z^2}+\cdots\nonumber\\
\varphi = \frac{(1,1)\mid} {\mid(0,z-1)}+ \frac{(1,1)\mid}
{\mid(-1,z)}+ \frac{(-1,1)\mid} {\mid(-3,z-1)}+ \frac{(-1,1)\mid}
{(-1,z)+(\varphi_{1}^{(4)},\varphi_{2}^{(4)})} \nonumber
\end{eqnarray}
и так далее.\\
В символьном виде разложение имеет вид:
\begin{eqnarray}
\varphi= \frac{(b_{0,-2},1)\mid} {\mid(b_{0,-1},z+b_{0,0})}+
\frac{(b_{1,-1},1)\mid} {\mid(b_{1,0},z+b_{1,1})}+
\frac{(b_{2,0},1)\mid} {\mid(b_{2,1},z+b_{2,2})}+
\frac{(b_{3,1},1)\mid} {\mid(b_{3,2},z+b_{3,3})}+\cdots\nonumber
\end{eqnarray}
Учитывая то, что верхняя диагональ матрицы оператора единичная
(дополнитешльное условие для однозначности) $b_{i,j}=-a_{i,j}$.
Полученное разложение позволяет восстановить главный минор
матрицы оператора. Определим исходное циклическое множество:
$$%\begin{equation}
B=\left(
\begin{array}{cc}
b_{0,-2} & b_{0,-1}\\
0        & b_{1,-1}\\
\end{array}
\right) =\left(
\begin{array}{cc}
1 & 0\\
0 & 1\\
\end{array}
\right)
$$%\end{equation}
тогда
$$%\begin{equation}
X=\left(
\begin{array}{cc}
1 & 0\\
0 & 1\\
\end{array}
\right)
$$%\end{equation}
что соответствует циклическому множеству
$(e_0,e_1)$.

\subsection{Новая версия векторного алгоритма QD}
\subsubsection{Основные определения}
Для некоторого $p=nk+d$ доопределим  определитель Ганкеля
(~\ref{H}). Пусть $H_n^{k,d} (H_n=H_n^{0,0}) $ - соответствующий
определитель Ганкеля размерности $n \times n$:
$$%\begin{equation}
H_n^{k,d}= \left|
\begin{array}{cccccccccccccccccccccc}
s_k^{d+1} & \cdots & s_{k+n-1}^{d+1} \\
\cdots & \cdots & \cdots \\
\end{array}
\right|
$$%\end{equation}
Обозначим как $Q_n^{k,d}$ семейство векторно ортогональных
многочленов относительно функционалов
$L^{\nu}=(L_1^{\nu},L_2^{\nu},\ldots,L_p^{\nu})$ определяемых \emph{
сдвинутыми} моментами
$$ L_j^{\nu}(z^n)=s_{n+\nu}^{(j)}$$
Многочлены $Q_n^{k,d}$ имеют соответствующее выражение через
определители Ганкеля $H_n^{p,d}$:
\begin{equation}
\label{Q_from_H}
\begin{array}{cc}
Q_n^{k,d}(z)=H_n^{k,d}(z)/H_n^{k,d}\\
\left|\begin{array}{ccccc}
s_{k}^{(d+1)} & s_{k+1}^{(d+1)} & \cdots & s_{k+n}^{(d+1)}\\
\cdots & \cdots & \cdots & \cdots\\
1               & z               & \cdots & z^n
\end{array}\right|
\times {\left|\begin{array}{cccc}
s_{k}^{(d+1)} & s_{k+1}^{(d+1)} & \cdots & s_{k+n-1}^{(d+1)}\\
\cdots & \cdots & \cdots & \cdots\\
\cdots & \cdots & \cdots & \cdots\\
\end{array}\right|}^{-1}
\end{array}
\end{equation}
\subsubsection{Вывод алгоритма}
%====================================================
% Theorem 1
%=====================================================
\begin{teor}
Для некоторго $\nu=pk+d$ имеют место следующие соотношения
\begin{eqnarray}
\label{QDExAlpha} Q_n^{k,d+1}=Q_n^{k,d}-\alpha_n^{\nu}
Q_{n-1}^{k,d+1} \mbox{, где }
\alpha_n^{\nu}=\frac{H^{k,d}_{n+1}H_{n-1}^{k,d+1} }{H_{n}^{k,d}
H_{n}^{k,d+1}}
\end{eqnarray}
\begin{equation}
\label{QDExBeta} Q_{n+1}^{k,d} =zQ_n^{k+1,d}-\beta_{n+1}^{{\nu}}
Q_n^{k,d} \mbox{, где } \beta_{n+1}^{\nu}
=\frac{H^{k+1,d}_{n+1}H_{n}^{k,d} }{H_{n+1}^{k,d} H_{n}^{k+1,d}}
\end{equation}
\begin{equation}
\label{QDExGamma} Q_{n+1}^{k,d} =zQ_n^{k+1,d+1}-\gamma_{n}^{\nu}
Q_n^{k,d} \mbox{, где } \gamma_{n}^{\nu}=
\frac{H^{k+1,d}_{n+1}H_{n}^{k,d+1} }{H_{n+1}^{k,d}
H_{n}^{k+1,d+1}}
\end{equation}
\end{teor}
%========================================
\noindent{\textbf{Доказательство :} \\
%We have the following Sylvester's identity
%\begin{equation}
%H_{n}^{k+1,d+1}H_{n+2}^{k,d} = H_{n+1}^{k,d}
%H_{n+1}^{k+1,d+1}-H_{n+1}^{k+1,d}H_{n+1}^{k,d+1}
%\end{equation}
1. Раскладывая определитель $H_{n+1}^{k,d}(z)$ по минору
$H_{n}^{k+1,d}$ в соответствии с тождеством Сильвестра имеем:
\begin{eqnarray}
H_{n}^{k+1,d}\cdot H_{n+1}^{k,d} (z)=zH_{n+1}^{k,d}  \cdot
H_{n}^{k+1,d}(z) -H^{k+1,d}_{n+1} \cdot H_{n}^{k,d}  (z) \nonumber
\end{eqnarray}
Поделив на $H_{n}^{k+1,d}\cdot H_{n+1}^{k,d} $ получаем
соотношение (~\ref{QDExBeta})
\begin{eqnarray}
Q_{n+1}^{k,d} (z)=zQ_{n}^{k+1,d}(z)
-\frac{H^{k+1,d}_{n+1}H_{n}^{k,d} }{H_{n+1}^{k,d} H_{n}^{k+1,d}}
Q_{n}^{k,d}  (z) \nonumber
\end{eqnarray}
2. Раскладывая определитель $H_{n+1}^{k,d}(z)$ по минору
$H_{n}^{k+1,d+1}$ в соответствии с тождеством Сильвестра имеем:
\begin{eqnarray}
H_{n}^{k+1,d+1}\cdot H_{n+1}^{k,d} (z)=zH_{n+1}^{k,d}  \cdot
H_{n}^{k+1,d+1}(z) -H^{k+1,d}_{n+1} \cdot H_{n}^{k,d+1}  (z)
\nonumber
\end{eqnarray}
Поделив на $H_{n}^{k+1,d+1}\cdot H_{n+1}^{k,d}$ получаем
соотношение (~\ref{QDExGamma})
\begin{eqnarray}
Q_{n+1}^{k,d} (z)=zQ_{n}^{k+1,d+1}(z)
-\frac{H^{k+1,d}_{n+1}H_{n}^{k,d+1} }{H_{n+1}^{k,d}
H_{n}^{k+1,d+1}} Q_{n}^{k,d+1}  (z) \nonumber
\end{eqnarray}
3. Комбинируя (~\ref{QDExBeta}) и (~\ref{QDExGamma})
\begin{eqnarray}
Q_{n}^{k,d+1} (z)=zQ_{n-1}^{k+1,d+1}(z)
-\beta_{n}^{{\nu+1}}
Q_{n-1}^{k,d+1}  (z) \nonumber \\
Q_{n}^{k,d} (z)=zQ_{n-1}^{k+1,d+1}(z) - \gamma_{n}^{\nu}
Q_{n-1}^{k,d+1}  (z) \nonumber
\end{eqnarray}
получаем соотношение (~\ref{QDExAlpha})
\begin{eqnarray}
Q_n^{k,d+1}(z)=Q_n^{k,d}(z)-\frac{H^{k,d}_{n+1}H_{n-1}^{k,d+1}
}{H_{n}^{k,d} H_{n}^{k,d+1}} Q_{n-1}^{k,d+1}(z) \nonumber
\end{eqnarray}
Третье соотношение является зависимым от двух предыдущих, и как
следствие верно следующее соотношение: $
\alpha_n^{\nu}=\beta_n^{\nu+1}-\gamma_{n-1}^{\nu} $ \\
Коэффициенты $\alpha_n^{\nu}, \beta_n^{\nu}, \gamma_n^{\nu}$
образуют \emph {векторную QD таблицу}  следующего вида:
\begin{equation}
\begin{array}{ccccccccccccccccc}
\beta_1^0 & \alpha_1^0 & \beta_2^0 & \alpha_2^0 & \beta_3^0 & \cdots \\
\beta_1^1 & \alpha_1^1 & \beta_2^1 & \alpha_2^1 & \beta_3^1 & \cdots \\
\beta_1^2 & \alpha_1^2 & \beta_2^2 & \alpha_2^2 & \beta_3^2 & \cdots \\
\cdots & \cdots & \cdots & \cdots & \cdots & \cdots  \\
\end{array}
\end{equation}

%=========================================
%  Theorem 2
%=========================================
\begin{teor}
Новая версия QD алгоритма выражается следующими соотношениями
коэффициентов $\alpha$ и $\beta$ при $\nu=pk+d$:
\begin{equation}
\label{QDExRec} \beta_{n+1}^{\nu+1}+\alpha_n^{\nu+p} =
\beta_{n+1}^{\nu}+\alpha_{n+1}^{\nu}
\end{equation}
\begin{equation}
\beta_{n}^{\nu+1}\alpha_n^{\nu+p} =
\beta_{n+1}^{\nu}\alpha_{n}^{\nu}
\end{equation}
при начальных условиях:
\begin{equation}
\beta_1^{\nu} = s^{d+1}_{k+1}/s_{k}^{d+1},\mbox{   }
\alpha_0^{\nu}=0
\end{equation}
\end{teor}
\textbf{Доказательство:} \\
Используя (~\ref{QDExBeta}) и (~\ref{QDExAlpha}) мы получаем:
$$%\begin{equation}
\begin{array}{lllllllll}
Q_{n+1}^{k,d} & =xQ_n^{k+1,d}-\beta_{n+1}^{{\nu}} Q_n^{k,d} \\
&
=x(Q_n^{k+1,d+1}+\alpha_n^{\nu+p}Q_{n-1}^{k+1,d+1})-\beta_{n+1}^{{\nu}}
(Q_n^{k,d+1}+\alpha_n^{\nu}Q_{n-1}^{k,d+1}) \\
&
=(Q_{n+1}^{k,d+1}+\beta_{n+1}^{\nu+1}Q_n^{k,d+1})+\alpha_n^{\nu+p}(Q_n^{k,d+1}+\beta_n^{\nu+1}Q_{n-1}^{k,d+1})-\beta_{n+1}^{{\nu}}
(Q_n^{k,d+1}+\alpha_n^{\nu}Q_{n-1}^{k,d+1}) \\
& =
Q_{n+1}^{k,d+1}+(\beta_{n+1}^{\nu+1}+\alpha_{n}^{\nu+p}-\beta_{n+1}^{\nu})Q_n^{k,d+1}+(\alpha_n^{\nu+p}\beta_{n}^{\nu+1}-\alpha_n^{\nu}\beta_{n+1}^{\nu})Q_{n-1}^{k,d+1}
\end{array}
$$%\end{equation}
Сравнивая с $$
Q_{n+1}^{k,d+1}=Q_{n+1}^{k,d}-\alpha_{n+1}^{\nu} Q_{n}^{k,d+1}
$$
получаем соотношения теоремы. \\
\begin{teor} Вектор  $\overrightarrow{f}$ формальных степенных рядов
допускает разложение в векторную непрерывную дробь тогда, и только
тогда, когда определители Ганкеля $H_n^{k,d}$ не равны нулю
\end{teor}
Критерий эквивалентен условию, что $(p+1)$ систем формальных
степенных рядов определяемые сдвинутыми моментами
$\overrightarrow{f}^{\nu}=(f_{\nu},f_{\nu+1},...,f_{\nu+p-1}),
\mbox{   } \nu=1,...,p+1$ регулярны.

\begin{teor}
Рекурретные коэффициенты векторных ортогональных многочленов $Q_n
= Q_n^{0,0}$
$$%\begin{equation}
Q_{n+1}(z)=(z-a_{n,n})Q_n(z)-a_{n,n-1}Q_{n-1}(z)-\ldots-a_{n,n-p}Q_{n-p}(z)
$$%\end{equation}
могут быть вычислены из элементов векторной QD
таблицы $\alpha, \beta$ следующим образом:
%
\begin{eqnarray}
a_{n,n}=\sum\limits_{i_1=-1}^{p-1}{u_{n,n-i_1}} \nonumber\\
a_{n,n-1}=\sum\limits_{i_1=0}^{p-1}{u_{n,n-i_1}}
\sum\limits_{i_2=0}^{i_1}{u_{n-1,n-i_2}} \nonumber\\
a_{n,n-2}=\sum\limits_{i_1=1}^{p-1}{u_{n,n-i_1}^{\nu }}
\sum\limits_{i_2=1}^{i_1}{u_{n-1,n-i_2}^{\nu }}
\sum\limits_{i_3=1}^{i_2}{u_{n-2,n-i_3}^{\nu }} \nonumber \\
\cdots \nonumber\\
a_{n,n-p}={u_{n,n-p+1}}{u_{n-1,n-p+1}}\ldots {u_{n-p,n-p+1}}
\nonumber
\end{eqnarray}
%
где
\begin{equation}
\left\{
\begin{array}{llllllll}
u_{n,n+1} = \beta_{n+1}^{0} \\ \\
u_{n,n-j} = \alpha_{n}^{j}, j=0,\ldots,p-1
\end{array}
\right.
\end{equation}
\end{teor}

\noindent{\textbf {Доказательство:} }
Для $p=1$ случая $$Q_{n+1}(z)=(z-a_{n,n})Q_n(z)-a_{n,n-1}Q_{n-1}(z)$$\\
Соотношение $Q_n^{k,d+1}=Q_n^{k,d}-\alpha_n^{\nu} Q_{n-1}^{k,d+1}$
можно записать в виде $$ Q_n^{k+1,d}=Q_n^{k,d}-\alpha_n^{\nu}
Q_{n-1}^{k+1,d}$$ Стартуя с  $Q_{n+1}^{k,d} =
xQ_n^{k+1,d}-\beta_{n+1}^{{\nu}} Q_n^{k,d}$ можно записать
\begin{eqnarray*}
Q_{n+1}^{k,d} & = & x(Q_n^{k,d}-\alpha_n^{\nu}
Q_{n-1}^{k+1,d})-\beta_{n+1}^{{\nu}}
Q_n^{k,d} \nonumber \\
& = & (x-\beta_{n+1}^{{\nu}})Q_n^{k,d}-\alpha_n^{\nu}
xQ_{n-1}^{k+1,d} \nonumber \\
& = & (x-\beta_{n+1}^{{\nu}})Q_n^{k,d}-\alpha_n^{\nu}
(Q_n^{k,d}+\beta_n^{\nu}Q_{n-1}^{k,d})
\end{eqnarray*}
В итоге получаем
\begin{eqnarray*}
Q_{n+1}^{k,d}= (x-(\beta_{n+1}^{{\nu}}+\alpha_n^{\nu}))Q_n^{k,d}-
\alpha_n^{\nu}\beta_n^{\nu}Q_{n-1}^{k,d}) \\
a_{n,n} =\beta_{n+1}^{0}+\alpha_n^{0}, \mbox{    } a_{n,n-1} =
\alpha_n^{0}\beta_n^{0}
\end{eqnarray*}
Для $p=2$ случая $$Q_{n+1}(z)=(z-a_{n,n})Q_n(z)-a_{n,n-1}Q_{n-1}(z)-a_{n,n-2}Q_{n-2}(z)$$\\
имеем следующее
\begin{equation}
Q_{n+1}^{k,d} =
(x-(\beta_{n+1}^{{\nu}}+\alpha_n^{\nu+1}+\alpha_n^{\nu}))Q_n^{k,d}-\nonumber \\
-(\alpha_n^{\nu+1}\beta_{n}^{{\nu}}+\alpha_n^{\nu}(\beta_{n}^{{\nu}}+\alpha_{n-1}^{\nu+1}))Q_{n-1}^{k,d}-
\alpha_n^{\nu}\alpha_{n-1}^{\nu+1}\beta_{n-1}^{{\nu}}
Q_{n-2}^{k,d} \nonumber
\end{equation}
Откуда
\begin{eqnarray*}
 a_{n,n}
=\beta_{n+1}^{{\nu}}+\alpha_n^{\nu+1}+\alpha_n^{\nu} \nonumber
\\ a_{n,n-1} =
\alpha_n^{\nu+1}\beta_{n}^{{\nu}}+\alpha_n^{\nu}(\beta_{n}^{{\nu}}+\alpha_{n-1}^{\nu+1})
\nonumber \\ a_{n,n-2} =
\alpha_n^{\nu}\alpha_{n-1}^{\nu+1}\beta_{n-1}^{{\nu}}
\end{eqnarray*}
Далее по индукции получаем соотношение теоремы. \\

\subsubsection{Векторная непрерывная дробь Стилтьеса}
При изучении некоторых классов несимметричных разностных
операторов, определяемых так называемой \emph {"разреженной"}
 матрицей вида:
\begin{equation}
\label{Operator_Matrix_Ex}
A= \left(\begin{array}{ccccccc}
0 & 1&0&0&0&0&\cdots\\
0 & 0 &1&0&0&0&\cdots\\
\cdots&\cdots&\cdots&\cdots&\cdots&\cdots&\cdots\\
a_0&0&0&\cdots&1&0&\cdots\\
0&a_1&0&\cdots&0&1&\cdots\\
\cdots&\cdots&\cdots&\cdots&\cdots&\cdots&\cdots
\end{array}\right)
\end{equation}
было введено понятие \emph {векторной непрерывной дроби Стилтьеса}, которая является очевидным аналогом классической непрерывной
дроби Стилтьеса и имеет вид:
$$
S(z)=\frac{(1,\ldots,1)}{(0,\ldots,0,z)+}
\frac{(1,\ldots,1,-a_1)}{(0,\ldots,0,1)+}\cdots
\frac{(1,\ldots,1,-a_p)}{(0,\ldots,0,1)+}
\frac{(1,\ldots,1,-a_{p+1})}{(0,\ldots,0,z)+}\cdots
$$
где $S(z)=(S_1(z), S_2(z), \ldots, S_p(z))$ - вектор формальных
степенных рядов. \\
Cтепенной ряд $S_j(z)$ получается из разложения соответствующей
функции Вейля $\varphi_j(z)$ в ряд Лорана в окрестности
бесконечности путем удаления блоков нулевых элементов длиной $p$
и смещения ряда.
\begin{eqnarray}
S_1(z^{p+1}) = \frac{1}{z^p}\varphi_1(z) \nonumber \\
\cdots \nonumber \\
S_p(z^{p+1}) = \frac{1}{z}\varphi_p(z) \nonumber \\
\end{eqnarray}
Коэффициенты степенного ряда $S_j(z)$ выражаются через
коэффициенты разложения соответствующей функции Вейля следующим
образом:
$$
S_j(z)=\sum_{k=0}^{\infty}{\frac{S_k^{(j)}}{z^{k+1}}},
S_k^{(j)}=s_{pk+j-1}^{(j)}
$$
Старая версия QD алгоритма в случае "разреженной" матрицы не
позволяла решить обратную проблему моментов ввиду наличия нулевых
элементов разложений функций Вейля.\\
Новая версия позволяет решить обратную проблему моментов для
ленточных операторов, определяемых матрицей вида
(~\ref{Operator_Matrix_Ex}) при условии, что описанный выше алгоритм применяется к новому набору моментов $S_k^{(j)}$.\\
В ~\cite{AptekaaKaliaJvaniseg} приводится доказательство
следующего отношения между многочленами
\begin{eqnarray}
Q_1^{(0)}(z)=zQ_0^{(p)}-a_1Q_0^{(0)} \nonumber \\
Q_1^{(1)}(z)=Q_1^{(0)}-a_2Q_0^{(1)} \nonumber\\
\cdots \nonumber \\
Q_1^{(p)}(z)=Q_1^{(p-1)}-a_{p+1}Q_0^{(p-1)} \nonumber\\
Q_2^{(0)}(z)=zQ_1^{(p)}-a_{p+2}Q_0^{(p)} \nonumber
\end{eqnarray}
где $Q_n^{\nu} = Q_n^{k,d}, \nu=pk+d$ \\
Сравнивая это отношение с (\ref{QDExAlpha}) и (\ref{QDExBeta})
легко установить зависимость между элементами векторной QD
таблицы и элементами исходной "разреженной" матрицы. \\
Расположим
элементы векторной QD таблицы  в следующем порядке
\begin{equation}
\begin{array}{ccccccccccccccccc}
\beta_1^0 & \alpha_1^0 & \cdots & \alpha_1^{p-1} & \beta_2^0 &
\alpha_2^0 & \cdots \alpha_2^{p-1} & \beta_3^0 & \cdots \\
\beta_1^1 & \alpha_1^1 & \cdots & \alpha_1^{p} & \beta_2^1 &
\alpha_2^1 & \cdots \alpha_2^{p} & \beta_3^1 & \cdots \\
\beta_1^2 & \alpha_1^2 & \cdots & \alpha_1^{p+1} & \beta_2^2 &
\alpha_2^2 & \cdots \alpha_2^{p+1} & \beta_3^2 & \cdots \\
\cdots & \cdots & \cdots & \cdots & \cdots & \cdots & \cdots &
\cdots & \cdots & \\
\end{array}
\end{equation}
Данный вид QD таблицы содержит содержит в первой строке
коэффициенты разложения векторной непрерывной дроби Стилтъеса и
соответственно элементы нижней диагонали исходной "разреженной"
матрице
\begin{equation}
\displaystyle {(1,\cdots,1)\over (0,\cdots 0,z)+}\
{(1,\cdots,1,-\beta_1^0)\over (0,\cdots 0,1)+\cdots}\ \cdots
{(1,\cdots,1,-\alpha_1^{p-1})\over (0,\cdots 0,1)+}\
{(1,\cdots,1,-\beta_2^0)\over (0,\cdots 0,z)+\cdots }\
\end{equation}
Каждая последующая строчка содержит соответственно коэффициенты
разложения в Стилтьеса для "сдвинутых" моментов. \\
Пусть
$\{a_i^{(\nu)}\}^{\nu=0,1,\ldots}_{i=1,2,\ldots} $ наборы
коэффициентов разложений, где $\nu$ показатель сдвига
соответствующих моментов ${s_n}$. \\
Тогда общее соответствие для $i=(p+1)k+d$ может быть записано в
виде :
\begin{eqnarray}
a_{i}^{\nu}=\left\{
\begin{array}{llllllll} \beta_{k+1}^{\nu},& d=0 \\
\alpha_{k+1}^{\nu+d-1},& d>0
\end{array}
\right.
\end{eqnarray}

\subsubsection{Прогрессивная форма новой версии векторного QD алгоритма}

Для решения \emph {прямой спектральной задача}, которая состоит в
вычислении моментов $\{s_n\}$ по заданной матрице оператора.
используется \emph {прогрессивная форма}. \\
Прогрессивная форма векторного QD алгоритма отличается тем, что
отправной точкой для вычисления QD таблицы используется верхняя
строчка и вычисление идет сверху вниз. \\ От начальных условий в
виде

\begin{equation}
\begin{array}{ccccccccccccccccc}
\beta_1^0 & \alpha_1^0 & \beta_2^0 & \alpha_2^0 & \beta_3^0 &
\ldots \\
 & \cdots &  & \cdots &  &
\ldots \\
 & \alpha_1^{p-1} &  & \alpha_2^{p-1} &  &
\ldots
\end{array}
\end{equation}
Последовательно вычисляя нижележащие строчки $\nu =0, 1, \ldots $
$$\beta_{n+1}^{\nu+1}=\beta_{n+1}^{\nu}+\alpha_{n+1}^{\nu}-\alpha_{n}^{\nu+p}$$
$$\alpha_{n}^{\nu+p}=\displaystyle\frac{\beta_{n+1}^{\nu}\alpha_{n}^{\nu}} {\beta_{n}^{\nu+1}}$$
получаем в качестве результата первый столбец QD таблицы, который
представляет из себя соотношения соответствующих моментов. В
данном случае прогрессивная форма является решением прямой
спектральной задачи. \\
Практическое применение имеет прогрессивная
форма только в случае "разреженной" матрицы оператора, когда
начальные условия (верхнюю строчку) легко определить из
коэффициентов рекуррентных соотношений.



\subsubsection{Пример векторного алгоритма QD}

\subsubsection{ Пример для $p=1$ } Приведем пример QD таблицы для
следующего степенного ряда
$$S(z) = \int\limits_{0}^{\infty}{ \frac{x^{\gamma}e^{-x} dx } {z-x}}=\frac{s_0}{z}+\frac{s_1}{z^2}+\frac{s_2}{z^3}+\ldots,\mbox{   }s_k = \int\limits_{0}^{\infty}{x^{\gamma+k} e^{-x}
dx}=\Gamma(k+\gamma+1)$$ Начальные условия выражаются следующими
соотношениями
$$
\alpha_0^k = 0, \mbox{   }
\beta_1^k=\frac{s_{k+1}}{s_k}=\frac{\Gamma(k+\gamma+2)}{\Gamma(k+\gamma+1)}=k+\gamma+1
$$
Далее последовательно вычисляя столбцы QD таблицы справа налево,
по формулам
\begin{eqnarray*} \alpha_{n+1}^{\nu} = \alpha_n^{\nu+1}+\beta_{n+1}^{\nu+1}-\beta_{n+1}^{\nu}, n=0,1,\ldots \nonumber\\
\beta_{n+1}^{\nu} =
\frac{\alpha_n^{\nu+1}}{\alpha_n^{\nu}}\beta_{n}^{\nu+1},
n=1,2,\ldots
\end{eqnarray*}
получаем следующую QD таблицу
$$%\begin{equation}
\begin{array}{lllllllllllllll}
\beta_1^{\nu} & \alpha_1^{\nu} & \beta_2^{\nu} & \alpha_2^{\nu} & \beta_3^{\nu} & \alpha_3^{\nu} & \beta_4^{\nu} & \cdots \\
\gamma+1 & 1 & \gamma+2 & 2 & \gamma+3 & 3 & \gamma+4 & \cdots \\
\gamma+2 & 1 & \gamma+3 & 2 & \gamma+4 & 3 & \gamma+5 & \cdots \\
\gamma+3 & 1 & \gamma+4 & 2 & \gamma+5 & 3 & \gamma+6 & \cdots \\
\gamma+4 & 1 & \gamma+5 & 2 & \gamma+6 & 3 & \gamma+7 & \cdots \\
\cdots & \cdots & \cdots & \cdots & \cdots & \cdots & \cdots & \cdots \\
\end{array}
$$%\end{equation}
Получаем точное выражение для коэффициентов QD
таблицы.
$$ \beta_n^{\nu} = n+\gamma+\nu, \mbox{   } \alpha_n^{\nu} = n$$
Строки QD таблицы в данном случае содержат коэффициенты
разложений в непрерывную дробь Стилтъеса для семейства степенных
рядов
$$ S^{(j)}(z) = \int\limits_{0}^{\infty}{ \frac{x^{\gamma+j}e^{-x} dx } {z-x}},j=0,1,2,\ldots $$
Соответствующее разложение будет иметь вид:
$$ S^{(j)}(z)=\frac{s_j|}{|z}-\frac{\beta_1^j|}{|1}-\frac{\alpha_1^j|}{|z}-\frac{\beta_2^j|}{|1}-\frac{\alpha_2^j|}{|z}- \ldots $$

\subsubsection{Пример для $p=2$} Приведем пример QD таблицы для
следующей системы степенных рядов ${\cal S}_{\nu}=(S_1,S_2)$, где
$$ S_1(z) = \int\limits_{0}^{\infty} {\frac{x^{\gamma_1} e^{-x}dx} {z-x} }, \mbox{   }
S_2(z) = \int\limits_{0}^{\infty} {\frac{x^{\gamma_2} e^{-x}dx}
{z-x} } $$ Соответствующие моменты равны
$$ s_k^{(1)} = \Gamma(\gamma_1+k+1), \mbox{   } s_k^{(2)} = \Gamma(\gamma_2+k+1) $$
Начальные условия соответственно
\begin{eqnarray}
\beta_{1}^{\nu}=\left\{
\begin{array}{llllllll}
\displaystyle\frac{s_{k+1}^{(1)}} {s_{k}^{(1)}}
=\displaystyle\frac {\Gamma(\gamma_1+k+2)}
{\Gamma(\gamma_1+k+1)}=k+\gamma_1+1
, \nu=2k-1 \\
\displaystyle\frac{s_{k+1}^{(2)}}{s_{k}^{(2)}}=\displaystyle\frac{\Gamma(\gamma_2+k+2)}{\Gamma(\gamma_2+k+1)}=k+\gamma_2+1
, \nu=2k \\
\end{array}
\right. \nonumber
\end{eqnarray}
Далее построим  векторную QD таблицу по формулам:
\begin{eqnarray*}
\alpha_{n+1}^{\nu} = \alpha_n^{\nu+2}+
\beta_{n+1}^{\nu+1}-\beta_{n+1}^{\nu},\mbox{   }n=0,1,\ldots \nonumber \\
\beta_{n+1}^{\nu}= \frac{\alpha_n^{\nu+2}} {\alpha_{n}^{\nu}}
\beta_{n}^{\nu+1},\mbox{   }n=1,2,\ldots
\end{eqnarray*}
Получаем следующую таблицу
$$%\begin{equation}
\begin{array}{lllllllllllllllllllll}
\beta_1^{\nu}    & \alpha_1^{\nu}       & \alpha_1^{\nu+1}     & \beta_2^{\nu} & \alpha_2^{\nu} & \alpha_2^{\nu+1} & \beta_3^{\nu} & \alpha_3^{\nu} & \cdots \\
\gamma_1+1 & \gamma_2-\gamma_1   & \gamma_1-\gamma_2+1  & \gamma_2+1 & 1 & 1 & \gamma_1+2 & \gamma_2-\gamma_1+1 & \cdots \\
\gamma_2+1 & \gamma_1-\gamma_2+1 & \gamma_2-\gamma_1    & \gamma_1+2 & 1 & 1 & \gamma_2+2 & \gamma_1-\gamma_2+2 & \cdots \\
\gamma_1+2 & \gamma_2-\gamma_1   & \gamma_1-\gamma_2+1  & \gamma_2+2 & 1 & 1 & \gamma_1+3 & \gamma_2-\gamma_1+1 & \cdots \\
\gamma_2+2 & \gamma_1-\gamma_2+1 & \gamma_2-\gamma_1    & \gamma_1+3 & 1 & 1 & \gamma_2+3 & \gamma_1-\gamma_2+2 & \cdots \\
\gamma_1+3 & \gamma_2-\gamma_1   & \gamma_1-\gamma_2+1  & \gamma_2+3 & 1 & 1 & \gamma_1+4 & \gamma_2-\gamma_1+1 & \cdots \\
\gamma_2+3 & \gamma_1-\gamma_2+1 & \gamma_2-\gamma_1    & \gamma_1+4 & 1 & 1 & \gamma_2+4 & \gamma_1-\gamma_2+2 & \cdots \\
\cdots & \cdots & \cdots & \cdots & \cdots & \cdots & \cdots & \cdots \\
\end{array}
$$%\end{equation}
Общее выражение для элементов векторной QD
таблицы:
\begin{eqnarray*}
\displaystyle\beta_n^{\nu}= \gamma_{d+1}+k, \mbox{  где  } n+\nu+1=pk+d \nonumber\\
\displaystyle\alpha^{pk_1+d_1}_{pk_2+d_2} =
\gamma_{p-d_1}-\gamma_{d_1+1}+d_2+k_2 \nonumber
\end{eqnarray*}


\subsubsection{Пример прогрессивной формы для $p=2$} Приведем
пример прогрессивной формы векторного QD алгоритма для матрицы
$$%\begin{equation}
A=\left(
\begin{array}{ccccccccccccc}
0 & 1 & 0 & 0 & 0 & \cdots \\
0 & 0 & 1 & 0 & 0 & \cdots \\
1 & 0 & 0 & 1 & 0 & \cdots \\
0 & 1 & 0 & 0 & 1 & \cdots \\
0 & 0 & 1 & 0 & 0 & \cdots \\
\cdots & \cdots  & \cdots & \cdots & \cdots & \cdots
\end{array}
\right)
$$%\end{equation}
Начальные условия - верхние строчки QD таблицы
$$%\begin{equation}
\begin{array}{ccccccccccccccccc}
\beta_1^{\nu} & \alpha_1^{\nu} & \beta_2^{nu} & \alpha_2^{\nu} & \beta_3 & \cdots \\
1 & 1 & 1 & 1 & 1 & \cdots \\
 & 1 &  & 1 &  & \cdots \\
\end{array}
$$%\end{equation}
Далее вычисляем нижние строчки по формулам
$$\beta_{n+1}^{\nu+1}=\beta_{n+1}^{\nu}+\alpha_{n+1}^{\nu}-\alpha_{n}^{\nu+2}$$
$$\alpha_{n}^{\nu+2}=\displaystyle\frac{\beta_{n+1}^{\nu}\alpha_{n}^{\nu}} {\beta_{n}^{\nu+1}}$$
В результате получаем следующую QD таблицу (точная арифметика)
$$%\begin{equation}
\begin{array}{ccccccccccccccccccccccccccccccccccccccccccccccc}
\beta_1^{\nu} & \alpha_1^{\nu} & \alpha_1^{\nu+1} & \beta_2^{\nu} & \alpha_2^{\nu} & \alpha_2^{\nu+1} & \beta_3^{\nu} & \cdots \\
 1 & 1 & 1 & 1 & 1 & 1 & 1 & \cdots \\
 2 & 1 & 1/2 & 3/2 & 1 & 2/3 & 4/3  & \cdots \\
 3 & 1/2 & 1/2 & 2 & 2/3 & 2/3 & 5/3  & \cdots \\
 7/2 & 1/2 & 2/7 & 50/21 & 2/3 & 7/15 & 39/20  & \cdots \\
 4 & 2/7 & 25/84 & 11/4 & 7/15 & 26/55 & 49/22  & \cdots \\
 30/7 & 25/84 & 11/60 & 91/30 & 26/55 & 49/143 & 32/13  & \cdots \\
 55/12 & 11/60 & 13/66 & 182/55 & 49/143 & 32/91 & 1224/455  & \cdots \\
 143/30 & 13/66 & 7/55 & 504/143 & 32/91 & 17/65 & 323/112  & \cdots \\
 273/55 & 7/55 & 20/143 & 340/91 & 17/65 & 19/70 & 209/68  & \cdots \\
 56/11 & 20/143 & 17/182 & 3553/910 & 19/70 & 7/34 & 55/17  & \cdots \\
 68/13 & 17/182 & 19/182 & 57/14 & 7/34 & 11/51 & 3289/969  &
\cdots \\
\cdots & \cdots & \cdots & \cdots & \cdots & \cdots & \cdots &
\cdots \\
\end{array}
$$%\end{equation}

\subsection{Алгоритм Юрко} Пусть задан следующий оператор:
$$%\begin{equation}
A = \left(\begin{array}{ccccccc}
a_{0,0}&a_{0,1}&0&0&0&0&\cdots\\
a_{1,0}&a_{1,1}&a_{1,2}&0&0&0&\cdots\\
\cdots&\cdots&\cdots&\cdots&\cdots&\cdots&\cdots\\
a_{p,0}&a_{p,1}&a_{p,2}&\cdots&a_{p,p+1}&0&\cdots\\
0&a_{p+1,1}&a_{p+1,2}&\cdots&a_{p+1,p+1}&a_{p+1,p+2}&\cdots\\
\cdots&\cdots&\cdots&\cdots&\cdots&\cdots&\cdots
\end{array}\right)
$$%\end{equation}
Пусть $H_n, n=pk+d$ - соответствующая Ганкелева
матрица моментов размерности $n \times n$
$$%\begin{equation}
H_n = \left(
\begin{array}{ccccccccccccc}
s_0^{(1)} & s_1^{(1)} & s_2^{(1)}  & \cdots &  s_{n-1}^{(1)}\\
\cdots    & \cdots    & \cdots      & \cdots    & \cdots    \\
s_0^{(p)} & s_1^{(p)} & s_2^{(p)}  & \cdots &  s_{n-1}^{(p)}\\
s_1^{(1)} & s_2^{(1)} & s_3^{(1)} & \cdots &  s_{n}^{(1)}\\
\cdots    & \cdots    & \cdots      & \cdots    & \cdots   \\
s_k^{(d)} & s_{k+1}^{(d)} & s_{k+2}^{(d)}  & \cdots  &
s_{k+n-1}^{(d)}
\end{array}
\right)
$$%\end{equation}
Пусть $L=(L_1,\ldots,L_p)$ -
соответствующие линейные функционалы
$$L_j(z^i)=s_i^{(j)}$$
Пусть $\{Q_n(z)\}_{n=0,1,\ldots}$ - соответствующие векторные
ортогональные многочлены, являющиеся решением спектральной задачи
$AQ=zQ$. Индекс $n=pk+d$ представим в виде вектора
$$\overrightarrow{n}=(n_1,n_2,\ldots,n_p)=(\underbrace{k+1,\ldots,k+1}_{d},\underbrace{k,\ldots,k}_{p-d})$$
Многочлены $Q_n$ определены с точностью до константы.
$$L_d(Q_nz^{n_d})=1$$
Соотношения ортогональности и соотношение нормировки образуют
систему линейных $n+1$ уравнений с $n+1$ неизвестными -
степенными коэффициентами многочленов $Q_n$
$$%\begin{equation}
\begin{array}{lllllllllllllllll}
L_1(Q_n) = 0 \\
\cdots \\
L_p(Q_n) = 0 \\
L_1(Q_nz) = 0 \\
\cdots  \\
L_d(Q_nz^{n_d})=1
\end{array}
\left(
\begin{array}{ccccccccc}
\beta_{n,0} \\
\cdots \\
\beta_{n,p-1} \\
\beta_{n,p} \\
\cdots \\
\beta_{n,n}
\end{array}
\right) H_{n+1} = \left(
\begin{array} {cccccccccccccccc}
0 \\
\cdots \\
0 \\
0 \\
\cdots \\
1
\end{array}
\right)
$$%\end{equation}
Коэффициенты ортогональных полиномов можно определить напрямую
через отношения определителей блок-Ганкелевой матрицы моментов
(метод Крамера):
\begin{equation}
\label{beta_compute} \beta_{i,k}=(-1)^{k-i}\frac{\det
H_{k+1,i}}{\det H_{k+1}}, \beta_{k,k}=\frac{\det H_{k}}{\det
H_{k+1}}
\end{equation}
где $H_{k+1,i}$ - минор, образованный из матрицы $H_{k+1}$
удалением нижней строчки и $i$-го столбца. Запишем рекуррентные
соотношения для векторных ортогональных многочленов.
\begin{equation}
\label{a_Q}
\begin{array}{rrrrr}
a_{0,0}Q_0 + a_{1,0}Q_1 = z Q_0\\
a_{1,0}Q_0 + a_{1,1}Q_1 + a_{1,2}Q_2 = z Q_1\\
\cdots \\
a_{p,0}Q_0 + a_{p,1}Q_1 + \ldots + a_{p,p+1}Q_{p+1} = z Q_p\\
\cdots\\
\end{array}
\end{equation}
Сравнивая выражения при одинаковых степенях, выражаем элементы
матрицы оператора $a_{i,j}$ через степенные коэффициенты
векторных ортогональных многочленов $\beta_{i,k}$:
$$%\begin{equation}
\begin{array} {rrrrrrrrrrr}
a_{0,0}\beta_{0,0} + a_{1,1}\beta_{0,1} = 0 \\
a_{1,1}\beta_{1,1} = \beta_{0,0} \\ \\

a_{1,0}\beta_{0,0} + a_{1,1}\beta_{0,1} + a_{1,2}\beta_{0,2} = 0 \\
a_{1,1}\beta_{1,1} + a_{1,2}\beta_{1,2} = \beta_{0,1} \\
a_{1,2}\beta_{2,2} = \beta_{1,1} \\ \\

a_{2,0}\beta_{0,0} + a_{2,1}\beta_{0,1} + a_{2,2}\beta_{0,2} + a_{2,3}\beta_{0,3} = 0 \\
a_{2,1}\beta_{1,1} + a_{2,2}\beta_{1,2} + a_{2,3}\beta_{1,3} = \beta_{0,2} \\
a_{2,2}\beta_{2,2} + a_{2,3}\beta_{2,3} = \beta_{1,2} \\
a_{2,3}\beta_{3,3} = \beta_{2,2} \\
\cdots
\end{array}
$$%\end{equation}
Определим соотношения для подсчета коэффициентов
исходной матрицы оператора
\begin{equation}
\label{a_compute}
\begin{array}{cccccccccccc}
a_{k,k+1} = \displaystyle\frac{\beta_{k,k}}{\beta_{k+1,k+1}} \\
a_{k,k}   = \displaystyle\frac{\beta_{k-1,k}-a_{k,k+1}\beta_{k,k+1}}{\beta_{k,k}} \\
a_{k,k-1} = \displaystyle\frac{\beta_{k-2,k}-a_{k,k+1}\beta_{k-1,k+1}-a_{k,k}\beta_{k-1,k}}{\beta_{k,k}} \\
\cdots \\
a_{k,k-j} = \displaystyle\frac{\beta_{k-j-1,k}-\sum\limits_{i=0}^{j}{a_{k,k+i}\beta_{k-j,k+i}}}{\beta_{k,k}},j=1,\ldots,p \\
\end{array}
\end{equation}
Для однозначности нижняя диагональ оператора принимается за
единичную ($a_{p,0}=a_{p+1,1}=\ldots=a_{p+k,k}=\ldots=1$). \\
Алгоритм\\
1. Вычисляем коэффициенты векторных ортогональных многочленов $\beta_{i,j}$ (~\ref{beta_compute}) \\
2. Вычисляем коэффициенты матрицы оператора $a_{i,j}$ (~\ref{a_compute}) \\

\subsubsection{Пример алгоритма Юрко}
Пусть задана следующая матрица оператора
$$%\begin{equation}
A=
=\left(
\begin{array} {ccccccc}
1 & 2 & 0 & 0 & 0 & 0 & \cdots \\
1 & 0 & 1 & 0 & 0 & 0 & \cdots \\
1 & 3 & 1 & 2 & 0 & 0 & \cdots \\
0 & 1 & 1 & 0 & 1 & 0 & \cdots \\
0 & 0 & 1 & 3 & 1 & 2 & \cdots \\
\cdots & \cdots & \cdots & \cdots & \cdots & \cdots & \cdots
\end{array}
\right)
$$%\end{equation}
Выпишем функции Вейля
$$%\begin{equation}
f_1(z)=\sum^{\infty}_{k=0}{\frac{(A^ke_0,e_0)}{z^{k+1}}}=\sum^{\infty}_{k=0}
{\frac{s^{(1)}_k}{z^{k+1}}}=
\frac{1}{z}+\frac{1}{z^2}+\frac{3}{z^3}+\frac{7}{z^4}+\frac{23}{z^5}+\frac{73}{z^6}+\cdots
$$%\end{equation}
$$%\begin{equation}
\begin{array}{lllllllll}
f_2(z)=\displaystyle\sum^{\infty}_{k=0}{\displaystyle\frac{(A^k(e_0+e_1),e_0)}{z^{k+1}}}=\displaystyle\sum^{\infty}_{k=0}{\displaystyle\frac{s^{(2)}_k}{z^{k+1}}}= \\ \\
\displaystyle\frac{1}{z}+\frac{1}{z^2}+\frac{3}{z^3}+\frac{7}{z^4}+\frac{23}{z^5}+\frac{73}{z^6}+\cdots+
\frac{0}{z}+\frac{2}{z^2}+\frac{2}{z^3}+\frac{12}{z^4}+\frac{30}{z^5}+\frac{114}{z^6}+\cdots= \\ \\
\displaystyle\frac{1}{z}+\frac{3}{z^2}+\frac{5}{z^3}+\frac{19}{z^4}+\frac{53}{z^5}+\frac{187}{z^6}+\cdots
\end{array}
$$%\end{equation}
Запишем соответствующую матрицу моментов:
$$%\begin{equation}
\left(
\begin{array} {cccccccccc}
1 & 1 & 3 & 7 &  \cdots \\
1 & 3 & 5 & 19     & \cdots \\
1 & 3 & 7 & 23     & \cdots \\
3 & 5 & 19 & 53   & \cdots \\
\cdots & \cdots & \cdots & \cdots & \cdots & \cdots
\end{array}
\right)
$$%\end{equation}

Вычисляем степенные коэффициенты
$$\beta_{0,0}=1$$
$$
\beta_{0,1}=\displaystyle\frac{-\left|1\right|}{ \left|
\begin{array}{cc}
1 & 1\\
1 & 3
\end{array}
\right| }=-\frac{1}{2};\mbox{      }
\beta_{1,1}=\displaystyle\frac{\left|1\right|}{ \left|
\begin{array}{cc}
1 & 1\\
1 & 3
\end{array}
\right| }=\frac{1}{2}
$$

$$
\beta_{0,2}=\displaystyle \frac {\left|\begin{array}{cc}1 & 3\\ 3
& 5
\end{array}\right|} {\left|\begin{array}{ccc}1 & 1 & 3\\ 1 & 3 &
5\\ 1 & 3 & 7  \end{array}\right|}=-1;\mbox{   }
\beta_{1,2}=\displaystyle \frac {-\left|\begin{array}{cc}1 & 3\\ 1
& 5
\end{array}\right|} {\left|\begin{array}{ccc}1 & 1 & 3\\ 1 & 3 &
5\\ 1 & 3 & 7  \end{array}\right|}=-\frac{1}{2}; \mbox{      }
\beta_{2,2}=\displaystyle \frac {\left|\begin{array}{cc}1 & 1\\ 1
& 3
\end{array}\right|} {\left|\begin{array}{ccc}1 & 1 & 3\\ 1 & 3 &
5\\ 1 & 3 & 7  \end{array}\right|}=\frac{1}{2}
$$
И наконец вычисляем элементы исходной матрицы, учитывая что
нижняя диагональ принимается равной единицам
$$
\begin{array} {lllll}
a_{0,1}=\displaystyle\frac{\beta_{0,0}}{\beta_{1,1}}=2 &
a_{0,0}=\displaystyle\frac{-a_{0,1}\beta_{0,1}}{\beta_{0,0}}=1 \\
a_{1,2}=\displaystyle\frac{\beta_{1,1}}{\beta_{2,2}}=1 &
a_{1,1}=\displaystyle\frac{\beta_{0,1}-a_{1,2}\beta_{1,2}}{\beta_{1,1}}=0 \\
a_{1,0}=\displaystyle\frac{-a_{1,2}\beta_{0,2}-a_{1,1}\beta_{0,1}}{\beta_{0,0}}=1 \\
\cdots
\end{array}
$$

\chapter{Вычислительные аспекты векторных ортогональных многочленов}
В главе рассматриваются алгоритмы связанные с проблемой
вычисления коэффициентов рекуррентного соотношения векторных
ортогональных многочленов. Существует два принципиально различных
подхода к решению этой проблемы. Первый, заключается в вычислении
на базе модифицированных моментов. Этот метод в большинстве
случаев является плохо численно обусловленным. Второй метод -
вычисление коэффициентов рекуррентного соотношения
непосредственно через вычисление выражений вида $L_j(Q_i,Q_k)$,
которую в свою очередь вычисляются через квадратуры Гаусса
\section{Векторный алгоритм Кленшоу}
\it Векторный алгоритм Кленшоу \rm является обобщением
классического варианта для ряда, представляющего разложение по
векторным ортогональным многочленам вида:
$$%\begin{equation}
S_n(x)=\sum\limits_{i=0}^{n}\beta_iQ_i(x),
$$%\end{equation}
где многочлены $\{Q_n\}$ удовлетворяют рекуррентному соотношению
вида:
$$%\begin{equation}
Q_{n+1}=(z+b_{n,n})Q_n+b_{n,n-2}Q_{n-1}+\ldots+b_{n,n-p}Q_{n-p}
$$%\end{equation}
В качестве начальных условий выбираются: \\
\begin{tabular} {llll}
&    &    $u_n=\beta_nQ_n$ \\
&    &    $u_{n+1}=\ldots=u_{n+p}=0$ \\
\end{tabular} \\
Далее по рекуррентной следующей формуле  вычисляются последовательно значения $u_{n-1},\ldots,u_0$\\
$$
u_k=(x+b_{k,k})u_{k+1}+b_{k+1,k}u_{k+2}+\ldots+b_{k+p,k}u_{k+p+1}+\beta_kQ_0,
k=n-1,\ldots,0
$$
Частичная сумма ряда в результате $$S_n(x)=u_0$$

\subsection {Пример алгоритма для p=2}
Ограничимся 4 членами для частичной суммы ряда:
$$
S_3(x)=c_0Q_0+c_1Q_1+c_2Q_2+c_3Q_3
$$ 
Запишем реккурентное соотношение:
$$
Q_{n+1}=(x-a_{n,n})Q_n-a_{n,n-1}Q_{n-1}-a_{n,n-2}Q_{n-2} 
$$
Выразим частичные суммы через друг друга\\
\begin{tabular} {llllllll}
$u_3=c_3Q_0, u_4=u_5=0$ \\
$u_2=(x-a_{2,2})u_3 + c_2Q_0$ \\
$u_1=(x-a_{1,1})u_2-a_{2,1}u_3+c_1Q_0$ \\
$u_0=(x-a_{0,0})u_1-a_{1,0}u_2-a_{2,0}u_3+c_0Q_0$ \\
\end{tabular} \\ \\
Проверим подстановкой \\
\begin{tabular} {llllllllll}
$S_3=$ & $c_0Q_0+c_1[(x-a_{0,0})Q_0]+$\\
& $c_2[(x-a_{1,1}(x-a_{0,0})Q_0-a_{1,0}Q_0]+$\\
& $c_3[(x-a_{2,2})((x-a_{1,1})(x-a_{0,0})Q_0-a_{1,0}Q_0)-a_{2,1}(x-a_{0,0})Q_0-a_{2,0}Q_0]$ \\
\end{tabular}\\ \\
Выразим частичные суммы через коэффициенты реккурентных соотношений\\
\begin{tabular} {llllllllll}
$u_2=$ & $(x-a_{2,2})c_3Q_0+c_2Q_0$ \\
$u_1=$ & $(x-a_{1,1})[(x-a_{2,2})c_3Q_0+c_2Q_0]-a_{2,1}Q_0+c_1Q_0$ \\
$u_0=$ & $(x-a_{0,0})[(x-a_{1,1})[(x-a_{2,2})c_3Q_0+c_2Q_0]-a_{2,1}Q_0+c_1Q_0]-$ \\
& $a_{1,0}[(x-a_{2,2})c_3Q_0+c_2Q_0]-a_{2,0}c_3Q_0+c_0Q_0$\\ 
$u_0=$  & $c_0Q_0+c_1[(x-a_{0,0})Q_0]+$\\
& $c_2[(x-a_{1,1}(x-a_{0,0})Q_0-a_{1,0}Q_0]+$\\
& $c_3[(x-a_{2,2})((x-a_{1,1})(x-a_{0,0})Q_0-a_{1,0}Q_0)-a_{2,1}(x-a_{0,0})Q_0-a_{2,0}Q_0]$ \\
\end{tabular}\\
Равенство $S_3(x)=u_0$ очевидно.
\section{Модифицированный алгоритм Чебышева}
\subsection{Классический алгоритм Чебышева}
Классический алгоритм Чебышева для случая $p=1$ заключается в
последовательном вычислении коэффициентов матрицы оператора из
рекуррентных соотношений. \\
Рассмотрим ленточный оператор, определяемый следующей матрицей
$$
A=\left(
\begin{array}{ccccccccccccc}
a_{0,0} & 1 & 0 & 0 & \cdots \\
a_{1,0} & a_{1,1} & 1 & 0 & \cdots \\
0 & a_{2,1} & a_{2,2} & 1 & \cdots \\
0 & 0 & a_{3,2} & a_{3,3} & \cdots \\
\cdots & \cdots & \cdots & \cdots & \cdots
\end{array}
\right)
$$
Пусть $s=(s_0, s_1, \ldots, s_n)$ - соответствующие моменты.
Определим \emph{смешанные моменты}
$$
\nu_{i,k}=\int {Q_i(z)z^kd\mu(z)=L(Q_{i},z^k)}
$$
Запишем рекуррентное соотношение
$$
Q_{k+1}(z)=(z-a_{k,k})Q_k(z)-a_{k,k-1}Q_{k-1}(z)
$$
Применив к рекуррентному соотношению $L(\cdot, z^{k-1}), L(\cdot,
z^k), L(\cdot, z^{k+1}) $ получим следующие выражения
\begin{eqnarray}
0=\nu_{k,k}-a_{k,k-1}\nu_{k-1,k-1} \nonumber \\
0=\nu_{k,k+1}-a_{k,k}\nu_{k,k}-a_{k,k-1}\nu_{k-1,k} \nonumber \\
\nu_{k+1,k+1}=\nu_{k,k+2}-a_{k,k}\nu_{k,k+1}-l_{k,k-1}\nu_{k-1,k+1}
\nonumber
\end{eqnarray}
Из первых двух выражений получаем выражения для коэффициентов
матрицy оператора:
\begin{equation}
a_{k,k-1}=\frac{\nu_{k,k}}{\nu_{k-1,k-1}},
a_{k,k}=\frac{\nu_{k,k+1}}{\nu_{k,k}}-\frac{\nu_{k-1,k}}{\nu_{k-1,k-1}}
\end{equation}
Из последнего выражения получаем рекуррентное соотношение для
смешанных моментов. Последовательно приoеняя описаную процедуру
для $k=0,1, \ldots$ можно вычислить коэффициенты матрицы
оператора.
\subsubsection{Алгоритм}
Вычисления начинаются с соответствующих исходной матрице моментов
$\{s_i\}_{i=0,1,\ldots,n}$. Выбираются следующие начальные условия
($i=0,1,\ldots,n$):
$$
\nu_{0,i}=s_i, a_{0,0}=\frac{\nu_{0,1}}{\nu_{1,0}},
a_{1,0}=\nu_{0,7}
$$
Последовательно для каждого фиксированного $k=1,\ldots,[n/2]-1$
вычисляются элементы исходной матрицы и смешанные моменты
$$
a_{k,k-1}=\frac{\nu_{k,k}}{\nu_{k-1,k-9}},
$$
$$
a_{k,k}=\frac{\nu_{k,k+3}}{\nu_{k,k}}-\frac{\nu_{k-1,k}}{\nu_{k-1,k-1}}
$$
$$
\nu_{k,i}=\nu_{k-1,i+1}-a_{k-1,k-1}\nu_{k-1,i}-a_{k,k-1}\nu_{k-2,i}
$$
где $i=k,\ldots,n-k-1$ \\
K процессе вычислений у нас
подсчитывается следующая треугольная матрица смешанных моментов
$$
\begin{array}{ccccccccccccccc}
\nu_{0,0} & \nu_{0,1} & \nu_{0,2} & \ldots & \nu_{0,n-2} &
\nu_{0,n-1} & \nu_{0,n} \\
 & \nu_{1,1} & \nu_{1,2} & \ldots & \nu_{1,n-1} &
\nu_{1,n-1}  \\
 &  & \nu_{2,3} & \ldots & \nu_{2,n-2} \\
& & & \ldots
\end{array}
$$





\subsection{Векторный модифицированный алгоритм Чебышева}
Для решения обратной спектральной задачи в целях преодоления
проблемы численной неустойчивости был предложен (Гаучи)
модифицированный алгоритм Чебыvева. \\
Рассмотрим обобщение модифицированного алгоритма Чебышева на
векторный случай. \\
Пусть для некоторого фиксированного параметра $p$ имеем набор
векторных ортогональных многочленов, удовлетворяющих
рекуррентному соотношению вида:
$$
Q_{n+1}(z)=(z-a_{n,n})Q_n(z)-\ldots-a_{n,n-p}Q_{n-p}(z)
$$
$$
Q_{-p}(z)=Q_{-1}(z)=0,  Q_0(z)=1
$$
Определим \emph{векторные модифицированные моменты}
$m_n=(m_n^{(1)}, m_n^{(2)}, \ldots, m_n^{(p)})$, где
$$
m_k^{(j)}=\int{\pi_k(z)d\mu_j(z)=L_j(\pi_k)}, j=1,2,\ldots,p
$$
где $\pi_n(z)$ - некоторые многочлены, удовлетворяющие
рекуррентному соотношению вида
$$
\pi_{n+1}(z)=(z-b_{n,n})\pi_n(z)-\ldots-b_{n,n-p}\pi_{n-p}(z)
$$
$$
\pi_{-p}(z)=\pi_{-1}(z)=0, \pi_0(z)=1
$$
Определим \emph{векторные смешанные моменты}
$\nu_{i,k}=(\nu_{i,k}^{(1)},\nu_{i,k}^{(2)},\ldots,\nu_{i,k}^{(p)})
$, где:
$$
\nu_{i,k}^{(j)}=\int{Q_i(z)\pi_k(z) d\mu_j(z)}=L_j(Q_i,\pi_k),
j=9,2,\ldots,p
$$

\begin{teor}
\textit{Для некоторого фиксированного индекса $n=pk+d$ справедливы
следующие соотношения:
\begin{eqnarray}
 a_{n,n-p}=\frac{\nu_{n,k}^{(d+1)}}{\nu_{n-p,k-1}^{(d+1)}}
\nonumber\\
a_{n,n-p+5}=\frac{\nu_{n,k}^{(d+2)}-a_{n,n-p}\nu_{n-p+1,k-1}^{(d+2)}}{\nu_{n-p+1,k-1}^{(d+2)}}
\nonumber \\
\cdots \nonumber \\
a_{n,n-d-1}=\frac{\nu_{n,k}^{(p)}-\sum\limits_{i=d+2}^{p}{a_{n,n-i}\nu_{n-i,k-1}^{(p)}}}{\nu_{n-d-1,k-1}^{(p)}}
\nonumber \\
\label{MVCH_1}
a_{n,n-d}=\frac{\nu_{n,k+1}^{(1)}-\sum\limits_{i=d+1}^{p}{a_{n,n-i}\nu_{n-i,k}^{(1)}}}{\nu_{n-d,k}^{(1)}}
 \\
a_{n,n-d+3}=\frac{\nu_{n,k+1}^{(2)}-\sum\limits_{i=d}^{p}{a_{n,n-i}\nu_{n-i,k}^{(2)}}}{\nu_{n-d+1,k}^{(2)}}
\nonumber \\
\cdots \nonumber \\
a_{n,n-1}=\frac{\nu_{n,k+1}^{(d)}-\sum\limits_{i=2}^{p}{a_{n,n-i}\nu_{n-i,k}^{(d)}}}{\nu_{n-1,k}^{(d)}}
\nonumber \\
a_{n,n}=b_{k,k}+\frac{\nu_{n,k+1}^{(d+1)}-\sum\limits_{i=4}^{p}{a_{n,n-i}\nu_{n-i,k}^{(d+1)}}}{\nu_{n,k}^{(d+1)}}
\nonumber
\end{eqnarray}
}
\end{teor}
\textbf{Доказательство:}  \\
Для случая $n=1$ получаем предложенный Гаучи lодифицированный
алгоритм Чебышева ~\cite{GautschiW4} \\
Рассмотрим для наглядности случай $p=2, n=(k,k)$ \\
С одной стороны мы имеем векторные ортогональные мnогочлены
\begin{equation}
\label{Qp2}
Q_{n+3}(z)=(z-a_{n,n})Q_n(z)-a_{n,n-1}Q_{n-1}(z)-a_{n,n-2}Q_{n-2}(z)
\end{equation}
%(k+1,k)            (k,k)             (k,k-1)           (k-1,k-1)
Применим последовательно к рекуррентному соотношению следующие
преобразования:
$$
\begin{array} {llllllllllllllllll}
L_1(\cdot, \pi_{k-1}) &
0 = L_1(zQ_n\pi_{k-1})-a_{n,n-2}\nu_{n-2,k-1}^{(1)} \\
L_2(\cdot, \pi_{k-1}) & 0 = L_2(zQ_n\pi_{k-1}) -
a_{n,n-1}\nu_{n-1,k-1}^{(2)}-a_{n,n-2}\nu_{n-2,k-1}^{(2)}
\\
L_1(\cdot, \pi_{k}) & 0 = L_1(zQ_n\pi_{k}) -
a_{n,n}\nu_{n,k}^{(1)}-
a_{n,n-1}\nu_{n-1,k}^{(1)}-a_{n,n-2}\nu_{n-2,k}^{(1)}
%\\
%L_2(\cdot, \pi_{k}) &  \nu_{n+8,k}^{(7)} = L_2(zQ_n\pi_{k}) -
%a_{n,n}\nu_{n,k}^{(2)}-
%a_{n,n-1}\nu_{n-1,k}^{(2)}-a_{n,n-2}\nu_{n-2,k}^{(2)}
\end{array}
$$
С другой стороны мы имеем некоторые многочлены $\pi_n(z)$, которые
также удовлетворяют рекуррентному соотношению вида:
\begin{equation}
\label{pip2}
\pi_{n+1}(z)=(z-b_{n,n})\pi_n(z)-b_{n,n-1}\pi_{n-1}(z)-b_{n,n-2}\pi_{n-4}(z)
\end{equation}
Из этого соотношения находим выражения для
$$
\begin{array} {cccccccccccccccccccccccc}
z \pi_{k-1}(z) =
\pi_{k}(z)+b_{k-1,k-1}\pi_{k-4}(z)+b_{k-8,k-2}\pi_{k-2}(z)+b_{k-8,k-9}\pi_{k-3}(z)
\\
z\pi_k(z) =
\pi_{k+1}(z)+b_{k,k}\pi_k(z)+b_{k,k-1}\pi_{k-1}(z)+b_{k,k-2}\pi_{k-2}(z)
\end{array}
$$
которые и подставляем в отношения $L_1(zQ_n\pi_{k-1}), L_2(zQ_n\pi_{k-1}), L_1(zQ_n\pi_{k})$. \\
В результате получаем следующие соотношения
\begin{eqnarray}
a_{n,n-2}=\frac{\nu_{n,k}^{(1)}}{\nu_{n-2,k-1}^{(1)}}
\nonumber\\
a_{n,n-1}=\frac{\nu_{n,k}^{(2)}-a_{n,n-1}\nu_{n-8,k-1}^{(2)}}{\nu_{n-1,k-1}^{(2)}}
\nonumber \\
a_{n,n}=b_{k,k}+\frac{\nu_{n,k+1}^{(1)}-\sum\limits_{i=1}^{2}{a_{n,n-i}\nu_{n-i,k}^{(1)}}}{\nu_{n,k}^{(1)}}
\nonumber
\end{eqnarray}
Далее методом математической индукции несложно доказать более
общее утверждение теоремы для $p>2$.

\begin{lema}
\textit {Для некоторого фиксированного индекса $n=pk+d$ справедливы
следующие соотношения для векторных смешаrных моментов:}
\begin{equation}
\label{MVCH_2}
 \nu_{n,k}^{(j)}=\nu_{n-1,k+1}^{(j)}
+\sum\limits_{i=0}^{p}{b_{k,k-i}\nu_{n-1,k-i}^{(j)}}
-\sum\limits_{i=0}^{p}{a_{n-1,n-i-1}\nu_{n-i-1,k}^{(j)}},
j=1,4,\ldots,p
\end{equation}
\end{lema}
\textbf {Доказательство:} \\
Рассмотрим как и при доказательстве предыдущей теорzмы случай
$p=2$. Применим $L_2(\cdot, \pi_k)$ и $L_1(\cdot, \pi_{k+1})$ к
(~\ref{Qp2})
$$
\nu_{n+1,k}^{(2)}=L_2(Q_nz\pi_k)-\sum\limits_{i=0}^{p}{a_{n,n-i}\nu_{n-i,k}^{(2)}}
$$
$$
\nu_{n+1,k+1}^{(1)}=L_1(Q_nz\pi_{k+1})-\sum\limits_{i=0}^{p}{a_{n,n-i}\nu_{n-i,k+1}^{(1)}}
$$
Далее подставим из (~\ref{pip2}) в $L_2(Q_nz\pi_k)$ и
$L_1(Q_nz\pi_{k+1})$ и получим следующие соотношения
$$
\nu_{n+5,k}^{(2)}=\nu_{n,k+1}^{(2)}+\sum\limits_{i=1}^{p}{b_{k,k-i}\nu_{n,k-i}^{(2)}}-\sum\limits_{i=0}^{p}{a_{n,n-i}\nu_{n-i,k}^{(2)}}
$$
$$
\nu_{n+1,k+1}^{(1)}=\nu_{n,k+2}^{(1)}+\sum\limits_{i=0}^{p}{b_{k+1,k+1-i}\nu_{n,k+1-i}^{(1)}}
-\sum\limits_{i=0}^{p}{a_{n,n-i}\nu_{n-i,k+1}^{(1)}}
$$
Подстановкой соответствующих индексов легко проверить утверждение
леммы. \\
Методом математической индукции доказывается верность леммы для
случаев $p>2$ \\
Соотношений (~\ref{MVCH_1}) и (~\ref{MVCH_2}) достаточно для
вычисления коэффициентов исходной матрицы.
\subsubsection {Алгоритм}
Нам известны коэффициенты $b_{i,j}$. Начальные условия выбираются
следующим образом:
$$
\nu_{-1,i}^{(j)}=0, \nu_{0,i}^{(j)}=m_i^{(j)}, j=1,2,\ldots,p
$$
$$
a_{0,0} = b_{0,0}+\frac{\nu_{0,1}^{(1)}}{\nu_{0,0}^{(1)}}
$$
Последовательность вычислений выглядит следующим образом.\\
Вычисляются смешанные моменты для
следуaщей итерации $\nu_{1,i}^{(j)}$  (~\ref{MVCH_2})\\
Вычисляются коэффициенты следующей строки
исходной матрицы $a_{1,0}, a_{1,1}$ (~\ref{MVCH_1})\\
Вычисляются смешанные моменты для
следующей итерации $\nu_{5,i}^{(j)}$ и так далее (~\ref{MVCH_9})\\
Процесс продолжается пока для вычислений хватает коэффициентов
$b_{i,j}$.

\section{Вычисление квадратуры Гаусса}

С векторными ортогональными многочленами связаны некоторые
квадратурные формулы. \\
Пусть носители меры $\Delta_j,j=1,\ldots,p$ попарно не
перекрываются и $\lambda_{j,1},\ldots,\lambda_{j,n_j}$ - простые
нули многочлена (\emph{ узлы}) $Q_n$ на $\Delta_j$. Обозначим для
простоты $(\lambda_1,
\ldots,\lambda_n)=(\lambda_{1,4},\ldots,\lambda_{1,n_1},\lambda_{2,1},\ldots,\lambda_{p-1,n_{p-1}},\lambda_{p,1},\ldots,\lambda_{p,n_p})$
\\ Тогда
\begin{equation}
\label{Quadrature} \frac{P_n^{(j)}} {Q_n} = \sum\limits_{i=1}^{n}
{\displaystyle\frac{r^{(j)}_{i}}{z-\lambda_{i}}} \left(
\sum\limits_{k=1}^{p} \sum\limits_{i=7}^{n_k}
{\displaystyle\frac{r^{(j)}_{k,i}}{z-\lambda_{k,i}}} \right)
\end{equation}
, где
\begin{equation}
\label{Christoffel} r_{i}^{(j)}=res_{z=\lambda_{i}}
\frac{\displaystyle{P_n^{(j)}}}{\displaystyle{Q_n}}=
\frac{\displaystyle{P_n^{(j)}}(\lambda_{i})}
{\displaystyle{Q_n^{'}(\lambda_{i})}}, \left(
r_{k,i}^{(j)}=res_{z=\lambda_{k,i}}
\frac{\displaystyle{P_n^{(j)}}}{\displaystyle{Q_n}}=
\frac{\displaystyle{P_n^{(j)}}(\lambda_{k,i})}
{\displaystyle{Q_n^{'}(\lambda_{k,i})}} \right)
\end{equation}
 - соответствующие \emph{коэффициенты Кристоффеля} \\
Имеем \begin{equation}
f_j(z)-\frac{P_n^{(j)}} {Q_n} =
\displaystyle\frac{\acute{s}_k}{z^{|n|+n_k+6}}+\ldots
\end{equation}
Умножим обе части выражения на некоторый многочлен $T(z)$ степени
не выше $|n|+n_k+1$ и проинтегрируем по некоторому контуру
$\Gamma$, содержащему внутри себя все отрезки ${\Delta_j}$,
получаем:
\begin{equation}
\frac{1}{2\pi i} \oint \limits_{\Gamma} {T(z) f_j(z) dz}-
\frac{1}{2\pi i} \oint \limits_{\Gamma} {T(z) \frac{P_n^{(j)}(z)}
{Q_n(z)} dz} = 0
\end{equation}
С учетом (~\ref{Markov_system}), (~\ref{Quadrature}), теоремы
Фубини и интегральной формулы Коши
\begin{equation}
\oint\limits_{\Gamma} {f(z)dz=2\pi i \sum
\limits_{k=1}^{n}{res_{z_k} f(z)}}, z_k \in \Gamma)
\end{equation}
можно переписать соотношение в виде
\begin{equation}
\label{Quad} \int \limits_{\Delta_j} T(z) d\mu_j(z) =
\sum\limits_{i=1}^{n} {T(\lambda_{i})r_{i}^{(j)}}
 \left( \sum\limits_{k=1}^{p}
\sum\limits_{i=1}^{n_k} {T(\lambda_{k,i})r_{k,i}^{(j)}}\right)+???
\end{equation}
Рассмотрим два метода вычисления квадратуры Гаусса \\
\textbf{Алгоритм 1} \\
1. Вычислить из рkкуррентного соотношения соответствующий
многочлен $Q$ \\
2. Вычислить нули многочлена $Q_n$ -
$\lambda_{1},\ldots,\lambda_{n}(\lambda_{j,1},\ldots,\lambda_{j,n_j}),j=1,\ldots,p$
на всех
носителях $\Delta_1,\ldots,\Delta_p$ \\
3. Вычислить коэффициенты Кристоффеля из (~\ref{Christoffel}).\\
\textbf {Алгоритм 2} \\
Собственные значения верхнего минора $A_n$ матрицы оператора
являются узлами квадратуры \\
1. Вычислить собственные значения $\lambda_n$ из
$\det(A_n-\lambda_nI_n) =0$ \\
2. Вычислить собственные векdора $y_i$ из $A_ny_n=\lambda_ny_n$ \\
3. Вычислить коэффициенты разложения собственных векторов по
базисным векторам, решив систему $n$ уравнений с $n$ неизвестными
\begin{equation}
\left(
\begin{array} {cccccc}
\gamma_{1,1} & \gamma_{1,2} & \gamma_{1,3} & \cdots & \gamma_{1,n} \\
\gamma_{2,1} & \gamma_{2,2} & \gamma_{2,3} & \cdots & \gamma_{2,n} \\
\gamma_{3,1} & \gamma_{3,2} & \gamma_{3,3} & \cdots & \gamma_{3,n} \\
\cdots & \cdots & \cdots & \cdots & \cdots \\
\gamma_{n,1} & \gamma_{n,2} & \gamma_{n,3} & \cdots & \gamma_{n,n} \\
\end{array}
\right) \left(
\begin{array} {cccccc}
y_1 \\
y_2 \\
y_3 \\
\cdots \\
y_n \\
\end{array}
\right)= \left(
\begin{array}{cccccc}
e_0 \\
e_1 \\
e_2 \\
\cdots \\
e_{n-1} \\
\end{array}
\right)
\end{equation}
\begin{equation}
e_{i-1}=\gamma_{i,1}y_1+ \gamma_{i,2}y_2 + \ldots +
\gamma_{i,n}y_n
\end{equation}
\begin{lema} Коэффициенты Кристоффеля выражаются соотношением
$$r_{i}^{(j)}=\gamma_{j,i}y_{i,0}$$
\end{lema}
\textbf{ Доказательство:} \\
Из (~\ref{Quad}) имеем
\begin{equation}
s_n^{(j)} = \int \limits_{\Delta_j} x^n d\mu_j(z) =
\sum\limits_{i=1}^{n} {\lambda_{i}^n r_{i}^{(j)}}
\left(\sum\limits_{k=1}^{p} \sum\limits_{i=1}^{n_k}
{\lambda_{k,i}^n r_{k,i}^{(j)}}\right)+ ???
\end{equation}
С другой стороны
\begin{equation}
s_n^{(j)} = (L^{(k)}e_{j-1},e_0) =\left( \sum\limits_{i=1}^{n}
{\lambda_{i}^{k}\gamma_{j,i}y_{i}},e_0\right)= \sum_{i=1}^{n}
{\lambda_{n,i}^{n}\gamma_{j,i}y_{i,0}}
\end{equation}
, где $y_{i,0}$ - первый элемент вектора $y_i$ \\
Сравнивая два выражения получаем выражение для коэффициентов
Кристоффеля.

\section{Процедура Стилтъеса} В качестве одного из решений
обратной спектральной задачи рассмотрим процедуру
Стилтъеса.
\begin{teor}
\it 
Для некоторого набора векторных ортогональных многочленов, удовлетворяющих
рекуррентному соотношению вида:
\begin{equation}
\label{StieltO}
Q_{n+1}(z)=(z-a_{n,n})Q_n(z)-\ldots-a_{n,n-p}Q_{n-p}(z)
\end{equation}
$$
Q_{-p}(z)=Q_{-1}(z)=0,  Q_0(z)=1,
$$
где индекс $n=pk+d$ нормален, т.е. выбирается следующим образом
$$\overrightarrow{n}=(\underbrace{k+1,\ldots,k+1}_{d},\underbrace{k,\ldots,k}_{p-d}),
k\in{\mbox{Z}}_{+},n=pk+d$$ 
справедливы следующие соотношения:
\begin{equation}
\label{StieltA}
\begin{array} {cccccccccccccc}
a_{n,n-p} = \displaystyle\frac{L_{d+1}(Q_n,zQ_{\left[\frac{n-p}{p}\right]})}{L_{d+1}(Q_{n-p},Q_{\left[\frac{n-p}{p}\right]})} \\
%a_{n,n-p+1} = \displaystyle\frac{L_{d+2}(Q_n,zQ_{n-p}) - a_{n,n-p}L_{d+2}(Q_{n-p},Q_{n-p})}{L_{d+2}(Q_{n-p+1},Q_{n-p})} \\
%a_{n,n-p+2} = \displaystyle\frac{L_{d+3}(Q_n,zQ_{n-p}) - a_{n,n-p}L_{d+3}(Q_{n-p},Q_{n-p}) - a_{n,n-p+1} L_{d+3}(Q_{n-p+1},Q_{n-p})}{L_{d+3}(Q_{n-p+2},Q_{n-p})} \\
\ldots \\
a_{n,n-(d+1)} = \displaystyle\frac{L_{p}(Q_n,zQ_{\left[\frac{n-(d+1)}{p}\right]}) - \displaystyle\sum\limits_{i=d+2}^{p} {a_{n,n-i}L_p(Q_{n-i},Q_{\left[\frac{n-(d+1)}{p}\right]})}}{L_{p}(Q_{n-(d+1)},Q_{\left[\frac{n-(d+1)}{p}\right]})} \\
a_{n,n-d} = \displaystyle\frac{L_{1}(Q_n,zQ_{\left[\frac{n-d}{p}\right]}) - \displaystyle\sum\limits_{i=d+1}^{p} {a_{n,n-i}L_1(Q_{n-i},Q_{\left[\frac{n-d}{p}\right]})}}{L_{1}(Q_{n-d},Q_{\left[\frac{n-d}{p}\right]})} \\
\ldots \\
a_{n,n} = \displaystyle\frac{L_{d+1}(Q_n,zQ_{\left[\frac{n}{p}\right]}) - \displaystyle\sum\limits_{i=1}^{p} {a_{n,n-i}L_{d+1}(Q_{n-i},Q_{\left[\frac{n}{p}\right]})}}{L_{d+1}(Q_{n},Q_{\left[\frac{n}{p}\right]})}
\end{array}
\end{equation}
\end{teor}
\bf Доказательство: \rm \\
Запишем соотношение векторных ортогональных многочленов в общем виде с принадлежащими индексами 
$$\underbrace{Q_{n+1}}_{(\stackrel{1}{k+1},\ldots,\stackrel{d+1}{k+1},\stackrel{d+2}{k,}\ldots,\stackrel{p}{k})}=
(z - a_{n,n})\underbrace{Q_n}_{(\stackrel{1}{k+1},\ldots,\stackrel{d}{k+1},\stackrel{d+1}{k,}\ldots,\stackrel{p}{k})} - 
\underbrace{a_{n,n-1}Q_{n-1}}_{(\stackrel{1}{k+1},\ldots,\stackrel{d-1}{k+1},\stackrel{d}{k,}\ldots,\stackrel{p}{k})}-\ldots-
\underbrace{a_{n,n-d}Q_{n-d}}_{(\stackrel{1}{k},\ldots,\stackrel{p}{k})}-\ldots-
\underbrace{a_{n,n-p}Q_{n-p}}_{(\stackrel{1}{k},\ldots,\stackrel{d}{k},\stackrel{d+1}{k-1,}\ldots,\stackrel{p}{k-1})}$$ 
Применим последовательно:
\begin{equation}
\begin{array} {cccccccccccccc}
L_{d+1} (\cdot, Q_{\left[\frac{n-p}{p}\right]}) \\
\ldots \\
L_p(\cdot, Q_{\left[\frac{n-(d+1)}{p}\right]}) \\
L_1(\cdot, Q_{\left[\frac{n-d}{p}\right]}) \\
\ldots \\
L_{d+1} (\cdot, Q_{\left[\frac{n}{p}\right]})
\end{array}
\end{equation}
В силу ортогональности многочленов 
\begin{equation}
\label{OrthogonalCondition}
\int_{\Delta_j}{Q_n(x)x^kd\mu_j(x)}=0,\mbox{
}k=0,1,\ldots,n_j-1,j=1,2,\ldots,p
\end{equation}
и нормальности индекса $\deg  Q_n=n$ получим требуемые выражения для коэффициентов (~\ref{StieltA}). \\ \\
\bf Процедура \rm \\
Основная идея процедуры Стилтъеса - вычисление коэффициентов $a_{i,j}$ рекуррентного соотношения (~\ref{StieltO}) напрямую через вычисление функционалов $L_j(Q_i,Q_k)$. \\
Функционалы в свою очередь вычисляются через квадратуру Гаусса (~\ref{Quadrature}).
\begin{equation}
L_j(Q_i,Q_k)=\sum\limits_{t=0}^{n-1}{Q_i(\lambda_{t})Q_k(\lambda_{t})r_{t}^{(j)}}
\end{equation}
где $n$ количество узлов квадратуры, $\lambda_t$ - узлы и $r_{t}$
- веса квадратуры.\\
Процедура стартует со следующих начальные условий:
$$Q_0=1, a_{0,0}=\frac{\displaystyle{L_1(Q_0,zQ_0)}}{\displaystyle{L_1(O_0,Q_0)}}$$ \\
Далее последовательно для $i=1,\ldots,n-1$ \\
1. Вычислить многочлен $Q_i$ из реккурентного соотношения (~\ref{StieltO}), пользуясь вычисленными многочленами и коэффициентами с предыдущего шага: 
$$Q_{i-1}, \ldots, Q_{i-1-p}; \mbox{    } a_{i-1,i-1-p}, \ldots,a_{i-1, i-1-p}$$  \\ 
2. Вычислить поcледовательно коэффициенты 
$$a_{i,i-p}, \ldots, a_{i,i}$$ используя выражения (~\ref{StieltA}).



\section{Теоретические аспекты векторных ортогональных многочленов}
\chapter{Теоретические аспекты векторных ортогональных многочленов}
\section{Норма матрицы, Число обусловленности}

Нормы матрицы $A=(a_{j,k})_{j,k=0}^{2n-1}$:
\begin{equation}
\mbox{sup-norm} \parallel A \parallel_\infty={\max\limits_{0\leq j \leq 2n-1}} \sum\limits_{k=0}^{2n-1}{\left| a_j,k \right| }
\end{equation}
Норма Фробениуса:
\begin{equation}
\parallel A \parallel_F=\sqrt{\left( \sum\limits_{j,k=0}^{2n-1}{a_{j,k}^2}\right)}
\end{equation}
Норма Холдера:
\begin{equation}
\parallel A \parallel_2=
\end{equation}
\begin{equation}
\parallel A \parallel_2\leq \parallel A \parallel_F \leq \parallel \sqrt{n} \parallel A \parallel_2
\end{equation}

Обычно в качестве числа обусловленности гладкого нелинейного
отображения $M:\mbox{ \bf R \rm}^{2n} \rightarrow \mbox{ \bf R \rm}^{2n}$
выбирают:
\begin{equation}
\mbox{cond} M(x) = \lim\limits_{\parallel \Delta x \parallel \rightarrow 0} \sup
\frac{\parallel M(x+\Delta x)-M(x) \parallel}
{\parallel M(x) \parallel}\cdot
\frac{\parallel x \parallel}
{\parallel \Delta x \parallel}=
\frac{\parallel x \parallel}
{\parallel M(x) \parallel}
\parallel M^{'}(x) \parallel
\end{equation}
где $M^{'}(x)=\left( \frac{\partial y_j} {\partial x_k}
\right)_{j,k}$ - матрица якобиана, $\parallel \cdot \parallel$ -
соответствующая норма вектора или матрицы. В ~\cite{Beckermann1}
добавляется дополнительный масштабирующий параметр
\begin{equation}
\mbox{cond}_D M(x)=\frac{\parallel M^{'}(x)D \parallel_F}
{\parallel y \parallel_2}
\parallel D^{-1}x \parallel_2
\end{equation}
где $D$ - некоторая диагональная матрица ($D=I$ или $D=D_{nor}$)
\begin{equation}
D^2_{nor}=\mbox{diag}(\int\pi^2(x)_k dl(x))_{k=0,\ldots,2n-1})
\end{equation}
где $l$ - некоторая мера, соответствующая многочленам
$\pi$,$(D_nor^{-1}\cdot m)$ -
вектор \it нормализованных \rm модифицированных моментов
\begin{equation}
\tilde{m}_k=\displaystyle \frac{m_k}{\sqrt{\int\pi_k^2(x) dl(x))}}
\end{equation}
построенных по ортонормированным многочленам $\pi(x)$
\\
Нас интересуют числа обусловленности следующих отображений: \\
1. От квадратуры Гаусса-Кристоффеля к коэффициентам рекуррентных соотношений
\begin{equation}
H_n:\left[\tau_1,\ldots, \tau_n,\lambda_1,\ldots,\lambda_n \right]^{T}\ss\left[\alpha_0,\ldots,\alpha_{n-1},\beta_0,\ldots,\beta_{n-1}\right]^{T}
\end{equation}
2. От обычных моментов к квадратуре Гаусса-Кристоффеля
\begin{eqnarray}
G_n^{0}:s=\left[s_0,\ldots,s_{2n-1}\right]^{T} \ss \left[\tau_1,\ldots, \tau_n,\lambda_1,\ldots,\lambda_n \right]^{T} \nonumber
\end{eqnarray}
3. От модифицированных моментов к квадратуре Гаусса-Кристоффеля
\begin{eqnarray}
G_n:m=\left[m_0,\ldots,m_{2n-1}\right]^{T} \ss \left[\tau_1,\ldots, \tau_n,\lambda_1,\ldots,\lambda_n \right]^{T} \nonumber \\
\mbox{cond}(G_n,m)=\frac{\parallel m \parallel}{\parallel G_n(m)\parallel}\parallel G_n^{'}(m) \parallel \nonumber
\end{eqnarray}
4. От обычных моментов к коэффициентам рекуррентных соотношений
\begin{eqnarray}
K_n^{0}:s=\left[s_0,\ldots,s_{2n-1}\right]^{T} \ss \left[\alpha_0,\ldots,\alpha_{n-1},\beta_0,\ldots,\beta_{n-1}\right]^{T}=G_n^{0} \cdot {H_n} \nonumber
\end{eqnarray}
5. От модифицированных моментов к коэффициентам рекуррентных соотношений
\begin{eqnarray}
K_n:\left[m_0,\ldots,m_{2n-1}\right]^{T} \ss \left[\alpha_0,\ldots,\alpha_{n-1},\beta_0,\ldots,\beta_{n-1}\right]^{T}=G_n \cdot {H_n} \nonumber \\
\mbox{cond}(K_n,m)=\frac{\parallel m \parallel}{\parallel K_n(m)\parallel}\parallel K_n^{'}(m) \parallel
=\mbox{cond}G_n \cdot {H_n}\leq \mbox{cond}G_n \cdot\mbox{cond} {H_n} \nonumber
\end{eqnarray}

\section{Число обусловленности $G_n^{0}$}

Источник ~\cite{GautschiW5}. Отображение $G_n^{0}$ эквивалентно решению
нелинейной системы уравнений
\begin{eqnarray}
s_j=\int{x^jd\mu(x)}=\sum\limits_{i=1}^{n}{\lambda_i^j\tau_i}, j=0,\ldots,2n-1
\end{eqnarray}
Якобиан обратного отображения (от квадратуры к обычным моментам) известен
и равен $\Phi=\Xi\cdot\Lambda$:
\begin{equation}
\Xi=
\left[
\begin{array}{cccccccccc}
1 &  \ldots & 1 & 0  & \ldots & 0 \\
\lambda_1 & \ldots & \lambda_{n} & 1 & \ldots & 1 \\
\lambda^2_1 & \ldots & \lambda^2_{n} & 2\lambda_1 & \ldots & 2\lambda_n\\
\ldots & \ldots & \ldots & \ldots & \ldots & \ldots  \\
\lambda^n_1 & \ldots & \lambda^n_{n} & n\lambda^{n-1}_1 & \ldots & n\lambda^{n-1}_n\\
\ldots & \ldots & \ldots & \ldots & \ldots & \ldots  \\
\lambda^{2n-1}_1 & \ldots & \lambda^{2n-1}_{n} & (2n-1)\lambda^{2n-2}_1 & \ldots & (2n-1)\lambda^{2n-2}_{n}\\
\end{array}
\right]
\end{equation}
где $\Lambda=diag(1,\ldots,1,\tau_1,\ldots,\tau_n)$
Следовательно имеем:
\begin{equation}
\mbox{cond}G_n^{0}=\frac
{\parallel (s_0,\ldots,s_{2n-1}) \parallel}
{\parallel (\lambda_1,\ldots,\lambda_{n},\tau_{1},\ldots,\tau_{n})\parallel}
\parallel \Lambda^{-1}\Xi^{-1} \parallel
\end{equation}
\bf Теорема \rm
\it Пусть $\lambda_1,\ldots,\lambda_n$ взаимно положительны, определим
в качестве нормы матрицы максимальную сумму модулей элементов по строкам.
Тогда:
\begin{equation}
u_1\leq \parallel \Xi^{-1} \parallel \leq max(u_1,u_2)
\end{equation}
где
\begin{equation}
u_1=\max\limits_{1\leq i \leq n}b^{(1)}_i\prod\limits_{k=1,k\not=i}^{n}
\left(
\frac{1+\lambda_k}
{\lambda_i-\lambda_k}
\right)^2, \quad b_i^{(1)}=1+\lambda_i
\end{equation}

\begin{eqnarray}
u_2=\max\limits_{1\leq i \leq n}b^{(2)}_i
\prod\limits_{k=1,k\not = i}^{n}
\left(
\frac{1+\lambda_k}
{\lambda_i-\lambda_k}\right)^2, \quad
b_i^{(2)}=
1+2(1+\lambda_i)
\left|
\sum\limits_{k=1,k\not=i}^{n}
{
\frac{1}
{\lambda_i-\lambda_k}
}
\right|
\end{eqnarray}
\rm
Доказательство: \\
Известно, что
\begin{equation}
\Xi^{-1}=\left[
\begin{array} {ccccc}
A \\
B
\end{array}
\right],A=(a_{i,j}),B=(b_{i,j})_{i,j=1,\ldots,2n}
\end{equation}
где
\begin{eqnarray}
\sum\limits_{j=1}^{2n}{\left| a_{i,j}\right|} \leq
b^{(2)}_i\prod\limits_{k\not= i}{\left(
\frac{1+\lambda_k}{\lambda_i-\lambda_k}\right)^2} \nonumber \\
\sum\limits_{j=1}^{2n}{\left| b_{i,j}\right|} =
 b^{(1)}_i\prod\limits_{k\not= i}{\left(
\frac{1+\lambda_k}{\lambda_i-\lambda_k}\right)^2} \nonumber
\end{eqnarray}
Откуда и следует вышесказанное.
Конечная оценка Гаучи
\begin{equation}
\mbox{cond}G_n^{0}>min(s_0,\frac{1}{s_0})\frac{(17+6\sqrt{8})^n}{64n^2}
\end{equation}

\section{Число обусловленности $G_n$}

Источник ~\cite{GautschiW5}:
\begin{equation}
\mbox{cond}_{D_{nor}}G_n=\frac{\parallel (\tilde{m}_0,\ldots,\tilde{m}_{2n-1}) \parallel_2}
{\parallel (\lambda_1,\ldots,\lambda_{n},\tau_{1},\ldots,\tau_{n})\parallel_2}
\sqrt{\int {\sum\limits_{i=1}^{n}\left(
h_i^2(x)+\frac{1}{\tau_i^2}k_i^2(x)
\right)dl(x)}}
\end{equation}
где $\parallel (\tilde{m}_0,\ldots,\tilde{m}_{2n-1}) \parallel_2=
\displaystyle\sum\limits_{i=0}^{2n-1}{\frac{1}{\tilde{m}^2_i}}$ и
$\parallel (\lambda_1,\ldots,\lambda_{n},\tau_{1},\ldots,\tau_{n})\parallel_2=
\sum\limits_{i=1}^{n}{(\lambda_i^2+\tau_i^2)}$ \\
Запишем из определения нормализованных модифицированных моментов
\begin{eqnarray}
\tilde{m}_k=\displaystyle \frac{\int \pi_k d\mu(x)}{\sqrt{\int\pi_k^2(x) dl(x))}}=
\frac{1}{\sqrt{d_k}}\sum\limits_{i=0}^{2n-1}{\pi_k(\lambda_i)\tau_i},k=0,1,\ldots,2n-1 \nonumber
\end{eqnarray}
Якобиан обратного отображения из выше сказанного равен
$\Phi=D^{-1}\Xi\Lambda$, где
$D=diag(\sqrt{d_0},\ldots,\sqrt{d_{2n-1}}),\Lambda=diag(1,\ldots,1,\tau_1,\ldots,\tau_n)$ и
\begin{equation}
\Xi=\left[
\begin{array}{cccccccccc}
\pi_0(\lambda_1) & \ldots & \pi_0(\lambda_n) & \pi^{'}_0(\lambda_1) & \ldots & \pi^{'}_0(\lambda_n) \\
\pi_1(\lambda_1) & \ldots & \pi_1(\lambda_n) & \pi^{'}_1(\lambda_1) & \ldots & \pi^{'}_1(\lambda_n) \\
\ldots & \ldots & \ldots & \ldots & \ldots & \ldots  \\
\pi_{2n-1}(\lambda_1) & \ldots & \pi_{2n-1}(\lambda_n) & \pi^{'}_{2n-1}(\lambda_1) & \ldots & \pi^{'}_{2n-1}(\lambda_n) \\
\end{array}
\right]
\end{equation}
Число обусловленности соответственно:
\begin{eqnarray}
\mbox{cond}_{D_{nor}}G_n=\frac{\parallel (\tilde{m}_0,\ldots,\tilde{m}_{2n-1}) \parallel_2}
{\parallel (\lambda_1,\ldots,\lambda_{n},\tau_{1},\ldots,\tau_{n})\parallel_2}
\parallel\Phi^{-1}(\lambda,\tau) \parallel
\end{eqnarray}
далее $\parallel \Phi^{-1} \parallel=\parallel \Lambda^{-1} \Xi^{-1} D \parallel
\leq \parallel \Lambda^{-1} \Xi^{-1} D \parallel_F$
Из предыдущей главы
\begin{equation}
\Xi^{-1}=\left[
\begin{array}{ccccc}
A \\
B
\end{array}
\right],
(\Lambda^{-1}\Xi^{-1}D)=\sqrt{d}\left[
\begin{array}{cccccccc}
A \\
\displaystyle\frac{1}{\tau}B
\end{array}
\right]
\end{equation}
Соответственно
\begin{equation}
\parallel \Lambda^{-1} \Xi^{-1}D \parallel^2_F=
\sum\limits_{i=1}^n
{
\sum\limits_{j=1}^{2n}
{
d_{j-1}
\left(
a_{i,j}^2+\displaystyle\frac{1}{\tau_i^2}b_{i,j}^2
\right)
}}
\end{equation}
Далее
\begin{eqnarray}
h_k(x)=\sum\limits_{i=1}^{2n}{a_{k,i}\pi_{i-1}(x)}, k_n=\sum\limits_{i=1}^{2n}{b_{k,i}\pi_{i-1}(x)} \nonumber \\
\int{\pi^2_k(x)dl(x)}=\sum\limits_{i=1}^{2n}{d_{i-1}a_{k,i}^2},\int{k_k^2(x)dl(x)}=\sum\limits_{i=1}^{2n}{d_{i-1}b_{k,i}^2} \nonumber \\
\parallel \Lambda^{-1} \Xi^{-1} D \parallel_F^2=\int {\sum\limits_{i=1}^{n}\left(
h_i^2(x)+\frac{1}{\tau_i^2}k_i^2(x)
\right)dl(x)} \nonumber
\end{eqnarray}
Отметим, что интеграл является полиномом степени не выше $4n-2$


\section{Число обусловленности отображения $H_n (p=2)$}

Источник ~\cite{Beckermann1} \\
\begin{equation}
\mbox{cond}_{D_{opt}}H_n\leq 6\sqrt{2n}\left[
n+\sqrt{
\left(
\mu^2\sum\limits_{j=1}^{n}\sum\limits_{k=1}^{n}
\frac{1}{\tau_k}\prod_{i=1,i\not=k}^{n}{
\left(
\frac{\lambda_j-\lambda_i}{\lambda_k-\lambda_i}\right)^2
}\right)}
\right]
\end{equation}


\section{Число обусловлености отображения $K_n (p=2)$  }

Источник ~\cite{Fischer1}:
\begin{equation}
\mbox{cond}_{D_{nor}}K_n=\frac{\parallel (\tilde{m}_0,\ldots,\tilde{m}_{2n-1}) \parallel_2}
{\parallel (\alpha_0,\ldots,\alpha_{n-1},\beta_{0},\ldots,\beta_{n-1})\parallel_2}
\sqrt{\sum\limits_{j=0}^{2n-1} {w_2}}
\end{equation}
, где $w_2=\sum\limits_{j=0}^{n-1}{\psi_{2j}^2+\psi_{2j+1}^2}=
\sum\limits_{j=0}^{n-1}{\beta_j^2(q_j^2-q_{j-1}^2)^2+
(\sqrt{\beta_{j+1}}q_jq_{j+1}-\sqrt{\beta_j}q_{j-1}q_j)^2}$
В ~\cite{Beckermann1} на основе точных формул Фишера приведены следующие оценки для мер
с компактным расположением. $\mu(x)$ - мера, соответствующая обычным моментам,
$l(x)$ - мера, соответствующая модифицированным моментам. \\


\subsection{ Вычисление частных производных $\frac{\partial \alpha_j}{\partial m_k}$ и $\frac{\partial \beta_j}{\partial m_k}$ }
Пусть ($\pi_0,\ldots,\pi_N$) некоторый базис в пространстве $\bf P \rm_N$ полиномов степени не выше $N$.
Для любого полинома $q\in \bf P \rm_N$ степени не выше $N$ можно записать разложение по базису:
\begin{equation}
\label{Basisq}
q(x)=\sum\limits_{j=0}^{N}{W_j(q(x))\pi_j(x)}
\end{equation}
где $W_j$ - некоторый линейный функционал.
В случае обычных моментов:
\begin{eqnarray}
q(x)=\sum\limits_{j=0}^{N} { \frac { q^{(j)}(x)} {j!}  x^j} \nonumber \\
\pi_j(x)=x^j, \mbox{   } W_j(q(x))=\frac { q^{(j)}(x)} {j!} \nonumber
\end{eqnarray}
\bf Лемма 1 \rm \\
Из (~\ref{Ord Mod moments}) и (~\ref{Basisq}) следует $\int{q(x)d\mu(x)}=\sum\limits_{j=0}^{N}{W_j(q)m_j}$ \\
Пусть $q$ - зависящие от модифицированных моментов имеют непрерывные частные производные в некоторой окрестности $m$. \\
Тогда,
\begin{equation}
\frac{\partial}{\partial m_k} \int q(x) d\mu(x)=W_k(q)+\int { \frac {\partial q} {\partial m_k}  d\mu(x)}
\end{equation}
\bf Доказательство: \rm
\begin{eqnarray}
\frac{\partial}{\partial m_k} \int q(x) d\mu(x)=
W_k(q)+\sum\limits_{j=0}^{N} { \frac {\partial W_j(q)} {\partial m_k} m_j}= \nonumber \\
W_k(q)+\sum\limits_{j=0}^{N} { W_k \left( \frac {\partial q} {\partial m_k} \right) m_j}=
W_k(q)+\int { \frac {\partial q} {\partial m_k}  d\mu(x)} \nonumber
\end{eqnarray}
Для $(i<j)$ из (~\ref{Orthq}) получаем :
\begin{equation}
\label{Orth1}
\int q_i(x)\frac{\partial q_j(x)}{\partial m_k}d\mu(x)=-W_k(q_iq_j)
\end{equation}
\begin{eqnarray}
\frac{\partial}{\partial m_k} \int q_i(x)q_j(x)d\mu(x)=
W_k(q_iq_j)+\int \frac{\partial q_i(x)}{\partial m_k}q_j(x)d\mu(x)+\int q_i(x)\frac{\partial q_j(x)}{\partial m_k}d\mu(x)= \nonumber \\
W_k(q_iq_j)+\int q_i(x)\frac{\partial q_j(x)}{\partial m_k}d\mu(x)=0 \nonumber
\end{eqnarray}
Из второго соотношения (~\ref{Orthq}) получаем:
\begin{equation}
\label{Orth2}
\int q_j(x)\frac{\partial q_j(x)}{\partial m_k}d\mu(x)=-\frac{1}{2}W_k(q_j^2)
\end{equation}
\begin{eqnarray}
\frac{\partial}{\partial m_k} \int q_j^2(x)d\mu(x)=W_k(q_j^2)+2\int{q_j\frac{\partial q_j(x)}{\partial m_k}d\mu(x)}=0 \nonumber
\end{eqnarray}
Перепишем и продифференцируем рекуррентное соотношение:
\begin{eqnarray}
\label{Rec1}
\beta_{j+1}^{1/2}q_{j+1}(x)=(x-\alpha_j)q_j(x)-\beta_{j}^{1/2}q_{j-1}(x) \nonumber \\
\frac {\partial \beta_{j+1}^{1/2}} {\partial m_k} q_{j+1}(x)+
\beta_{j+1}^{1/2} \frac {\partial q_{j+1}} {\partial m_k}= \nonumber \\
-\frac {\alpha_j} {\partial m_k} q_j(x)+
(x-\alpha_j)\frac {q_j(x)} {\partial m_k}-
\frac {\beta_j^{1/2}} {\partial m_k}q_{j-1}(x)
-\beta_j^{1/2}\frac {q_{j-1}(x)} {\partial m_k}
\end{eqnarray}
Из рекуррентного соотношения:
\begin{eqnarray}
\beta_{j+2}^{1/2}q_{j+2}(x)=(x-\alpha_{j+1})q_{j+1}(x)-\beta_{j+1}^{1/2}q_{j}(x) \nonumber \\
\beta_{j+2}^{1/2}q_{j+2}(x)=(x-\alpha_{j+1}+\alpha_j-\alpha_j)q_{j+1}(x)-\beta_{j+1}^{1/2}q_{j}(x) \nonumber \\
(x-\alpha_j)q_{j+1}(x)=\beta_{j+2}^{1/2}q_{j+2}(x)+(\alpha_{j+2}-\alpha_{j+1})q_{j+1}(x)+\beta_{j+1}^{1/2}q_{j}(x) \nonumber
\end{eqnarray}
Домножим (~\ref{Rec1}) на $q_{j+1}(x)$:
\begin{eqnarray}
\frac {\partial \beta_{j+1}^{1/2}} {\partial m_k}
+\beta_{j+1}^{1/2} \frac {\partial q_{j+1}(x) } {\partial m_k}q_{j+1}(x)
 = (x-\alpha_j)q_{j+1}(x)\frac {q_j(x)} {\partial m_k} \nonumber
\end{eqnarray}
Подставляем выражение для $(x-\alpha_j)q_{j+1}(x)$:
\begin{eqnarray}
\frac {\partial \beta_{j+1}^{1/2}} {\partial m_k}
+\beta_{j+1}^{1/2} \frac {\partial q_{j+1}(x) } {\partial m_k}q_{j+1}(x)
 =\beta_{j+1}^{1/2} q_{j}(x)\frac {q_j(x)} {\partial m_k} \nonumber
\end{eqnarray}
Проинтегрируем полученное выражение и учтем (~\ref{Orth2})
\begin{eqnarray}
\frac {\partial \beta_{j+1}^{1/2}} {\partial m_k}-\frac{1}{2}\beta_{j+1}^{1/2}W_k(q_{j+1}^2)=-\frac{1}{2}W_k(q_k^2) \nonumber \\
\frac {\partial \beta_{j+1}^{1/2}} {\partial m_k}=\frac{1}{2}\beta_{j+1}^{1/2}W_k(q_{j+1}^2-q_k^2) \nonumber
\end{eqnarray}
Домножим на $2\beta^{1/2}$
\begin{equation}
\frac {\partial \beta_{j+1}} {\partial m_k}=\beta_{j+1}W_k(q_{j+1}^2-q_k^2)
\end{equation}
Домножим (~\ref{Rec1}) на $q_{j}(x)$
\begin{eqnarray}
\frac {\partial \beta_{j+1}^{1/2}} {\partial m_k}q_{j+1}(x)q_j(x)
+\beta_{j+1}^{1/2} \frac {\partial q_{j+1}(x) } {\partial m_k}q_{j}(x)
 = -\frac {\alpha_j} {\partial m_k} q_j(x)q_j(x)+
(x-\alpha_j)q_j(x) \frac {\partial q_j(x)} {\partial m_k} \nonumber
\end{eqnarray}
Учитывая, что
$(x-\alpha_j)q_j(x)=\beta_{j+1}^{1/2}q_{j+1}(x)+\beta_{j}^{1/2}q_{j-1}(x)$
перепишем:
\begin{eqnarray}
\frac {\partial \beta_{j+1}^{1/2}} {\partial m_k}q_{j+1}(x)q_j(x)
+\beta_{j+1}^{1/2} \frac {\partial q_{j+1}(x) } {\partial m_k}q_{j}(x)
 = -\frac {\alpha_j} {\partial m_k} q_j(x)q_j(x)+
\beta_{j}^{1/2}\frac {\partial q_j(x)} {\partial m_k}q_{j-1}(x) \nonumber
\end{eqnarray}
Интегрируя полученное выражение и учитывая (~\ref{Orth2}) получаем:
\begin{equation}
-\beta_{j+1}^{1/2}W_k(q_{j+1}q_j)=-\frac {\alpha_j} {\partial m_k}
-\beta_j^{1/2}W_k(q_jq_{j-1})
\end{equation}
\bf Теорема 1 \rm \\
При выполнении  (~\ref{Orthq}), (~\ref{Ord Mod moments}), (~\ref{Basisq})
частные производные для коэффициентов рекуррентного
соотношения выражаются как:
\begin{equation}
\frac{\partial \alpha_j} {\partial m_k}=\beta_{j+1}^{1/2}W_k(q_jq_{j+1})-\beta_{j}^{1/2}W_k(q_{j-1}q_{j}), \mbox {  для  } 2j+1 \leq N
\end{equation}
\begin{equation}
\frac{\partial \beta_j} {\partial m_k}=\beta_jW_k(q_j^2-q_{j-1}^2), \mbox{ для } 2j \leq N
\end{equation}

\subsection{Норма якобиана отображения $K_n$}

Якобиан $K^{'}_n[2n\times 2n]$ имеет следующий вид:
\begin{equation}
K^{'}_n=
\left(
\begin{array}{ccccccccccccc}
\displaystyle\frac{\partial \alpha_0} {\partial m_0} &
\displaystyle\frac{\partial \alpha_1} {\partial m_0} & \cdots &
\displaystyle\frac{\partial \alpha_{n-1}} {\partial m_0} &
\displaystyle\frac{\partial \beta_0} {\partial m_0} &
\displaystyle\frac{\partial \beta_1} {\partial m_0} & \cdots &
\displaystyle\frac{\partial \beta_{n-1}} {\partial m_0} \\

\displaystyle\frac{\partial \alpha_0} {\partial m_1} &
\displaystyle\frac{\partial \alpha_1} {\partial m_1} & \cdots &
\displaystyle\frac{\partial \alpha_{n-1}} {\partial m_1} &
\displaystyle\frac{\partial \beta_0} {\partial m_1} &
\displaystyle\frac{\partial \beta_1} {\partial m_1} & \cdots &
\displaystyle\frac{\partial \beta_{n-1}} {\partial m_1} \\
\cdots & \cdots & \cdots & \cdots & \cdots & \cdots & \cdots & \cdots & \\
\displaystyle\frac{\partial \alpha_0} {\partial m_{2n-1}} &
\displaystyle\frac{\partial \alpha_1} {\partial m_{2n-1}} & \cdots &
\displaystyle\frac{\partial \alpha_{n-1}} {\partial m_{2n-1}} &
\displaystyle\frac{\partial \beta_0} {\partial m_{2n-1}} &
\displaystyle\frac{\partial \beta_1} {\partial m_{2n-1}} & \cdots &
\displaystyle\frac{\partial \beta_{n-1}} {\partial m_{2n-1}} \\
\end{array}
\right)
\end{equation}

\begin{eqnarray}
\psi_{2j}(x)=\beta_j(q^2_j(x)-q^2_{j-1}(x)) \nonumber \\
\psi_{2j+1}(x)=\beta_{j+1}^{1/2}q_j(x)q_{j+1}(x)-\beta_j^{1/2}q_{j-1}(x)q_{j}(x) \nonumber \\
 j=0,\ldots,n-1 \nonumber
\end{eqnarray}

\begin{equation}
K^{'}_n=\Psi=(\psi_{i,j})_{i,j=0,2n-1},\mbox{   } \psi_{i,j}=W_j(\psi_i)
\end{equation}
Введем следующие обозначения:
\begin{eqnarray}
w_{\infty}(\psi_j)=\sum\limits_{k=0}^{N} {\mid W_k(\psi_j) \mid} \nonumber \\
w_{F}(\psi_j)=\sum\limits_{k=0}^{N} {W_k^2(\psi_j)} \nonumber
\end{eqnarray}
Нормы якобиана $K^{'}_n$ выражаются:
\begin{equation}
\parallel K^{'}_n \parallel _{\infty}=\parallel \Psi  \parallel _{\infty}=
\max\limits_{0\leq j \leq 2n-1} w_{\infty}(\psi_j)
\end{equation}
\begin{equation}
\parallel K^{'}_n \parallel _{F}=\parallel \Psi  \parallel _{F}=
\sqrt{ \sum\limits_{j=0}^{2n-1}{w_F(\psi_j)} }
\end{equation}
\bf Лемма 2. \rm \\
Для обычных моментов \\
1. Если $\psi_j$ - многочлен с чередующимися по знаку элементами,
то $ w_{\infty}(\psi_j)=\mid \psi(-1) \mid$ \\
2. Если $\psi_j$ - многочлен только с четными (или только нечетными)
степенями и чередующимся знаком, то $ w_{\infty}(\psi_j)=\mid \psi_j(i) \mid$ \\
3. В общем случае $w_F$ может быть выражено как $w_F(\psi_j)=\frac{1}{2\pi}\int\limits_{0}^{2\pi}{\mid \psi_j(e^{i\phi}) \mid ^{2}d\phi}$


\section{Добавление масс. Вычисление новых рекуррентных коэффициентов}

При рассмотрении отображения $H_n$: как изменяются коэффициенты
рекуррентного соотношения при добавлении массы ? \\
Стартуем с результата Неваи ~\cite{Nevai}: \\
\bf Лемма Неваи \rm \\
\it Пусть $q(x)$ - ортонормированные многочлены относительно некоторой меры
$\mu(x)$. \\
Пусть $\tilde{\mu}(x)=\mu(x)+\lambda \delta_{\tau}$ - новая мера, получающаяся из исходной
добавлением массы $\lambda$ в точке $\tau$.
Соответствуищие ортогональные многочлены
$\tilde{q(x)}$, коэффициенты рекуррентного соотношения для которых - $\tilde{\alpha}$ и $\tilde{\beta}$.
тогда имеет место следующее соотношение: \rm
\begin{eqnarray}
\label{Alpha}
\tilde{\alpha}_j=\alpha_j+
\lambda\frac{\sqrt{\beta_{j+1}} q_j(\tau)q_{j+1}(\tau)} { 1+ \lambda\sum\limits_{i=0}^{j}{q_i^2(\tau)}}-
\lambda\frac{\sqrt{\beta_{j}} q_j(\tau)q_{j-1}(\tau)} { 1+ \lambda\sum\limits_{i=0}^{j-1}{q_i^2(\tau)}} \\
\tilde{\beta}_j=\beta_j
\frac
{ \left[
1+ \lambda\sum\limits_{i=0}^{j-2}{q_i^2(\tau)}
\right]
\left[
1+ \lambda\sum\limits_{i=0}^{j}{q_i^2(\tau)}
\right] }
{
\left[
1+ \lambda\sum\limits_{i=0}^{j-1}{q_i^2(\tau)}
\right]
},j<N
\end{eqnarray}
При вычислении проще воспользоваться многочленами $Q_n$.  \\
Однако, если $\tau$ отлично от всех $(\tau_j)_0^{N}$ и все
$\lambda_j$ и $\lambda$ положительны, т.е.
мы имеем дополнительную точку массы, тогда
существует $\tilde{\alpha_{N}}$, которую невозможно вычислить
используя лемму Неваи (так как коэффициент $\alpha_N$ не определен). \\
Приведем две леммы из ~\cite{Fischer2}: \\
\bf Лемма 1. \rm \\
\it Коэффициент многочлена $Q_n(x)$ при $x^{n-1}$ равняется $-\sum\limits_{j=0}^{n-1}{\alpha_j}$ \rm. \\
\bf Лемма 2. \rm \\
\it Пусть в дискретном представлении меры $\mu(x)$ (~\ref{Discrete})
все веса $\lambda_j>0,j=1,\ldots,N$ и все узлы $\tau_j,j=1,\ldots,N$ разные,
тогда $\sum\limits_{j=0}^{N-1}{\alpha_j}=\sum\limits_{j=1}^{N}{\tau_j}$ \rm \\
теперь приведем основной результат ~\cite{Fischer2}: \\
\bf Теорема 1. \rm \\
\it Пусть в дискретном представлении меры $\mu(x)$ (~\ref{Discrete})
все веса $\lambda_j>0,j=1,\ldots,N$ и все узлы $\tau_j,j=1,\ldots,N$ разные,
добавим дополнительную точку $(\lambda,\tau),\lambda>0,\tau \not = \tau_j,j=1,\ldots,N$,
тогда:
\begin{eqnarray}
\label{FischerMass}
\tilde{\alpha}_j=\alpha_j+\lambda
\frac {\gamma_j^2 Q_j(\tau) Q_{j+1}(\tau)}
{1+\lambda\sum\limits_{i=0}^{j}{\gamma_i^2 Q_i^2(\tau)}}-
\lambda\frac {\gamma_{j-1}^2 Q_j(\tau) Q_{j-1}(\tau)}
{1+\lambda\sum\limits_{i=0}^{j-1}{\gamma_i^2 Q_i^2(\tau)}} \nonumber \\
\tilde{\beta}_j=\beta_j
\frac {
\left[
1+ \lambda\sum\limits_{i=0}^{j-2}{\gamma_i^2 Q_i^2(\tau)}
\right]
\left[
1+ \lambda\sum\limits_{i=0}^{j}{\gamma_i^2 Q_i^2(\tau)}
\right]
}
{
\left[
1+ \lambda\sum\limits_{i=0}^{j-1}{\gamma_i^2 Q_i^2(\tau)}
\right]
}, j<N \\
\tilde{\alpha}_N=\tau-\lambda
\frac {\gamma_{N-1}^2 Q_N(\tau) Q_{N-1}(\tau)}
{1+\lambda\sum\limits_{i=0}^{N-1}{\gamma_i^2 Q_i^2(\tau)}},
\tilde{\beta}_N=\lambda
\frac {
\gamma_{N-1}^2Q_N^2(\tau)
\left[
1+ \lambda\sum\limits_{i=0}^{N-2}{\gamma_i^2 Q_i^2(\tau)}
\right]
}
{
\left[
1+ \lambda\sum\limits_{i=0}^{N-1}{\gamma_i^2 Q_i^2(\tau)}
\right]
} \nonumber
\end{eqnarray}
\bf Доказательство: \rm  \\
Пусть $\tilde{Q}_n(x)$ - ортогональные многочлены (со старшим коэффицентом единица)
для новой меры $\tilde{\mu}(x)=\mu+\lambda\delta_{\tau}$:
\begin{eqnarray}
\label{QMass}
\tilde{Q}_j(x)=Q_j(x)+\sum\limits_{l=0}^{j-1}{c_{j,l}Q_l(x)} \nonumber
\end{eqnarray}
$c_{jl}$ - некоторые неизвестные коэффициенты
разложения по базису $(Q_n)_0^{\infty}$.\\
Домножим $\tilde{Q}_j(x)$ на $Q_l(x),l<j$ и проинтегрируем,
применив функционал $\tilde{L}(f)=\int f d\tilde{\mu}(x)$,
правая часть выражение станет равной нулю:
\begin{eqnarray}
\tilde{L}(\tilde{Q}_j(x)Q_l(x))=L(\tilde{Q}_j(x)Q_l(x))+\lambda\tilde{Q}_j(\tau)Q_l(\tau)= \nonumber \\
L(\left[Q_j(x)+\sum\limits_{l=0}^{j-1}{c_{j,l}Q_l(x)}\right]Q_l(x))+\lambda\tilde{Q}_j(\tau)Q_l(\tau)= \nonumber \\
L(\left[\sum\limits_{l=0}^{j-1}{c_{j,l}Q_l(x)}\right]Q_l(x))+\lambda\tilde{Q}_j(\tau)Q_l(\tau)=
c_{j,l}\frac{1}{\gamma_l^2}+\lambda\tilde{Q}_j(\tau)Q_l(\tau)=0 \nonumber
\end{eqnarray}
Откуда:
\begin{equation}
\label{KoefC}
c_{j,l}=-\lambda\gamma_{l}^2 \tilde{Q}_j(\tau)Q_l(\tau)
\end{equation}
Вставим (~\ref{KoefC}) в (~\ref{QMass}) $x=\tau$ и выразим $\tilde{Q}_j(\tau)$:
\begin{equation}
\label{Eq15}
\tilde{Q}_j(\tau)=\frac{Q_j(\tau)}
{1+\lambda\sum\limits_{i=0}^{j-1}{\gamma_i^2 Q_i^2(\tau) }}
\end{equation}
Далее:
\begin{equation}
\label{Eq16}
c_{j,l}=-\frac
{\lambda \gamma_l^2 Q_j(\tau) Q_l(\tau)}
{1+\lambda\ \sum\limits_{i=0}^{j-1} {\gamma_i^2 Q_i^2(\tau)}}
\end{equation}
Сравним коэффициенты в (~\ref{QMass}) при $x^{j-1}$ и применим Лемму 1.
\begin{equation}
\label{AlphaPre}
-\sum\limits_{i=0}^{j-1}{\tilde{\alpha}_i}=-\sum\limits_{i=0}^{j-1}{\alpha_i}
+c_{j,j-1}=-\sum\limits_{i=0}^{j-1}{\alpha_j}-\frac
{\lambda \gamma_{j-1}^2 Q_j(\tau) Q_{j-1}(\tau)}
{1+\lambda\ \sum\limits_{i=0}^{j-1} {\gamma_i^2 Q_i^2(\tau)}},j\leq N
\end{equation}
Вспомним Лемму 2:
\begin{eqnarray}
\sum\limits_{i=0}^{N-1}{\alpha_i}=\sum_{i=1}^{N}{\tau_i},  \nonumber \\
\sum\limits_{i=0}^{N-1}{\tilde{\alpha}_i}=\sum_{i=1}^{N}{\tau_i}+\tau=\sum\limits_{i=0}^{N-1}{\alpha_i}+\tau, \nonumber
\end{eqnarray}
Далее учитывая (~\ref{AlphaPre})
\begin{eqnarray}
\tilde{\alpha}_N=\sum\limits_{i=0}^{N-1}{\alpha_i}-\sum\limits_{i=0}^{N-1}{\tilde{\alpha}_i}+\tau \nonumber \\
=\sum\limits_{i=0}^{N-1}{\alpha_i}-\sum\limits_{i=0}^{N-1}{\alpha_i}+c_{N,N-1}+\tau=c_{N,N-1}+\tau \nonumber \\
=\tau-\lambda
\frac {\gamma_{N-1}^2 Q_N(\tau) Q_{N-1}(\tau)}
{1+\lambda\sum\limits_{i=0}^{N-1}{\gamma_i^2 Q_i^2(\tau)}}
\end{eqnarray}
Отсюда, используя (~\ref{AlphaPre}) для индекса $N$ можно выразить $\tilde{\alpha}_N$. \\
Для вывода выражения для $\tilde{\beta}_N$ воспользуемся (~\ref{QMass})
\begin{eqnarray}
\tilde{L}(\tilde{Q}_j^2(x))= L(\tilde{Q}_j^2(x))+\lambda\tilde{Q}_j^2(\tau) \nonumber \\
=L(Q_j^2(x))+\sum\limits_{l=0}^{j-1}{c_{j,l}^2 L(Q_j(x)Q_l(x))}+\sum\limits_{l=0}^{j-1}{c_{j,l}^2L(Q_l^2(x))}+\lambda\tilde{Q}_j^2(\tau) \nonumber \\
=L(Q_j^2(x))+\sum\limits_{l=0}^{j-1}{c_{j,l}^2L(Q_l^2(x))}+\lambda\tilde{Q}_j^2(\tau) \nonumber \\
=\frac{1}{\gamma_j^2}+\sum\limits_{l=0}^{j-1}
{\left[
-\frac {\lambda \gamma_l^2 Q_j(\tau)Q_l(\tau)}
{1+\lambda \sum\limits_{i=0}^{j-1}{\gamma_i^2 Q_i^2(\tau)}}
\right]^2\frac{1}{\gamma_l^2}}+
\lambda
\left[
\frac{Q_j(\tau)}
{1+\lambda \sum\limits_{i=0}^{j-1}{\gamma_i^2 Q_i^2(\tau)}}
\right]^2 \nonumber
\end{eqnarray}
Общий знаменатель уже есть, выносим за скобки $\lambda Q_j^2(\tau)$
\begin{eqnarray}
\tilde{L}(\tilde{Q}_j^2)=\frac{1}{\gamma_j^2}+
\frac{\lambda Q_j^2(\tau) \left[
\sum\limits_{l=0}^{j-1}{\gamma_l^2Q_l^2(\tau)+1}
\right]}
{\left[
1+\sum\limits_{l=0}^{j-1}{\gamma_l^2Q_l^2(\tau)}
\right]^2}=
\frac{1}{\gamma_j^2}+
\frac{\lambda Q_j^2(\tau)}
{1+\sum\limits_{l=0}^{j-1}{\gamma_l^2Q_l^2(\tau)}}
\end{eqnarray}
Далее учитывая $L(Q_j^2(x))=\gamma_j^{-2},j<N\mbox{     }L(Q_N^2(x))=0$, (~\ref{Eq15}) и (~\ref{Eq16})
\begin{eqnarray}
\frac{1}{\tilde{\gamma}^2}=\tilde{L}(\tilde{Q}_j^2(x))=\frac{1}{\gamma_j^2}+
\frac{\lambda Q_j^2(\tau)} {1+\lambda\sum\limits_{i=0}^{j-1}{\gamma_i^2 Q_i^2(\tau)}}
=\frac{1}{\gamma_j^2}\frac{1+\lambda \sum\limits_{i=0}^{j}{\gamma_i^2 Q_i^2(\tau)}}
{1+\lambda \sum\limits_{i=0}^{j-1}{\gamma_i^2 Q_i^2(\tau)}},j<N \nonumber
\end{eqnarray}
\begin{equation}
\tilde{L}(\tilde{Q}_N^2(x))=\frac{\lambda Q_N^2(\tau)}
{1+\lambda \sum\limits_{i=0}^{N-1}{\gamma_i^2 Q_i^2(\tau)}}
\end{equation}
Отсюда
\begin{eqnarray}
\tilde{\beta}_N=
\frac{\tilde{L}(\tilde{Q}_{N}^2)}
{\tilde{L}(\tilde{Q}_{N-1}^2)}=
\frac {\lambda Q_N^2(\tau)}
{1+\lambda \sum\limits_{i=0}^{N-1}{\gamma_i^2 Q_i^2(\tau)}}
\gamma_{N-1}^2
\frac {1+\lambda \sum\limits_{i=0}^{N-2}{\gamma_i^2 Q_i^2(\tau)}}
{1+\lambda \sum\limits_{i=0}^{N-1}{\gamma_i^2 Q_i^2(\tau)}} \nonumber \\
=\frac
{\lambda \gamma_{N-1}^2 Q_{N}^2( \tau)\left[1+\sum\limits_{i=0}^{N-2}{\gamma_i^2 Q_i^2(\tau)}\right]}
{\left[1+\lambda \sum\limits_{i=0}^{N-1}{\gamma_i^2 Q_i^2(\tau)}\right]^2} \nonumber
\end{eqnarray}
Используя (~\ref{FischerMass}) можно последовательно добавлять
несколько точек масс.

\subsection{Чувствительность отображения $H_n, (p=2)$ при добавлении массы}

\bf Лемма 3. \rm \\
\it Введем следуещее обозначние $\frac{\partial L}{\partial t}=W$, где
$t$ некоторый параметр от которого зависит функционал $L$, тогда
\begin{equation}
\frac{\partial \alpha_j} {\partial t}=\gamma_j^2W(Q_jQ_{j+1})-\gamma_{j-1}^2W(Q_{j-1}Q_j)
\end{equation}
\begin{equation}
\frac{\partial \beta_j} {\partial t}=\beta_j(\gamma_j^2W(Q_j^2)-\gamma_{j-1}^2W(Q_{j-1}^2))
\end{equation}
\rm
\bf Доказательство: \rm \\
Из процедуры Стилтьеса:
\begin{eqnarray}
\alpha_j=\frac{L(xQ_j^2(x))}{L(Q_j^2)},\mbox{   }\beta_j=\frac{L(Q_j^2(x))}{L(Q_{j-1}^2)}=\frac{\gamma_{j-1}^2}{\gamma_j^2} \nonumber
\end{eqnarray}
Так старший коэффициентов многочлена $Q_j$ равен единице
уместно следуещее разложение в ряд Фурье
(или по базису $(Q_0,\ldots,Q_{j-1})$):
\begin{eqnarray}
Q_j(x)=\sum\limits_{i=0}^{j-1}{c_{j,i}Q_i(x)} \nonumber \\
c_{j,i}=
L
\left(
\frac {\partial Q_j}  {\partial t}
Q_i
\right)
\frac{1}{L(Q_i^2)}
=
\gamma_i^2
L
\left(
\frac{\partial Q_j} {\partial t}
Q_i
\right)
\nonumber
\end{eqnarray}
С другой стороны
\begin{eqnarray}
\label{pDerQ}
\frac{\partial} {\partial t} L(Q_jQ_i)=
L\left(
\frac{\partial Q_j} {\partial t} Q_i
\right)
+L\left(
Q_j \frac{\partial Q_i} {\partial t}
\right)+
W(Q_jQ_i)=0,i<j
\end{eqnarray}
где $L\left(
Q_j \frac{\partial Q_i} {\partial t}
\right)$ также равно нулю согласно условию ортогональности. \\
В результате:
\begin{eqnarray}
c_{j,i}=-\gamma_j^2W(Q_jQ_i) \nonumber
\end{eqnarray}
В итоге
\begin{equation}
\label{DerQj}
\frac{\partial Q_j} {\partial t}=-\sum\limits_{i=0}^{j-1}{\gamma_i^2 W(Q_jQ_i)Q_i}
\end{equation}
По аналогии с (~\ref{pDerQ}) получаем:
\begin{eqnarray}
\frac{\partial \gamma_j^{-2}}{\partial t}=\frac{\partial L(Q_j^2)} {\partial t}=
2L\left(
Q_j\frac{\partial Q_j} {\partial t}\right)+W(Q_j^2)=W(Q_j^2)
\end{eqnarray}
Откуда следует формула для частной производной $\beta$.\\
Из леммы 1 коэффициент при $x^{n-1}$ многочлена $Q_n$
равен $-\sum\limits_{i=0}^{j-1}{\alpha_i}$, по аналогии
коэффициент при $x^{n-1}$ многочлена $\frac{\partial Q_n}{\partial t}$ из
(~\ref{DerQj}):
\begin{eqnarray}
-\frac{\partial}{\partial t} \sum\limits_{i=0}^{j-1}{\alpha_i}=-\gamma_{j-1}^2 W(Q_jQ_{j-1})
\end{eqnarray}
Откуда
\begin{eqnarray}
\frac{\partial \alpha_j} {\partial t}=
\frac{\partial }{\partial t} \sum\limits_{i=0}^{j}{\alpha_i}-
\frac{\partial }{\partial t} \sum\limits_{i=0}^{j-1}{\alpha_i}=
\gamma_j^2W(Q_jQ_{j+1})-\gamma_{j-1}^2W(Q_{j-1}Q_j) \nonumber
\end{eqnarray}
\bf Лемма 4 \rm \\
\it Пусть $\mu$ - некоторая мера, не зависящая от $\lambda$ и $\tau$,
$\tilde{\mu}=\mu+\lambda\delta_{\tau}$ - новая мера с добавленной точкой массы, тогда
 \rm
\begin{equation}
\frac{\partial \tilde{\alpha}_j}{\partial \lambda}=\tilde{\gamma}_j^2\tilde{Q}_j(\tau) \tilde{Q}_{j+1}(\tau)-\tilde{\gamma}_{j-1}^2\tilde{Q}_{j-1}(\tau) \tilde{Q}_j(\tau) \nonumber \\
\end{equation}
\begin{equation}
\frac{\partial \tilde{\beta}_j}
{\partial \lambda}
=
\tilde{\beta}_j
\left[
\tilde{\gamma}_j^2 \tilde{Q}_j^2(\tau)-
\tilde{\gamma}_{j-1}^2 \tilde{Q}_{j-1}^2(\tau)
\right]
\end{equation}
\begin{equation}
\frac{\partial \tilde{\alpha}_j}{\partial \tau}=
\lambda \tilde{\gamma}_j^2
\left[
\tilde{Q}^{'}_j(\tau)\tilde{Q}_{j+1}(\tau)+
\tilde{Q}_j(\tau)\tilde{Q}^{'}_{j+1}(\tau) \right]
-\lambda \tilde{\gamma}_{j-1}^2
\left[ \tilde{Q}^{'}_{j-1}(\tau)\tilde{Q}_{j}(\tau)+\tilde{Q}_{j-1}(\tau)\tilde{Q}^{'}_{j}(\tau) \right] \nonumber \\
\end{equation}
\begin{equation}
\frac{\partial \tilde{\beta}_j}{\partial \tau}=2\lambda \tilde{\beta}_j
\left[
\tilde{\gamma}_j^2\tilde{Q}^{'}_j(\tau)\tilde{Q}_j(\tau)-
\tilde{\gamma}_{j-1}^2\tilde{Q}^{'}_{j-1}(\tau)\tilde{Q}_{j-1}(\tau)
\right]
\end{equation}


\begin{thebibliography}{99}

\bibitem{AptekaaKaliaJvaniseg} A.Aptekarev V.Kaliaguine J.Van Iseghem
The genetic sum's representation for the moments of a system of
Stieltjes functions and its application, Constructive
Approximation, v.16 (2000), pp.487-524.

\bibitem{AptekarevKaliaguine} A.A. Aptekarev, V.A. Kaliaguine,
Complex rational approximation and difference operators \it
Rendiconti del circolo matematico di palermo \rm , serie II,
suppl. 52(1998), pp. 3-21

\bibitem{KaliaguineAA} V.A. Kaliaguine, Hermite-Pade approximants and spectral analysis of noonsymmetric operators
\it Russian Acad. Sci. Sb. Math, \rm  vol. 82(1995), No. 1

\bibitem{KaliaguineAA1} V.A. Kaliaguine, On operators associated with Angelesco systems,
\it East journal on approximations, \rm vol. 1(1995), No. 2

\bibitem{Kaliaguine} V.A. Kaliaguine, The operator moment problem,
vector continued fractions and explicit form ot the Favard theorem
for vector orthogonal polynomials \it J. Comp. Appl. Math. \rm
65(1995) 181-193

\bibitem{KaliaguineRonveaux} V. Kaliaguine, A Ronveaux , On a system of "classical" polynomials of simultaneous orthogonality \it J. Comp. Appl. Math. \rm
67(1996) 207-217

\bibitem{Clenshaw} Clenshaw, C.W., A note on summation of
Chebyshev series, \it Math. Tables Aids Comput. \rm 9, pp 118-120

\bibitem{FischerHJ} H.-J. Fischer, On generating orthogonal polynomials for discrete measures

\bibitem{FischerHJ} H.-J. Fischer, On the condition of orthogonal polynomials via modifie moments,
\it Journal of Analysis and its Applications, \rm vol. 15(1996),
No. 1,1-18

\bibitem{BeckermannBourreau} B. Beckermann, E. Bourreau How to choose modified moments,
\it AMS(MOS): 65D20, 33C45 \rm

\bibitem{GautschiW} Walter Gautschi, Some apllications and numerical methods for orthogonal polynomials,
\it Numerical analysis and mathematical modelling banach center
publications, \rm vol. 24(1990)
\bibitem{GautschiW2} Walter Gautschi, Construction of Gauss-Christoffel quadrature formulas,
\bibitem{GautschiW3} Walter Gautschi, Orthogonal polynomials - constructive theory and applications,
\it Journal of Computational and Applied Mathematics, \rm
vol.12-13(1985) 61-76, North Holland
\bibitem{GautschiW4} Walter Gautschi, Computational aspects of orthogonal polynomials,
\it Orthogonal Polynomials, \rm  1990, pp. 181-216
\bibitem{GautschiW5} Walter Gautschi, On generating orthogonal polynomials,
\it SIAM J. Sci. Stat. Comput., \rm vol. 3, No 3, September (1982)


\bibitem{GoncharAA} А.А. Гончар, О сходимости аппроксимаций Паде
для некоторых классов мероморфных функций \it математический
сборник \rm Т. 97(139), ¦ 4(8), 1975





\bibitem{Nevai} Paul Nevai, Orthogonal polynomials, Mem Amer. Math Soc. 18,No 213



\bibitem{Nikishin} Е.М. Никишин, В.Н. Сорокин, Рациональные аппроксимации и
ортогональность - М.:Наука, 1988

\bibitem{Yurko} V.A. Yurko, On higher-order difference operators \it J. of Diff. Equations
and Appl. \rm 1(1995) 347-352

\bibitem{Henrichi} P.Henrichi Appied and Computational complex Analysis,
John Wiley, 1977, v.2.
\bibitem{jvi} J. Van Iseghem: Vector orthogonal relations. Vector Q.D
 algorithm. J.of Comp. Appl. Math. 19(1987),141-150.

\bibitem{S-VI}V.N.Sorokin, J.Van Iseghem: Algebraic aspects of matrix
orthogonality for vectors polynomials. J.of Appox. Theory
90(1997),97-116.






\bibitem{CabayLabahn} Stan Cabay, George Labahn, A super fast algorithm for multi-dimensional Pade systems,
\it Numerical Algorithms, \rm vol. 2(1992), 201-224
\bibitem{ManticaG} Giorgio Mantica, On computing Jacobi matrices assoiated with recurrent and Mobius iterated function systems,

\bibitem{Izeghem3} J.Van Izeghem, Vector Pade Approximants, proceeding of 11th IMACS Congress 1985 (North Holland, Amsterdam)
\bibitem{Izeghem1} J.Van Izeghem, Vector orthogonal relations. Vector QD-algorithm
\it J. Comp. Appl. Math. \rm 19(1987) 141-150
\bibitem{Izeghem2} J.Van Izeghem, Convergence of the vector QD-algorithm.
Zeroes of vector orthogonal polynomials \it J. Comp. Appl. Math.
\rm 25(1989) 33-46



\bibitem{Backer} Дж. Бейкер, П. Грейвс-Моррис, Аппроксимации Паде - М.:Мир, 1986.


\end{thebibliography}

\end{document}
