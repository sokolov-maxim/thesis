\chapter {Векторные ортогональные многочлены  }
\section{Определение, общие свойства}
%===================================================================
%===================================================================
%\begin{scriptsize}
%\it Борелевской $\sigma$-алгеброй (на прямой \bf R \rm ) называют
%наименьшую $\sigma$-алгебру, содержащую все компактные
%подмножества \bf R  \it Конечная позитивная борелевская мера\rm -
%$\sigma$-аддитивная функция множества, определенная
% на борелевской $\sigma$-алгебре и принимающая конечные неотрицательные значения
%Точку $\lambda_0\in \bf R \rm $ - \it точка роста меры \rm если
%для любого $\varepsilon>0$ имеем
%$\mu(\lambda_0-\varepsilon,\lambda_0+\varepsilon)>0$ Множество
%точек роста меры замкнуто и называется \it спектром меры.
%Носитель меры $\Delta_i$ - наименьший промежуток, содержащий
%спектр\rm \\
%\end{scriptsize}
%================================================================
%================================================================

Рассмотрим набор позитивных борелевских мер
$\mu_1,\mu_2,\ldots,\mu_p$ с бесконечно большим количеством точек
роста на соответствующих носителях
$\Delta_1,\Delta_2,\cdots,\Delta_p$. \\
Введем вектор индексов $\overrightarrow{n}=(n_1,\ldots,n_p)$ \\
%=========================================================
Пусть
\begin{equation}
\label{Moments} s_n^{(j)}=\int \limits_{\Delta_j} {z^n d\mu_j
(z)}, n \in \textbf{Z}_{+}, j=1,2,\ldots,p
\end{equation}
\textit{степенные моменты, построенные по данным позитивным
мерам} \\
%============================================================
Определим $p$ линейных функционалов $L_1,L_2,\ldots,L_p$ в
комплексном линейном пространстве многочленов $\textbf{C}[z]$,
соответствующих последовательностям степенных моментов ${s^{(1)}},
\ldots, {s^{(p)}}$:
\begin{equation}
\label{Functionals}
L_j(z^n)=\int_{\Delta_j}{z^n(x)d\mu_j(x)}=s_n^{(j)},\mbox{    }
j=1,2,\ldots,p
\end{equation}
%=======================================================
\begin{prope}
Важным свойством позитивных мер является позитивность моментов
$s_n^{(j)}$
\end{prope}
%=========================================================
\begin{prope}
Функционалы $L_1,L_2,\ldots,L_p$, отвечающие $s_n^{(j)}$
позитивны, т.е. для любого многочлена $Q(z) \in \textbf{C}[z],
Q(z) \geq 0, z \in [-\infty,+\infty]$ выполняется неравенство
$L_j(Q(z)) \geq 0$, причем $L_j(Q(z))=0$, только если $Q = 0$. \rm
\end{prope}
%============================================================
\begin{defi} Совместно ортогональными многочленами \rm в данном
случае называются многочлены $Q_n$, степени не выше
$|\overrightarrow{n}|=n_1+n_2+\cdots+n_p$, которые удовлетворяют
следующим условиям ортогональности:
\begin{equation}
\label{OrthogonalCondition}
\int_{\Delta_j}{Q_n(x)x^kd\mu_j(x)}=0,\mbox{
}k=0,1,\ldots,n_j-1,j=1,2,\ldots,p
\end{equation}
\end{defi}
%=================================================================
\begin{defi} Векторными ортогональными многочленами \rm называются
совместно ортогональные многочлены, для которых вектор индексов
выбирается в общем случаем как
$\overrightarrow{n}=(\underbrace{k+1,\ldots,k+1}_{d},\underbrace{k,\ldots,k}_{p-d}),
k\in{\mbox{Z}}_{+},n=pk+d$ \\
В общем случае векторные ортогональные многочлены многочлены $Q_n$
определяются не единственным образом. Для определенности будем
считать, что индекс $n$ - \it нормален \rm ($\deg  Q_n=n$) и
$Q_n=z^n+\ldots$. Приведем некоторые важные свойства.
\end{defi}
%===================================================================
\begin{prope} Если носители мер попарно не пересекаются
$\Delta_j\cap\Delta_i=0, j,i=1,2,\ldots,p$, то выполняется:
\begin{equation}
\label{Ortogonality} \int_{\Delta_j}{Q_{\overrightarrow{n}}(x)}
x^{n_j}d\mu_j \not=0, \hspace{1cm}j=1,2,\ldots,p
\end{equation}
\end{prope}
%=====================================================================
\begin{prope}
Векторные ортогональные многочлены $Q_n$ удовлетворяют
рекуррентному соотношению:
\begin{equation}
\label{QRecurrrence}
Q_{n+1}=(z+b_{n,n})Q_n+b_{n,n-1}Q_{n-1}+\ldots+b_{n,n-p}Q_{n-p},
\hspace{1cm} n=pk+d
\end{equation}
при условии $Q_{-p}=\ldots=Q_{-1}=0,Q_0=1, b_{n,n-p}\not=0$ \\
\end{prope}
\bf Доказательство: \rm \\
Соотношение легко проверить. \\ Рассмотрим общий случай для
индекса $n=pk+d$, соответствующий $Q_n$ вектор индексов имеет
следующий вид:
$\overrightarrow{n}=(\underbrace{k+1,\ldots,k+1}_{d},\underbrace{k,\ldots,k}_{p-d})$.
\\ Разложим многочлен $xQ_n$ по базису $Q_0,Q_1,\ldots,Q_{n+1}$
\begin{equation}
\label{xQ}
xQ_n=\sum\limits_{i=0}^{n+1}{\alpha_{n,i}Q_i},\mbox{
}\alpha_{n,n+1}=1
\end{equation}
Рассмотрим случай $k=0$. Применим функционал $L_1$ к разложению
(~\ref{xQ}). Из условия ортогональности левая часть выражения
станет равной нулю, в правой части останется только одно
слагаемое $\alpha_{n,0}L_1(Q_0)$.
$L_1(Q_0)\not=0$, следовательно $\alpha_{n,0}=0$. \\
Далее применим к (~\ref{xQ}) функционал $L_2$. Левая часть по
прежнему останется равной нулю, в правой части останется два
слагаемых $\alpha_{n,1}L_2(Q_1)+\alpha_{n,0}L_2(Q_0)$.
Учитывая $\alpha_{n,0}=0$ и $L_2(Q_1)\not=0$ получаем $\alpha_{n,1}=0$. \\
Последовательно применяя функционалы $L_1,L_2,\ldots,L_p$ получим
$\alpha_{n,0}=\alpha_{n,1}=\ldots=\alpha_{n,p-1}=0$ Ha некотором
шаге $j=0,1,\ldots,k-2$ домножая выражение (~\ref{xQ}) с обоих
сторон на $x^j$ и затем последовательно применяя функционалы
$L_1,L_2,\ldots,L_p$ в результате получим
$\alpha_{n,0}=\ldots\alpha_{n,p-1}=\ldots=\alpha_{n,jp}=\ldots=\alpha_{n,(k-1)p-1}=0$ \\
На последующих шагах процедуры $j=k-1,k,\ldots$ левая часть
выражения (~\ref{xQ}) уже не принимает нулевое значение. Отбросив
нулевые коэффициенты разложения можно записать:
$$
xQ_n=\sum\limits_{i=n-p}^{n+1}{\alpha_{n,i}Q_i}
$$
т.е. $b_{n,i}=-\alpha_{n,i},i=n-p,n-p+1,\ldots,n$ \\ Обратное
утверждение носит название \it теоремы Фавара \rm : если для
любого $n$ многочлены удовлетворяют рекуррентному соотношению вида
(~\ref{QRecurrrence}), то существует некоторый набор функционалов
$L_1,L_2,\ldots,L_p$, по отношению к которому многочлены векторно
ортогональны.\\ \\
