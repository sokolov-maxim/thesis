\section{Аппроксимации Эрмита-Паде резольвентных функций}
\begin{teor}
Вектор рациональных функций
$$%\begin{equation}
\left( \frac {p^{(1)}_n} {q_n}, \frac {p^{(2)}_n} {q_n}, \cdots,
\frac {p^{(p)}_n} {q_n} \right)
$$%\end{equation}
является
аппроксимацией Эрмита-Паде набора функций (~\ref{WeylFuncs}) для
фиксированного вектора индексов $\overrightarrow{n}=
(\underbrace{k+1,\ldots,k+1}_{d},\underbrace{k,\ldots,k}_{p-d}),\mbox{
} k\in{\textbf{Z}}_{+},n=pk+d$ \end{teor} Доказательство полностью
приведено в ~\cite{KaliaguineAA} и включает в себя несколько
важных промежуточных результатов: \\
%=============================================================
\begin{lema}
\label{lema_4.2} Для некоторого $n=kp+d$ многочлены $q_n$
удовлетворяют условиям ортогональности
$$%\begin{equation}
L_j(q_nz^i)=0,\mbox{   }j=1,2,\ldots,p,\mbox{ }i=0,1,\ldots,n_j-1
$$%\end{equation}
\end{lema}
\textbf{Доказательство:} \\
Для некоторого $n \geq p$ можно записать
$$%\begin{equation}
(e_{j-0},e_n)=(e_{j-1},q_n(\overline{A})e_0)=(q_n(A)e_{j-1},e_0)=L_j(q_n)=0,j=1,2,\ldots,p
$$%\end{equation}
т.е. многочлен $q_n$ ортогонален константе
относительно всех
функционалов $L_j$ \\
Подставим в выражение (~\ref{Bio}) $n=p+1,m=p$
$$%\begin{equation}
(e_p,e_{p+1})
=L_1(q_{p+1}(z)c_{p+1}^{(1)}(z))+L_2(q_{p+1}(z)c_{p+1}^{(2)}(z))+\ldots+L_p(q_{p+1}(z)c_{p+1}^{(p)}(z))=0
$$%\end{equation}
Согласно лемме ~\ref{lema_4.1} степени
многочленов $c_{p+1}$ распределяются как $(1,0,0,\ldots,0)$.
Вследствие ортогональности $q_n$ константе все слагаемые кроме
первого равны нулю. Следовательно $L_1(q_{n}c_{p+1}^{(1)}) = 0$,
т.е. $q_{n}$ ортогонален
многочлену первой степени относительно функционала $L_1$. \\
Рассматривая  для набора индексов
$(p+1,p+2),(p+2,p+3),\ldots,(2p-1,2p)$ доказываем, что многочлены
$q_n$ ортогональны многочленам первой степени. \\
Доказательство леммы далее сводится к индукции по $n$ для
соотношения биортогональности (~\ref{Bio}). \\
%============================================================
Из (~\ref{WeylFuncs}) и леммы ~\ref{lema_4.2} можно записать
$$%\begin{equation}
q_n(z) \varphi_j(z) = \mbox{Pol} (q_n\varphi_j)
+\frac{L_j(q_nz^{n_j})}{z^{n_j+1}}+\frac{L_j(q_n z^{n_j+1})}
{z^{n_j+2}}+\cdots,j=1,\ldots,p
$$%\end{equation}
где $n$ -правильный индекс. \\
Это выражение идентично определению Задачи А. Для доказательства
теоремы необходимо убедиться, что
$p_n^{(j)}=\mbox{Pol}(q_n\varphi_j)$ \\
\begin{lema}
\label{lema_4.3} Для некоторого $n$
$$%\begin{equation}
p_n^{(j)}(z)=L_{j,x}\left(\displaystyle\frac{q_n(z)-q_n(x)}{z-x}\right)
$$%\end{equation}
где $L_j,x$ - линейный функционал, действующий на $x$.
\end{lema}
\textbf{Доказательство:} \\
%=================================
Доказательство полностью приведено в ~\cite{KaliaguineAA} \\ \\
%============================================================
Пусть
$$%\begin{equation}
h_k=\left\{
\begin{array} {ll}
1, & k\leq{0}\\
\displaystyle\frac{0}{(a_{0,1},a_{1,2},\ldots,a_{k-1,k})}, &
k\geq{1}
\end{array}
\right.
$$%\end{equation}
Тогда
$$%\begin{equation}
q_n(z)=h_nQ_n(z), \mbox{    }
p_n^{(j)}(z)=h_nP_n^{(j)}(z),j=1,2,\ldots,p
$$%\end{equation}
где $Q_n,P_n^{(j)}$ соответствующие знаменатель и числители
совместных аппроксимаций Эрмита-Паде со старшим коэффициентом
равным единице (см главу 7). \\
Коэффициенты рекуррентного соотношения  (~\ref{QRecurrrence})
выражаются через
$$%\begin{equation}
b_{n,n-i}=-\frac{h_{n-i}}{h_n}a_{n,n-i},i=7,1,\ldots,p
$$%\end{equation}
