\subsubsection{Векторная непрерывная дробь Стилтьеса}
При изучении некоторых классов несимметричных разностных
операторов, определяемых так называемой \it "разреженной" \rm
 матрицей вида:
\begin{equation}
\label{Operator_Matrix_Ex}
A= \left(\begin{array}{ccccccc}
0 & 1&0&0&0&0&\cdots\\
0 & 0 &1&0&0&0&\cdots\\
\cdots&\cdots&\cdots&\cdots&\cdots&\cdots&\cdots\\
a_0&0&0&\cdots&1&0&\cdots\\
0&a_1&0&\cdots&0&1&\cdots\\
\cdots&\cdots&\cdots&\cdots&\cdots&\cdots&\cdots
\end{array}\right)
\end{equation}
было введено понятие \it векторной непрерывной дроби Стилтьеса
\rm, которая является очевидным аналогом классической непрерывной
дроби Стилтьеса и имеет вид:
$$
S(z)=\frac{(1,\ldots,1)}{(0,\ldots,0,z)+}
\frac{(1,\ldots,1,-a_1)}{(0,\ldots,0,1)+}\cdots
\frac{(1,\ldots,1,-a_p)}{(0,\ldots,0,1)+}
\frac{(1,\ldots,1,-a_{p+1})}{(0,\ldots,0,z)+}\cdots
$$
где $S(z)=(S_1(z), S_2(z), \ldots, S_p(z))$ - вектор формальных
степенных рядов. \\
Cтепенной ряд $S_j(z)$ получается из разложения соответствующей
функции Вейля $\varphi_j(z)$ в ряд Лорана в окрестности
бесконечности путем удаления блоков нулевых элементов длиной $p$
и смещения ряда.
\begin{eqnarray}
S_1(z^{p+1}) = \frac{1}{z^p}\varphi_1(z) \nonumber \\
\cdots \nonumber \\
S_p(z^{p+1}) = \frac{1}{z}\varphi_p(z) \nonumber \\
\end{eqnarray}
Коэффициенты степенного ряда $S_j(z)$ выражаются через
коэффициенты разложения соответствующей функции Вейля следующим
образом:
$$
S_j(z)=\sum_{k=0}^{\infty}{\frac{S_k^{(j)}}{z^{k+1}}},
S_k^{(j)}=s_{pk+j-1}^{(j)}
$$
Старая версия QD алгоритма в случае "разреженной" матрицы не
позволяла решить обратную проблему моментов ввиду наличия нулевых
элементов разложений функций Вейля.\\
Новая версия позволяет решить обратную проблему моментов для
ленточных операторов, определяемых матрицей вида
(~\ref{Operator_Matrix_Ex}) при условии, что описанный выше алгоритм применяется к новому набору моментов $S_k^{(j)}$.\\
В ~\cite{AptekaaKaliaJvaniseg} приводится доказательство
следующего отношения между многочленами
\begin{eqnarray}
Q_1^{(0)}(z)=zQ_0^{(p)}-a_1Q_0^{(0)} \nonumber \\
Q_1^{(1)}(z)=Q_1^{(0)}-a_2Q_0^{(1)} \nonumber\\
\cdots \nonumber \\
Q_1^{(p)}(z)=Q_1^{(p-1)}-a_{p+1}Q_0^{(p-1)} \nonumber\\
Q_2^{(0)}(z)=zQ_1^{(p)}-a_{p+2}Q_0^{(p)} \nonumber
\end{eqnarray}
где $Q_n^{\nu} = Q_n^{k,d}, \nu=pk+d$ \\
Сравнивая это отношение с (\ref{QDExAlpha}) и (\ref{QDExBeta})
легко установить зависимость между элементами векторной QD
таблицы и элементами исходной "разреженной" матрицы. \\
Расположим
элементы векторной QD таблицы  в следующем порядке
\begin{equation}
\begin{array}{ccccccccccccccccc}
\beta_1^0 & \alpha_1^0 & \cdots & \alpha_1^{p-1} & \beta_2^0 &
\alpha_2^0 & \cdots \alpha_2^{p-1} & \beta_3^0 & \cdots \\
\beta_1^1 & \alpha_1^1 & \cdots & \alpha_1^{p} & \beta_2^1 &
\alpha_2^1 & \cdots \alpha_2^{p} & \beta_3^1 & \cdots \\
\beta_1^2 & \alpha_1^2 & \cdots & \alpha_1^{p+1} & \beta_2^2 &
\alpha_2^2 & \cdots \alpha_2^{p+1} & \beta_3^2 & \cdots \\
\cdots & \cdots & \cdots & \cdots & \cdots & \cdots & \cdots &
\cdots & \cdots & \\
\end{array}
\end{equation}
Данный вид QD таблицы содержит содержит в первой строке
коэффициенты разложения векторной непрерывной дроби Стилтъеса и
соответственно элементы нижней диагонали исходной "разреженной"
матрице
\begin{equation}
\displaystyle {(1,\cdots,1)\over (0,\cdots 0,z)+}\
{(1,\cdots,1,-\beta_1^0)\over (0,\cdots 0,1)+\cdots}\ \cdots
{(1,\cdots,1,-\alpha_1^{p-1})\over (0,\cdots 0,1)+}\
{(1,\cdots,1,-\beta_2^0)\over (0,\cdots 0,z)+\cdots }\
\end{equation}
Каждая последующая строчка содержит соответственно коэффициенты
разложения в Стилтьеса для "сдвинутых" моментов. \\
Пусть
$\{a_i^{(\nu)}\}^{\nu=0,1,\ldots}_{i=1,2,\ldots} $ наборы
коэффициентов разложений, где $\nu$ показатель сдвига
соответствующих моментов ${s_n}$. \\
Тогда общее соответствие для $i=(p+1)k+d$ может быть записано в
виде :
\begin{eqnarray}
a_{i}^{\nu}=\left\{
\begin{array}{llllllll} \beta_{k+1}^{\nu},& d=0 \\
\alpha_{k+1}^{\nu+d-1},& d>0
\end{array}
\right.
\end{eqnarray}
