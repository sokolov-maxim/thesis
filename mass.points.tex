\section{Добавление масс. Вычисление новых рекуррентных коэффициентов}

При рассмотрении отображения $H_n$: как изменяются коэффициенты
рекуррентного соотношения при добавлении массы ? \\
Стартуем с результата Неваи ~\cite{Nevai}: \\
\bf Лемма Неваи \rm \\
\it Пусть $q(x)$ - ортонормированные многочлены относительно некоторой меры
$\mu(x)$. \\
Пусть $\tilde{\mu}(x)=\mu(x)+\lambda \delta_{\tau}$ - новая мера, получающаяся из исходной
добавлением массы $\lambda$ в точке $\tau$.
Соответствуищие ортогональные многочлены
$\tilde{q(x)}$, коэффициенты рекуррентного соотношения для которых - $\tilde{\alpha}$ и $\tilde{\beta}$.
тогда имеет место следующее соотношение: \rm
\begin{eqnarray}
\label{Alpha}
\tilde{\alpha}_j=\alpha_j+
\lambda\frac{\sqrt{\beta_{j+1}} q_j(\tau)q_{j+1}(\tau)} { 1+ \lambda\sum\limits_{i=0}^{j}{q_i^2(\tau)}}-
\lambda\frac{\sqrt{\beta_{j}} q_j(\tau)q_{j-1}(\tau)} { 1+ \lambda\sum\limits_{i=0}^{j-1}{q_i^2(\tau)}} \\
\tilde{\beta}_j=\beta_j
\frac
{ \left[
1+ \lambda\sum\limits_{i=0}^{j-2}{q_i^2(\tau)}
\right]
\left[
1+ \lambda\sum\limits_{i=0}^{j}{q_i^2(\tau)}
\right] }
{
\left[
1+ \lambda\sum\limits_{i=0}^{j-1}{q_i^2(\tau)}
\right]
},j<N
\end{eqnarray}
При вычислении проще воспользоваться многочленами $Q_n$.  \\
Однако, если $\tau$ отлично от всех $(\tau_j)_0^{N}$ и все
$\lambda_j$ и $\lambda$ положительны, т.е.
мы имеем дополнительную точку массы, тогда
существует $\tilde{\alpha_{N}}$, которую невозможно вычислить
используя лемму Неваи (так как коэффициент $\alpha_N$ не определен). \\
Приведем две леммы из ~\cite{Fischer2}: \\
\bf Лемма 1. \rm \\
\it Коэффициент многочлена $Q_n(x)$ при $x^{n-1}$ равняется $-\sum\limits_{j=0}^{n-1}{\alpha_j}$ \rm. \\
\bf Лемма 2. \rm \\
\it Пусть в дискретном представлении меры $\mu(x)$ (~\ref{Discrete})
все веса $\lambda_j>0,j=1,\ldots,N$ и все узлы $\tau_j,j=1,\ldots,N$ разные,
тогда $\sum\limits_{j=0}^{N-1}{\alpha_j}=\sum\limits_{j=1}^{N}{\tau_j}$ \rm \\
теперь приведем основной результат ~\cite{Fischer2}: \\
\bf Теорема 1. \rm \\
\it Пусть в дискретном представлении меры $\mu(x)$ (~\ref{Discrete})
все веса $\lambda_j>0,j=1,\ldots,N$ и все узлы $\tau_j,j=1,\ldots,N$ разные,
добавим дополнительную точку $(\lambda,\tau),\lambda>0,\tau \not = \tau_j,j=1,\ldots,N$,
тогда:
\begin{eqnarray}
\label{FischerMass}
\tilde{\alpha}_j=\alpha_j+\lambda
\frac {\gamma_j^2 Q_j(\tau) Q_{j+1}(\tau)}
{1+\lambda\sum\limits_{i=0}^{j}{\gamma_i^2 Q_i^2(\tau)}}-
\lambda\frac {\gamma_{j-1}^2 Q_j(\tau) Q_{j-1}(\tau)}
{1+\lambda\sum\limits_{i=0}^{j-1}{\gamma_i^2 Q_i^2(\tau)}} \nonumber \\
\tilde{\beta}_j=\beta_j
\frac {
\left[
1+ \lambda\sum\limits_{i=0}^{j-2}{\gamma_i^2 Q_i^2(\tau)}
\right]
\left[
1+ \lambda\sum\limits_{i=0}^{j}{\gamma_i^2 Q_i^2(\tau)}
\right]
}
{
\left[
1+ \lambda\sum\limits_{i=0}^{j-1}{\gamma_i^2 Q_i^2(\tau)}
\right]
}, j<N \\
\tilde{\alpha}_N=\tau-\lambda
\frac {\gamma_{N-1}^2 Q_N(\tau) Q_{N-1}(\tau)}
{1+\lambda\sum\limits_{i=0}^{N-1}{\gamma_i^2 Q_i^2(\tau)}},
\tilde{\beta}_N=\lambda
\frac {
\gamma_{N-1}^2Q_N^2(\tau)
\left[
1+ \lambda\sum\limits_{i=0}^{N-2}{\gamma_i^2 Q_i^2(\tau)}
\right]
}
{
\left[
1+ \lambda\sum\limits_{i=0}^{N-1}{\gamma_i^2 Q_i^2(\tau)}
\right]
} \nonumber
\end{eqnarray}
\bf Доказательство: \rm  \\
Пусть $\tilde{Q}_n(x)$ - ортогональные многочлены (со старшим коэффицентом единица)
для новой меры $\tilde{\mu}(x)=\mu+\lambda\delta_{\tau}$:
\begin{eqnarray}
\label{QMass}
\tilde{Q}_j(x)=Q_j(x)+\sum\limits_{l=0}^{j-1}{c_{j,l}Q_l(x)} \nonumber
\end{eqnarray}
$c_{jl}$ - некоторые неизвестные коэффициенты
разложения по базису $(Q_n)_0^{\infty}$.\\
Домножим $\tilde{Q}_j(x)$ на $Q_l(x),l<j$ и проинтегрируем,
применив функционал $\tilde{L}(f)=\int f d\tilde{\mu}(x)$,
правая часть выражение станет равной нулю:
\begin{eqnarray}
\tilde{L}(\tilde{Q}_j(x)Q_l(x))=L(\tilde{Q}_j(x)Q_l(x))+\lambda\tilde{Q}_j(\tau)Q_l(\tau)= \nonumber \\
L(\left[Q_j(x)+\sum\limits_{l=0}^{j-1}{c_{j,l}Q_l(x)}\right]Q_l(x))+\lambda\tilde{Q}_j(\tau)Q_l(\tau)= \nonumber \\
L(\left[\sum\limits_{l=0}^{j-1}{c_{j,l}Q_l(x)}\right]Q_l(x))+\lambda\tilde{Q}_j(\tau)Q_l(\tau)=
c_{j,l}\frac{1}{\gamma_l^2}+\lambda\tilde{Q}_j(\tau)Q_l(\tau)=0 \nonumber
\end{eqnarray}
Откуда:
\begin{equation}
\label{KoefC}
c_{j,l}=-\lambda\gamma_{l}^2 \tilde{Q}_j(\tau)Q_l(\tau)
\end{equation}
Вставим (~\ref{KoefC}) в (~\ref{QMass}) $x=\tau$ и выразим $\tilde{Q}_j(\tau)$:
\begin{equation}
\label{Eq15}
\tilde{Q}_j(\tau)=\frac{Q_j(\tau)}
{1+\lambda\sum\limits_{i=0}^{j-1}{\gamma_i^2 Q_i^2(\tau) }}
\end{equation}
Далее:
\begin{equation}
\label{Eq16}
c_{j,l}=-\frac
{\lambda \gamma_l^2 Q_j(\tau) Q_l(\tau)}
{1+\lambda\ \sum\limits_{i=0}^{j-1} {\gamma_i^2 Q_i^2(\tau)}}
\end{equation}
Сравним коэффициенты в (~\ref{QMass}) при $x^{j-1}$ и применим Лемму 1.
\begin{equation}
\label{AlphaPre}
-\sum\limits_{i=0}^{j-1}{\tilde{\alpha}_i}=-\sum\limits_{i=0}^{j-1}{\alpha_i}
+c_{j,j-1}=-\sum\limits_{i=0}^{j-1}{\alpha_j}-\frac
{\lambda \gamma_{j-1}^2 Q_j(\tau) Q_{j-1}(\tau)}
{1+\lambda\ \sum\limits_{i=0}^{j-1} {\gamma_i^2 Q_i^2(\tau)}},j\leq N
\end{equation}
Вспомним Лемму 2:
\begin{eqnarray}
\sum\limits_{i=0}^{N-1}{\alpha_i}=\sum_{i=1}^{N}{\tau_i},  \nonumber \\
\sum\limits_{i=0}^{N-1}{\tilde{\alpha}_i}=\sum_{i=1}^{N}{\tau_i}+\tau=\sum\limits_{i=0}^{N-1}{\alpha_i}+\tau, \nonumber
\end{eqnarray}
Далее учитывая (~\ref{AlphaPre})
\begin{eqnarray}
\tilde{\alpha}_N=\sum\limits_{i=0}^{N-1}{\alpha_i}-\sum\limits_{i=0}^{N-1}{\tilde{\alpha}_i}+\tau \nonumber \\
=\sum\limits_{i=0}^{N-1}{\alpha_i}-\sum\limits_{i=0}^{N-1}{\alpha_i}+c_{N,N-1}+\tau=c_{N,N-1}+\tau \nonumber \\
=\tau-\lambda
\frac {\gamma_{N-1}^2 Q_N(\tau) Q_{N-1}(\tau)}
{1+\lambda\sum\limits_{i=0}^{N-1}{\gamma_i^2 Q_i^2(\tau)}}
\end{eqnarray}
Отсюда, используя (~\ref{AlphaPre}) для индекса $N$ можно выразить $\tilde{\alpha}_N$. \\
Для вывода выражения для $\tilde{\beta}_N$ воспользуемся (~\ref{QMass})
\begin{eqnarray}
\tilde{L}(\tilde{Q}_j^2(x))= L(\tilde{Q}_j^2(x))+\lambda\tilde{Q}_j^2(\tau) \nonumber \\
=L(Q_j^2(x))+\sum\limits_{l=0}^{j-1}{c_{j,l}^2 L(Q_j(x)Q_l(x))}+\sum\limits_{l=0}^{j-1}{c_{j,l}^2L(Q_l^2(x))}+\lambda\tilde{Q}_j^2(\tau) \nonumber \\
=L(Q_j^2(x))+\sum\limits_{l=0}^{j-1}{c_{j,l}^2L(Q_l^2(x))}+\lambda\tilde{Q}_j^2(\tau) \nonumber \\
=\frac{1}{\gamma_j^2}+\sum\limits_{l=0}^{j-1}
{\left[
-\frac {\lambda \gamma_l^2 Q_j(\tau)Q_l(\tau)}
{1+\lambda \sum\limits_{i=0}^{j-1}{\gamma_i^2 Q_i^2(\tau)}}
\right]^2\frac{1}{\gamma_l^2}}+
\lambda
\left[
\frac{Q_j(\tau)}
{1+\lambda \sum\limits_{i=0}^{j-1}{\gamma_i^2 Q_i^2(\tau)}}
\right]^2 \nonumber
\end{eqnarray}
Общий знаменатель уже есть, выносим за скобки $\lambda Q_j^2(\tau)$
\begin{eqnarray}
\tilde{L}(\tilde{Q}_j^2)=\frac{1}{\gamma_j^2}+
\frac{\lambda Q_j^2(\tau) \left[
\sum\limits_{l=0}^{j-1}{\gamma_l^2Q_l^2(\tau)+1}
\right]}
{\left[
1+\sum\limits_{l=0}^{j-1}{\gamma_l^2Q_l^2(\tau)}
\right]^2}=
\frac{1}{\gamma_j^2}+
\frac{\lambda Q_j^2(\tau)}
{1+\sum\limits_{l=0}^{j-1}{\gamma_l^2Q_l^2(\tau)}}
\end{eqnarray}
Далее учитывая $L(Q_j^2(x))=\gamma_j^{-2},j<N\mbox{     }L(Q_N^2(x))=0$, (~\ref{Eq15}) и (~\ref{Eq16})
\begin{eqnarray}
\frac{1}{\tilde{\gamma}^2}=\tilde{L}(\tilde{Q}_j^2(x))=\frac{1}{\gamma_j^2}+
\frac{\lambda Q_j^2(\tau)} {1+\lambda\sum\limits_{i=0}^{j-1}{\gamma_i^2 Q_i^2(\tau)}}
=\frac{1}{\gamma_j^2}\frac{1+\lambda \sum\limits_{i=0}^{j}{\gamma_i^2 Q_i^2(\tau)}}
{1+\lambda \sum\limits_{i=0}^{j-1}{\gamma_i^2 Q_i^2(\tau)}},j<N \nonumber
\end{eqnarray}
\begin{equation}
\tilde{L}(\tilde{Q}_N^2(x))=\frac{\lambda Q_N^2(\tau)}
{1+\lambda \sum\limits_{i=0}^{N-1}{\gamma_i^2 Q_i^2(\tau)}}
\end{equation}
Отсюда
\begin{eqnarray}
\tilde{\beta}_N=
\frac{\tilde{L}(\tilde{Q}_{N}^2)}
{\tilde{L}(\tilde{Q}_{N-1}^2)}=
\frac {\lambda Q_N^2(\tau)}
{1+\lambda \sum\limits_{i=0}^{N-1}{\gamma_i^2 Q_i^2(\tau)}}
\gamma_{N-1}^2
\frac {1+\lambda \sum\limits_{i=0}^{N-2}{\gamma_i^2 Q_i^2(\tau)}}
{1+\lambda \sum\limits_{i=0}^{N-1}{\gamma_i^2 Q_i^2(\tau)}} \nonumber \\
=\frac
{\lambda \gamma_{N-1}^2 Q_{N}^2( \tau)\left[1+\sum\limits_{i=0}^{N-2}{\gamma_i^2 Q_i^2(\tau)}\right]}
{\left[1+\lambda \sum\limits_{i=0}^{N-1}{\gamma_i^2 Q_i^2(\tau)}\right]^2} \nonumber
\end{eqnarray}
Используя (~\ref{FischerMass}) можно последовательно добавлять
несколько точек масс.

\subsection{Чувствительность отображения $H_n, (p=2)$ при добавлении массы}

\bf Лемма 3. \rm \\
\it Введем следуещее обозначние $\frac{\partial L}{\partial t}=W$, где
$t$ некоторый параметр от которого зависит функционал $L$, тогда
\begin{equation}
\frac{\partial \alpha_j} {\partial t}=\gamma_j^2W(Q_jQ_{j+1})-\gamma_{j-1}^2W(Q_{j-1}Q_j)
\end{equation}
\begin{equation}
\frac{\partial \beta_j} {\partial t}=\beta_j(\gamma_j^2W(Q_j^2)-\gamma_{j-1}^2W(Q_{j-1}^2))
\end{equation}
\rm
\bf Доказательство: \rm \\
Из процедуры Стилтьеса:
\begin{eqnarray}
\alpha_j=\frac{L(xQ_j^2(x))}{L(Q_j^2)},\mbox{   }\beta_j=\frac{L(Q_j^2(x))}{L(Q_{j-1}^2)}=\frac{\gamma_{j-1}^2}{\gamma_j^2} \nonumber
\end{eqnarray}
Так старший коэффициентов многочлена $Q_j$ равен единице
уместно следуещее разложение в ряд Фурье
(или по базису $(Q_0,\ldots,Q_{j-1})$):
\begin{eqnarray}
Q_j(x)=\sum\limits_{i=0}^{j-1}{c_{j,i}Q_i(x)} \nonumber \\
c_{j,i}=
L
\left(
\frac {\partial Q_j}  {\partial t}
Q_i
\right)
\frac{1}{L(Q_i^2)}
=
\gamma_i^2
L
\left(
\frac{\partial Q_j} {\partial t}
Q_i
\right)
\nonumber
\end{eqnarray}
С другой стороны
\begin{eqnarray}
\label{pDerQ}
\frac{\partial} {\partial t} L(Q_jQ_i)=
L\left(
\frac{\partial Q_j} {\partial t} Q_i
\right)
+L\left(
Q_j \frac{\partial Q_i} {\partial t}
\right)+
W(Q_jQ_i)=0,i<j
\end{eqnarray}
где $L\left(
Q_j \frac{\partial Q_i} {\partial t}
\right)$ также равно нулю согласно условию ортогональности. \\
В результате:
\begin{eqnarray}
c_{j,i}=-\gamma_j^2W(Q_jQ_i) \nonumber
\end{eqnarray}
В итоге
\begin{equation}
\label{DerQj}
\frac{\partial Q_j} {\partial t}=-\sum\limits_{i=0}^{j-1}{\gamma_i^2 W(Q_jQ_i)Q_i}
\end{equation}
По аналогии с (~\ref{pDerQ}) получаем:
\begin{eqnarray}
\frac{\partial \gamma_j^{-2}}{\partial t}=\frac{\partial L(Q_j^2)} {\partial t}=
2L\left(
Q_j\frac{\partial Q_j} {\partial t}\right)+W(Q_j^2)=W(Q_j^2)
\end{eqnarray}
Откуда следует формула для частной производной $\beta$.\\
Из леммы 1 коэффициент при $x^{n-1}$ многочлена $Q_n$
равен $-\sum\limits_{i=0}^{j-1}{\alpha_i}$, по аналогии
коэффициент при $x^{n-1}$ многочлена $\frac{\partial Q_n}{\partial t}$ из
(~\ref{DerQj}):
\begin{eqnarray}
-\frac{\partial}{\partial t} \sum\limits_{i=0}^{j-1}{\alpha_i}=-\gamma_{j-1}^2 W(Q_jQ_{j-1})
\end{eqnarray}
Откуда
\begin{eqnarray}
\frac{\partial \alpha_j} {\partial t}=
\frac{\partial }{\partial t} \sum\limits_{i=0}^{j}{\alpha_i}-
\frac{\partial }{\partial t} \sum\limits_{i=0}^{j-1}{\alpha_i}=
\gamma_j^2W(Q_jQ_{j+1})-\gamma_{j-1}^2W(Q_{j-1}Q_j) \nonumber
\end{eqnarray}
\bf Лемма 4 \rm \\
\it Пусть $\mu$ - некоторая мера, не зависящая от $\lambda$ и $\tau$,
$\tilde{\mu}=\mu+\lambda\delta_{\tau}$ - новая мера с добавленной точкой массы, тогда
 \rm
\begin{equation}
\frac{\partial \tilde{\alpha}_j}{\partial \lambda}=\tilde{\gamma}_j^2\tilde{Q}_j(\tau) \tilde{Q}_{j+1}(\tau)-\tilde{\gamma}_{j-1}^2\tilde{Q}_{j-1}(\tau) \tilde{Q}_j(\tau) \nonumber \\
\end{equation}
\begin{equation}
\frac{\partial \tilde{\beta}_j}
{\partial \lambda}
=
\tilde{\beta}_j
\left[
\tilde{\gamma}_j^2 \tilde{Q}_j^2(\tau)-
\tilde{\gamma}_{j-1}^2 \tilde{Q}_{j-1}^2(\tau)
\right]
\end{equation}
\begin{equation}
\frac{\partial \tilde{\alpha}_j}{\partial \tau}=
\lambda \tilde{\gamma}_j^2
\left[
\tilde{Q}^{'}_j(\tau)\tilde{Q}_{j+1}(\tau)+
\tilde{Q}_j(\tau)\tilde{Q}^{'}_{j+1}(\tau) \right]
-\lambda \tilde{\gamma}_{j-1}^2
\left[ \tilde{Q}^{'}_{j-1}(\tau)\tilde{Q}_{j}(\tau)+\tilde{Q}_{j-1}(\tau)\tilde{Q}^{'}_{j}(\tau) \right] \nonumber \\
\end{equation}
\begin{equation}
\frac{\partial \tilde{\beta}_j}{\partial \tau}=2\lambda \tilde{\beta}_j
\left[
\tilde{\gamma}_j^2\tilde{Q}^{'}_j(\tau)\tilde{Q}_j(\tau)-
\tilde{\gamma}_{j-1}^2\tilde{Q}^{'}_{j-1}(\tau)\tilde{Q}_{j-1}(\tau)
\right]
\end{equation}
