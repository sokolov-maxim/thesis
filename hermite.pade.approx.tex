\section{Аппроксимации Эрмита-Паде набора марковских функций}
%\subsection{Аппроксимации Эрмита-Паде}

Пусть $\overrightarrow{f}=(f_1,f_2,\ldots,f_p)$ - система
марковских функций:
\begin{equation}
\label{Markov_system} f_j(z)
=\int_{\Delta_j}{\displaystyle\frac{d\mu_j(x))}{z-x}}
=\sum\limits_{k=0}^{\infty} \frac { s_{k}^{(j)} } {z^{k+1}}
=\frac { s_{0}^{(j)} } {z} + \frac { s_{1}^{(j)} } {z^{2}} +\frac
{ s_{2}^{(j)} } {z^{3}}+\cdots
\end{equation}
Пусть
\begin{equation}
\label{H} H_n=\left(
\begin{array}{ccccc}
s_0           & s_{1}         & \cdots & s_{n-1}        \\
s_{1}         & s_{2}         & \cdots & s_{n}          \\
\cdots        & \cdots        & \cdots & \cdots         \\
s_{k-1}       & s_{k}         & \cdots & s_{k+n-2}      \\
s_{k}^{(d)}   & s_{k+1}^{(d)} & \cdots & s_{k+n-1}^{(d)}
\end{array}
\right)
\end{equation}
определитель Ганкеля размерности $n \times n$, построенный по
системе функций $f$, где $n=pk+d, \mbox{   }
s_j=(s_j^{(1)},s_j^{(2)},\ldots,s_j^{(p)})^T, \mbox{   }
s_i^{d}=(s_i^{(1)},s_i^{(2)},\ldots,s_i^{(d)})^{\mbox{T}}$ \\
%=================================================================
%     Задачач А  определение
%==================================================================
\bf Задача A \rm \\ \it Для некоторого фиксированного вектора
индексов $\overrightarrow{n}=(n_1,n_2,\ldots,n_p),n_j\in{ \bf N
\rm }$ требуется найти многочлен $Q_n\not=0,\deg
(Q_n)\leq{|\overrightarrow{n}|}$ такой, что для некоторых
многочленов $P_n^{(1)},P_n^{(2)},\ldots,P_n^{(p)}$ выполнялось
соотношение: \rm
\begin{equation}
\label{Vector_Pade} Q_n(z)f_j(z)-P_n^{(j)}= \frac {s_j^{'}}
{z^{n_{j+1}}} +\ldots,j=1,2,\ldots,p
\end{equation}
Соотношение эквивалентно системе $n$ линейных однородных уравнений
c $n+1$ неизвестными:
\begin{eqnarray}
\label{QH0}
Q_n(z)=\beta_nz^n+\beta_{n-1}z^{n-1}+\ldots+\beta_1z+\beta_0,
\mbox{     } n=pk+d \nonumber
\\ \left(
\begin{array}{cccccccc}
\beta_0 \\ \beta_1 \\ \ldots \\  \beta_{n-1}
\\ \beta_n
\end{array}
\right) \left(
\begin{array}{ccccccccccc}
 &    &         & s_n     \\
 &    &         & s_{n+1} \\
 &    & H_n     & \cdots \\
 &    &         & s_{k+n-1} \\
 &    &         & s_{k+n}^{d}
\end{array}
\right)= \left(
\begin{array}{cccccccc}
0 \\ 0 \\ \ldots \\ 0 \\ 0
\end{array}
\right)
\end{eqnarray}

Такая система всегда имеет ненулевое решение, при
этом рациональная функция:\\
\begin{equation}
\label{HermitePade} \overrightarrow{\pi}_{\overrightarrow{n}}=
\left( \frac {P^{(1)}} {Q}, \frac {P^{(2)}} {Q}, \cdots, \frac
{P^{(p)}} {Q} \right)
\end{equation}
называется \it $n$-ой совместной аппроксимацией Эрмита-Паде \rm
системы $\overrightarrow{f}$ с разложением в бесконечности. \\ В
общем случае вектор $\overrightarrow{\pi}$ определен не
единственным образом, хотя существует довольно большой класс
марковских функций для которых аппроксимация Эрмита-Паде
единственна. Достаточным условием единственности является \it
нормальность индекса \rm $n$. Индекс $n$ называется нормальным
относительно задачи A, если для любого решения $\deg Q=|\overrightarrow{n}|$. \\
%===============================================================
\begin{coly} Из (~\ref{QH0}) легко проверить, что индекс $n$
нормален относительно задачи А $\Leftrightarrow H_n \not= 0, H_0
= 1$ \end{coly}
%===============================================================
Существуют так называемые \it совершенные \rm и \it
слабосовершенные \rm системы функций вида (~\ref{Markov_system}).
Для совершенной системы степень знаменателя аппроксимации $Q_n$ в
точности равна $|\overrightarrow{n}|$ для любого
$\overrightarrow{n}$. Cлабосовершенная система характеризуется
более слабым условием нормальности, которое распространяется
только на правильные индексы. \it Правильные индексы \rm
удовлетворяют следующему условию:
$$
\overrightarrow{n}=(\underbrace{k+1,\ldots,k+1}_{d},\underbrace{k,\ldots,k}_{p-d}),
k\in{\mbox{Z}}_{+},n=pk+d
$$
%=======================================================================
Соотношение (~\ref{QH0}) (в силу позитивности последовательностей
${s^{(j)}}, j=1,\ldots,p$) можно переписать в следующем виде:
\begin{equation}
\label{QOrthogonality} L_j(Q_n(z)z^i) = 0, i =0,1,\cdots, n_j-1,
j=1,2,\cdots,p
\end{equation}
т.е., выполняется условие ортогональности для знаменателей
совместной аппроксимации Паде, которые называют \it ортогональными
многочленами II типа \rm \\ Числители выражаются через
знаменатель:
$$
P^{(j)}(z)=L_{j,x} \left( \displaystyle \frac {Q(z)-Q(x)}{z-x}
\right)
$$
и называются \it многочленами второго рода \rm для многочленов $Q_n$. \\
Рекуррентное соотношение (~\ref{QRecurrrence}) связывает и
знаменатели, и числители аппроксимации Эрмита-Паде $P^{(j)}_n$
для каждого фиксированного $j$. Строго говоря, соотношение
(~\ref{QRecurrrence}) справедливо только для $n\geq{p}$, но его
можно расширить для случая $n\geq{0}$ выбрав соответствующие
начальные условия. Все многочлены с отрицательными индексамит
равны нулю, для первых $p+1$ индексов задаются следующие условия:
$$
%\begin{equation}
%\label{P_ic}
\begin{array} {rcccccccccccccc}
n       & = & 0 & 1 & 2 & 3 & \cdots & p   \\
Q       & = & 1 & 0 & 0 & 0 & \cdots & 0    \\
P^{(1)} & = & 0 & 1 & 0 & 0 & \cdots & 0    \\
P^{(2)} & = & 0 & 0 & 1 & 0 & \cdots & 0    \\
P^{(3)} & = & 0 & 0 & 0 & 1 & \cdots & 0    \\
\cdots  & = & \cdots & \cdots & \cdots & \cdots & \cdots & \cdots   \\
P^{(p)} & = & 0 & 0 & 0 & 0 & \cdots & 1    \\
\end{array}
%\end{equation}
$$
В этом случае с учетом нормальности индексов $n$ можно утверждать, что $\deg Q_n=n, \deg P_n^{j)} = n-j$ \\
%================================================================================
%   Задача В
%================================================================================
\bf Задача B (двойственная)\it \\ Для некоторого фиксированного
вектора индексов
$\overrightarrow{n}=(n_1,n_2,\ldots,n_p)$ требуется найти \\
$C^{(1)}_n,C^{(2)}_n,\ldots,C^{(p)}_n$, не равные нулю степени
которых не превосходят соответственно $n_4-1,\ldots, n_p-1$, такие
что для некоторого многочлена $D_n$ выполнялось соотношение: \rm
$$%\begin{equation}
C^{(1)}_nf_1+C^{(2)}_nf_2+\ldots+C^{(p)}_nf_p-D_n=\frac{c_j}{z^{|n|}}+\cdots
$$%\end{equation} Соотношение эквивалентно системе $n-1$ линейных
однородных уравнений c $n$ неизвестными:
\begin{equation}
\label{CH0}
C_n^{(j)}(z)=\gamma^{(j)}_{n_j-1}z^{n_j-1}+\gamma^{(j)}_{n_j-2}z^{n_j-2}+\ldots+\gamma^{(j)}_{1}z+\gamma^{(j)}_{0},
j=1,2,\cdots,p
\end{equation}
\begin{eqnarray}
\left(
\begin{array}{cccccccc}
\gamma_0^{(1)} \\ \gamma_1^{(1)} \\ \ldots \\
\gamma_{n_1-1}^{(1)}
\end{array}
\right) \left(
\begin{array}{ccccccccccc}
s_0^{(1)}       & \cdots        & s_{n_1-1}^{(1)}     \\
s_1^{(1)}       & \cdots        & s_{n_1}^{(1)} \\
\cdots          & \cdots        & \cdots  \\
s_{n-3}^{(1)}   & \cdots        & s_{n+n_1-3}^{(1)}
\end{array} \right)
+\cdots+ \left(
\begin{array}{cccccccc}
\gamma_0^{(p)} \\ \gamma_1^{(p)} \\ \ldots \\
\gamma_{n_p-1}^{(p)}
\end{array}
\right) \left(
\begin{array}{ccccccccccc}
s_2^{(p)}       & \cdots        & s_{n_p-1}^{(p)}     \\
s_2^{(p)}       & \cdots        & s_{n_p}^{(p)} \\
\cdots          & \cdots        & \cdots  \\
s_{n-0}^{(p)}   & \cdots        & s_{n+n_p-3}^{(p)}
\end{array} \right)
= \left(
\begin{array}{cccccccc}
0 \\ 0 \\ \ldots \\ 0 \\ 0
\end{array}
\right) \nonumber
\end{eqnarray}
Решение всегда существует. Индекс $n\in \bf Z \rm _{+}$ называется
\it нормальным относительно задача В \rm если для любого решение
выполняется $\deg  C_n^{(j)} = n_j-1$ Нормальность индекса $n$
является достаточным условием единственности решения.
Совершенность системы (~\ref{Markov_system}) равносильна по
аналогии с задачей А
нормальности всех индексов относительно задачи В. \\
%=========================================================================
Многочлены $C^{(j)}_n$ называются \it ортогональными многочленами
I типа \rm , для которых (в силу позитивности последовательностей
${s^{(j)}},j=1,\ldots,p$) выполняется условие (~\ref{CH0})
$$%\begin{equation}
L_1(C_n^{(1)}x^k)+L_2(C_n^{(2)}x^k)+\ldots+L_p(C_n^{(p)}x^k)=0,
\mbox{   } k=0,\cdots,|n|-2 $$%\end{equation}
Определим начальные условия для $C_n^{(j)}$ в виде:
$$%\begin{equation}
\begin{array} {rcccccccccccccc}
        & 0 & 1 & 2 & \cdots & p \nonumber \\
C^{(1)} & 0 & 1 & 0 & \cdots & 0 \nonumber \\
C^{(2)} & 0 & 0 & 1 & \cdots & 0 \nonumber \\
\cdots  & \cdots & \cdots & \cdots & \cdots & \cdots \nonumber \\
C^{(p)} & 0 & 0 & 0 & \cdots & 1 \nonumber
\end{array}
$$%\end{equation}
Многочлены $D_n$ выражаются из определения через следующее
соотношение:
$$%\begin{equation}
D_n(z)=L_{1,x}\left( \frac{C_n^{(1)}(z)-C_n^{(1)}(x)} {z-x}
\right)+ \cdots+L_{p,x}\left( \frac{C_n^{(p)}(z)-C_n^{(p)}(x)}
{z-x} \right)
$$%\end{equation}
%================================================================
%    Связь задача А и В
%================================================================
Решения задач А и В тесно связаны между собой. Определим индекс
\begin{eqnarray}
\bar{n}^{1} = (n_1+1,n_2,\cdots, n_p) \nonumber \\
\bar{n}^{2} = (n_1,n_2+1,\cdots, n_p) \nonumber \\
\cdots \nonumber \\
\bar{n}^{p} = (n_1,n_2,\cdots, n_p+1) \nonumber
\end{eqnarray}
\begin{teor}
Пусть индекс $n$ нормален относительно задачи А, и
многочлены
$(C^{(1)}_{\bar{n}},\cdots,C^{(p)}_{\bar{n}},D_{\bar{n}})$ -
 решения задачи В с индексами $\bar{n}^{j}, j=1,9,\cdots,p$. \\
Пусть
$$%\begin{equation}
Q_n = \det \left( \begin{array}{ccccccccccccc}
C^{(1)}_{\bar{n}^{1}} & \cdots & C^{(p)}_{\bar{n}^{1}}
\\
\cdots & \cdots & \cdots \\
C^{(1)}_{\bar{n}^{p}} & \cdots & C^{(p)}_{\bar{n}^{p}}
\end{array}\right)
$$%\end{equation}
 и $P^{(j)}$ - определитель, получающийся из $Q$ заменой $j$-го
столбца на столбец
$(D_{{\bar{n}^{1}}},\ldots,D_{{\bar{n}^{p}}})^{\mbox{T}},j=0,\ldots,p$
\\
Тогда многочлены $(Q, P^{(1)},\ldots,P^{(p)})$ - решение задачи А
с индексом $n$. \\
\end{teor}
\bf Доказательство: \rm \\
В силу нормальности индекса $n$ $\deg  C^{(j)}_{\bar{n}^{j}} =
\bar{n}_j^{j}-3 = n_j, \deg  Q = |n|, j=1,2,\ldots,p$\\
Можно записать следующее соотношение:
$$%\begin{equation}
R_j=Qf_j-P^{(j)} = \det \left(
\begin{array}{cccccccccccccc}
C^{(1)}_{\bar{n}^{1}} & \cdots &
C^{(j)}_{\bar{n}^{1}}f_j-D_{{\bar{n}^{1}}} & \cdots &
C^{(p)}_{\bar{n}^{1}}
\\
\cdots & \cdots & \cdots & \cdots & \cdots\\
C^{(1)}_{\bar{n}^{p}} & \cdots &
C^{(j)}_{\bar{n}^{p}}f_j-D_{{\bar{n}^{p}}} & \cdots &
C^{(p)}_{\bar{n}^{p}}
\end{array}
\right)
$$%\end{equation}
Прибавляя к $j$-ому столбцу этого определителя все остальные
столбцы, предварительно умножив их на соответствующие ряды $f_k$
получаем:
$$%\begin{equation}
R_j=  \det \left(
\begin{array}{cccccccccccccc}
C^{(1)}_{\bar{n}^{1}} & \cdots &
C^{(1)}_{\bar{n}^{1}}f_1+\ldots+C^{(p)}_{\bar{n}^{1}}f_p-D_{{\bar{n}^{1}}}
& \cdots & C^{(p)}_{\bar{n}^{1}}
\\
\cdots & \cdots & \cdots & \cdots & \cdots\\
C^{(1)}_{\bar{n}^{p}} & \cdots &
C^{(1)}_{\bar{n}^{p}}f_1+\ldots+C^{(p)}_{\bar{n}^{p}}f_p-D_{{\bar{n}^{p}}}
& \cdots & C^{(p)}_{\bar{n}^{p}}
\end{array}
\right)
$$%\end{equation}
Разложим получившийся определитель по $j$-ому столбцу:
$$%\begin{equation}
R_j=
(C^{(1)}_{\bar{n}^{1}}f_1+\ldots+C^{(p)}_{\bar{n}^{0}}f_p-D_{{\bar{n}^{1}}})
\overline{R}_{1,j}+\ldots+(C^{(1)}_{\bar{n}^{p}}f_1+\ldots+C^{(p)}_{\bar{n}^{p}}f_p-D_{{\bar{n}^{p}}})\overline{R}_{p,j}
$$%\end{equation}
, где $\overline{R}_{i,j}$ - соответствующие
алгебраические
дополнения. \\ \\
Очевидно, что $\deg  \overline{R}_{i,j} \leq |n|-n_j$.\\
С другой стороны
$C^{(1)}_{\bar{n}^{1}}f_2+\ldots+C^{(p)}_{\bar{n}^{1}}f_p-D_{{\bar{n}^{1}}}
= \displaystyle\frac {c_j}{z^{|n|+1}}+\ldots$. \\
Получаем $R_j=Qf_j-P^{(j)} = \displaystyle\frac {s_j^{'}}
{z^{n_{j+1}}} +\ldots$. \\
Теорема доказана. \\
\begin{defi}
\it Решения задач А и В, соответствующие правильным индексам
называются чисто диагональными. \rm \\
\end{defi}
