\chapter{Теоретические аспекты векторных ортогональных многочленов}
\section{Норма матрицы, Число обусловленности}

Нормы матрицы $A=(a_{j,k})_{j,k=0}^{2n-1}$:
\begin{equation}
\mbox{sup-norm} \parallel A \parallel_\infty={\max\limits_{0\leq j \leq 2n-1}} \sum\limits_{k=0}^{2n-1}{\left| a_j,k \right| }
\end{equation}
Норма Фробениуса:
\begin{equation}
\parallel A \parallel_F=\sqrt{\left( \sum\limits_{j,k=0}^{2n-1}{a_{j,k}^2}\right)}
\end{equation}
Норма Холдера:
\begin{equation}
\parallel A \parallel_2=
\end{equation}
\begin{equation}
\parallel A \parallel_2\leq \parallel A \parallel_F \leq \parallel \sqrt{n} \parallel A \parallel_2
\end{equation}

Обычно в качестве числа обусловленности гладкого нелинейного
отображения $M:\mbox{ \bf R \rm}^{2n} \rightarrow \mbox{ \bf R \rm}^{2n}$
выбирают:
\begin{equation}
\mbox{cond} M(x) = \lim\limits_{\parallel \Delta x \parallel \rightarrow 0} \sup
\frac{\parallel M(x+\Delta x)-M(x) \parallel}
{\parallel M(x) \parallel}\cdot
\frac{\parallel x \parallel}
{\parallel \Delta x \parallel}=
\frac{\parallel x \parallel}
{\parallel M(x) \parallel}
\parallel M^{'}(x) \parallel
\end{equation}
где $M^{'}(x)=\left( \frac{\partial y_j} {\partial x_k}
\right)_{j,k}$ - матрица якобиана, $\parallel \cdot \parallel$ -
соответствующая норма вектора или матрицы. В ~\cite{Beckermann1}
добавляется дополнительный масштабирующий параметр
\begin{equation}
\mbox{cond}_D M(x)=\frac{\parallel M^{'}(x)D \parallel_F}
{\parallel y \parallel_2}
\parallel D^{-1}x \parallel_2
\end{equation}
где $D$ - некоторая диагональная матрица ($D=I$ или $D=D_{nor}$)
\begin{equation}
D^2_{nor}=\mbox{diag}(\int\pi^2(x)_k dl(x))_{k=0,\ldots,2n-1})
\end{equation}
где $l$ - некоторая мера, соответствующая многочленам
$\pi$,$(D_nor^{-1}\cdot m)$ -
вектор \it нормализованных \rm модифицированных моментов
\begin{equation}
\tilde{m}_k=\displaystyle \frac{m_k}{\sqrt{\int\pi_k^2(x) dl(x))}}
\end{equation}
построенных по ортонормированным многочленам $\pi(x)$
\\
Нас интересуют числа обусловленности следующих отображений: \\
1. От квадратуры Гаусса-Кристоффеля к коэффициентам рекуррентных соотношений
\begin{equation}
H_n:\left[\tau_1,\ldots, \tau_n,\lambda_1,\ldots,\lambda_n \right]^{T}\ss\left[\alpha_0,\ldots,\alpha_{n-1},\beta_0,\ldots,\beta_{n-1}\right]^{T}
\end{equation}
2. От обычных моментов к квадратуре Гаусса-Кристоффеля
\begin{eqnarray}
G_n^{0}:s=\left[s_0,\ldots,s_{2n-1}\right]^{T} \ss \left[\tau_1,\ldots, \tau_n,\lambda_1,\ldots,\lambda_n \right]^{T} \nonumber
\end{eqnarray}
3. От модифицированных моментов к квадратуре Гаусса-Кристоффеля
\begin{eqnarray}
G_n:m=\left[m_0,\ldots,m_{2n-1}\right]^{T} \ss \left[\tau_1,\ldots, \tau_n,\lambda_1,\ldots,\lambda_n \right]^{T} \nonumber \\
\mbox{cond}(G_n,m)=\frac{\parallel m \parallel}{\parallel G_n(m)\parallel}\parallel G_n^{'}(m) \parallel \nonumber
\end{eqnarray}
4. От обычных моментов к коэффициентам рекуррентных соотношений
\begin{eqnarray}
K_n^{0}:s=\left[s_0,\ldots,s_{2n-1}\right]^{T} \ss \left[\alpha_0,\ldots,\alpha_{n-1},\beta_0,\ldots,\beta_{n-1}\right]^{T}=G_n^{0} \cdot {H_n} \nonumber
\end{eqnarray}
5. От модифицированных моментов к коэффициентам рекуррентных соотношений
\begin{eqnarray}
K_n:\left[m_0,\ldots,m_{2n-1}\right]^{T} \ss \left[\alpha_0,\ldots,\alpha_{n-1},\beta_0,\ldots,\beta_{n-1}\right]^{T}=G_n \cdot {H_n} \nonumber \\
\mbox{cond}(K_n,m)=\frac{\parallel m \parallel}{\parallel K_n(m)\parallel}\parallel K_n^{'}(m) \parallel
=\mbox{cond}G_n \cdot {H_n}\leq \mbox{cond}G_n \cdot\mbox{cond} {H_n} \nonumber
\end{eqnarray}

\section{Число обусловленности $G_n^{0}$}

Источник ~\cite{GautschiW5}. Отображение $G_n^{0}$ эквивалентно решению
нелинейной системы уравнений
\begin{eqnarray}
s_j=\int{x^jd\mu(x)}=\sum\limits_{i=1}^{n}{\lambda_i^j\tau_i}, j=0,\ldots,2n-1
\end{eqnarray}
Якобиан обратного отображения (от квадратуры к обычным моментам) известен
и равен $\Phi=\Xi\cdot\Lambda$:
\begin{equation}
\Xi=
\left[
\begin{array}{cccccccccc}
1 &  \ldots & 1 & 0  & \ldots & 0 \\
\lambda_1 & \ldots & \lambda_{n} & 1 & \ldots & 1 \\
\lambda^2_1 & \ldots & \lambda^2_{n} & 2\lambda_1 & \ldots & 2\lambda_n\\
\ldots & \ldots & \ldots & \ldots & \ldots & \ldots  \\
\lambda^n_1 & \ldots & \lambda^n_{n} & n\lambda^{n-1}_1 & \ldots & n\lambda^{n-1}_n\\
\ldots & \ldots & \ldots & \ldots & \ldots & \ldots  \\
\lambda^{2n-1}_1 & \ldots & \lambda^{2n-1}_{n} & (2n-1)\lambda^{2n-2}_1 & \ldots & (2n-1)\lambda^{2n-2}_{n}\\
\end{array}
\right]
\end{equation}
где $\Lambda=diag(1,\ldots,1,\tau_1,\ldots,\tau_n)$
Следовательно имеем:
\begin{equation}
\mbox{cond}G_n^{0}=\frac
{\parallel (s_0,\ldots,s_{2n-1}) \parallel}
{\parallel (\lambda_1,\ldots,\lambda_{n},\tau_{1},\ldots,\tau_{n})\parallel}
\parallel \Lambda^{-1}\Xi^{-1} \parallel
\end{equation}
\bf Теорема \rm
\it Пусть $\lambda_1,\ldots,\lambda_n$ взаимно положительны, определим
в качестве нормы матрицы максимальную сумму модулей элементов по строкам.
Тогда:
\begin{equation}
u_1\leq \parallel \Xi^{-1} \parallel \leq max(u_1,u_2)
\end{equation}
где
\begin{equation}
u_1=\max\limits_{1\leq i \leq n}b^{(1)}_i\prod\limits_{k=1,k\not=i}^{n}
\left(
\frac{1+\lambda_k}
{\lambda_i-\lambda_k}
\right)^2, \quad b_i^{(1)}=1+\lambda_i
\end{equation}

\begin{eqnarray}
u_2=\max\limits_{1\leq i \leq n}b^{(2)}_i
\prod\limits_{k=1,k\not = i}^{n}
\left(
\frac{1+\lambda_k}
{\lambda_i-\lambda_k}\right)^2, \quad
b_i^{(2)}=
1+2(1+\lambda_i)
\left|
\sum\limits_{k=1,k\not=i}^{n}
{
\frac{1}
{\lambda_i-\lambda_k}
}
\right|
\end{eqnarray}
\rm
Доказательство: \\
Известно, что
\begin{equation}
\Xi^{-1}=\left[
\begin{array} {ccccc}
A \\
B
\end{array}
\right],A=(a_{i,j}),B=(b_{i,j})_{i,j=1,\ldots,2n}
\end{equation}
где
\begin{eqnarray}
\sum\limits_{j=1}^{2n}{\left| a_{i,j}\right|} \leq
b^{(2)}_i\prod\limits_{k\not= i}{\left(
\frac{1+\lambda_k}{\lambda_i-\lambda_k}\right)^2} \nonumber \\
\sum\limits_{j=1}^{2n}{\left| b_{i,j}\right|} =
 b^{(1)}_i\prod\limits_{k\not= i}{\left(
\frac{1+\lambda_k}{\lambda_i-\lambda_k}\right)^2} \nonumber
\end{eqnarray}
Откуда и следует вышесказанное.
Конечная оценка Гаучи
\begin{equation}
\mbox{cond}G_n^{0}>min(s_0,\frac{1}{s_0})\frac{(17+6\sqrt{8})^n}{64n^2}
\end{equation}

\section{Число обусловленности $G_n$}

Источник ~\cite{GautschiW5}:
\begin{equation}
\mbox{cond}_{D_{nor}}G_n=\frac{\parallel (\tilde{m}_0,\ldots,\tilde{m}_{2n-1}) \parallel_2}
{\parallel (\lambda_1,\ldots,\lambda_{n},\tau_{1},\ldots,\tau_{n})\parallel_2}
\sqrt{\int {\sum\limits_{i=1}^{n}\left(
h_i^2(x)+\frac{1}{\tau_i^2}k_i^2(x)
\right)dl(x)}}
\end{equation}
где $\parallel (\tilde{m}_0,\ldots,\tilde{m}_{2n-1}) \parallel_2=
\displaystyle\sum\limits_{i=0}^{2n-1}{\frac{1}{\tilde{m}^2_i}}$ и
$\parallel (\lambda_1,\ldots,\lambda_{n},\tau_{1},\ldots,\tau_{n})\parallel_2=
\sum\limits_{i=1}^{n}{(\lambda_i^2+\tau_i^2)}$ \\
Запишем из определения нормализованных модифицированных моментов
\begin{eqnarray}
\tilde{m}_k=\displaystyle \frac{\int \pi_k d\mu(x)}{\sqrt{\int\pi_k^2(x) dl(x))}}=
\frac{1}{\sqrt{d_k}}\sum\limits_{i=0}^{2n-1}{\pi_k(\lambda_i)\tau_i},k=0,1,\ldots,2n-1 \nonumber
\end{eqnarray}
Якобиан обратного отображения из выше сказанного равен
$\Phi=D^{-1}\Xi\Lambda$, где
$D=diag(\sqrt{d_0},\ldots,\sqrt{d_{2n-1}}),\Lambda=diag(1,\ldots,1,\tau_1,\ldots,\tau_n)$ и
\begin{equation}
\Xi=\left[
\begin{array}{cccccccccc}
\pi_0(\lambda_1) & \ldots & \pi_0(\lambda_n) & \pi^{'}_0(\lambda_1) & \ldots & \pi^{'}_0(\lambda_n) \\
\pi_1(\lambda_1) & \ldots & \pi_1(\lambda_n) & \pi^{'}_1(\lambda_1) & \ldots & \pi^{'}_1(\lambda_n) \\
\ldots & \ldots & \ldots & \ldots & \ldots & \ldots  \\
\pi_{2n-1}(\lambda_1) & \ldots & \pi_{2n-1}(\lambda_n) & \pi^{'}_{2n-1}(\lambda_1) & \ldots & \pi^{'}_{2n-1}(\lambda_n) \\
\end{array}
\right]
\end{equation}
Число обусловленности соответственно:
\begin{eqnarray}
\mbox{cond}_{D_{nor}}G_n=\frac{\parallel (\tilde{m}_0,\ldots,\tilde{m}_{2n-1}) \parallel_2}
{\parallel (\lambda_1,\ldots,\lambda_{n},\tau_{1},\ldots,\tau_{n})\parallel_2}
\parallel\Phi^{-1}(\lambda,\tau) \parallel
\end{eqnarray}
далее $\parallel \Phi^{-1} \parallel=\parallel \Lambda^{-1} \Xi^{-1} D \parallel
\leq \parallel \Lambda^{-1} \Xi^{-1} D \parallel_F$
Из предыдущей главы
\begin{equation}
\Xi^{-1}=\left[
\begin{array}{ccccc}
A \\
B
\end{array}
\right],
(\Lambda^{-1}\Xi^{-1}D)=\sqrt{d}\left[
\begin{array}{cccccccc}
A \\
\displaystyle\frac{1}{\tau}B
\end{array}
\right]
\end{equation}
Соответственно
\begin{equation}
\parallel \Lambda^{-1} \Xi^{-1}D \parallel^2_F=
\sum\limits_{i=1}^n
{
\sum\limits_{j=1}^{2n}
{
d_{j-1}
\left(
a_{i,j}^2+\displaystyle\frac{1}{\tau_i^2}b_{i,j}^2
\right)
}}
\end{equation}
Далее
\begin{eqnarray}
h_k(x)=\sum\limits_{i=1}^{2n}{a_{k,i}\pi_{i-1}(x)}, k_n=\sum\limits_{i=1}^{2n}{b_{k,i}\pi_{i-1}(x)} \nonumber \\
\int{\pi^2_k(x)dl(x)}=\sum\limits_{i=1}^{2n}{d_{i-1}a_{k,i}^2},\int{k_k^2(x)dl(x)}=\sum\limits_{i=1}^{2n}{d_{i-1}b_{k,i}^2} \nonumber \\
\parallel \Lambda^{-1} \Xi^{-1} D \parallel_F^2=\int {\sum\limits_{i=1}^{n}\left(
h_i^2(x)+\frac{1}{\tau_i^2}k_i^2(x)
\right)dl(x)} \nonumber
\end{eqnarray}
Отметим, что интеграл является полиномом степени не выше $4n-2$


\section{Число обусловленности отображения $H_n (p=2)$}

Источник ~\cite{Beckermann1} \\
\begin{equation}
\mbox{cond}_{D_{opt}}H_n\leq 6\sqrt{2n}\left[
n+\sqrt{
\left(
\mu^2\sum\limits_{j=1}^{n}\sum\limits_{k=1}^{n}
\frac{1}{\tau_k}\prod_{i=1,i\not=k}^{n}{
\left(
\frac{\lambda_j-\lambda_i}{\lambda_k-\lambda_i}\right)^2
}\right)}
\right]
\end{equation}


\section{Число обусловлености отображения $K_n (p=2)$  }

Источник ~\cite{Fischer1}:
\begin{equation}
\mbox{cond}_{D_{nor}}K_n=\frac{\parallel (\tilde{m}_0,\ldots,\tilde{m}_{2n-1}) \parallel_2}
{\parallel (\alpha_0,\ldots,\alpha_{n-1},\beta_{0},\ldots,\beta_{n-1})\parallel_2}
\sqrt{\sum\limits_{j=0}^{2n-1} {w_2}}
\end{equation}
, где $w_2=\sum\limits_{j=0}^{n-1}{\psi_{2j}^2+\psi_{2j+1}^2}=
\sum\limits_{j=0}^{n-1}{\beta_j^2(q_j^2-q_{j-1}^2)^2+
(\sqrt{\beta_{j+1}}q_jq_{j+1}-\sqrt{\beta_j}q_{j-1}q_j)^2}$
В ~\cite{Beckermann1} на основе точных формул Фишера приведены следующие оценки для мер
с компактным расположением. $\mu(x)$ - мера, соответствующая обычным моментам,
$l(x)$ - мера, соответствующая модифицированным моментам. \\


\subsection{ Вычисление частных производных $\frac{\partial \alpha_j}{\partial m_k}$ и $\frac{\partial \beta_j}{\partial m_k}$ }
Пусть ($\pi_0,\ldots,\pi_N$) некоторый базис в пространстве $\bf P \rm_N$ полиномов степени не выше $N$.
Для любого полинома $q\in \bf P \rm_N$ степени не выше $N$ можно записать разложение по базису:
\begin{equation}
\label{Basisq}
q(x)=\sum\limits_{j=0}^{N}{W_j(q(x))\pi_j(x)}
\end{equation}
где $W_j$ - некоторый линейный функционал.
В случае обычных моментов:
\begin{eqnarray}
q(x)=\sum\limits_{j=0}^{N} { \frac { q^{(j)}(x)} {j!}  x^j} \nonumber \\
\pi_j(x)=x^j, \mbox{   } W_j(q(x))=\frac { q^{(j)}(x)} {j!} \nonumber
\end{eqnarray}
\bf Лемма 1 \rm \\
Из (~\ref{Ord Mod moments}) и (~\ref{Basisq}) следует $\int{q(x)d\mu(x)}=\sum\limits_{j=0}^{N}{W_j(q)m_j}$ \\
Пусть $q$ - зависящие от модифицированных моментов имеют непрерывные частные производные в некоторой окрестности $m$. \\
Тогда,
\begin{equation}
\frac{\partial}{\partial m_k} \int q(x) d\mu(x)=W_k(q)+\int { \frac {\partial q} {\partial m_k}  d\mu(x)}
\end{equation}
\bf Доказательство: \rm
\begin{eqnarray}
\frac{\partial}{\partial m_k} \int q(x) d\mu(x)=
W_k(q)+\sum\limits_{j=0}^{N} { \frac {\partial W_j(q)} {\partial m_k} m_j}= \nonumber \\
W_k(q)+\sum\limits_{j=0}^{N} { W_k \left( \frac {\partial q} {\partial m_k} \right) m_j}=
W_k(q)+\int { \frac {\partial q} {\partial m_k}  d\mu(x)} \nonumber
\end{eqnarray}
Для $(i<j)$ из (~\ref{Orthq}) получаем :
\begin{equation}
\label{Orth1}
\int q_i(x)\frac{\partial q_j(x)}{\partial m_k}d\mu(x)=-W_k(q_iq_j)
\end{equation}
\begin{eqnarray}
\frac{\partial}{\partial m_k} \int q_i(x)q_j(x)d\mu(x)=
W_k(q_iq_j)+\int \frac{\partial q_i(x)}{\partial m_k}q_j(x)d\mu(x)+\int q_i(x)\frac{\partial q_j(x)}{\partial m_k}d\mu(x)= \nonumber \\
W_k(q_iq_j)+\int q_i(x)\frac{\partial q_j(x)}{\partial m_k}d\mu(x)=0 \nonumber
\end{eqnarray}
Из второго соотношения (~\ref{Orthq}) получаем:
\begin{equation}
\label{Orth2}
\int q_j(x)\frac{\partial q_j(x)}{\partial m_k}d\mu(x)=-\frac{1}{2}W_k(q_j^2)
\end{equation}
\begin{eqnarray}
\frac{\partial}{\partial m_k} \int q_j^2(x)d\mu(x)=W_k(q_j^2)+2\int{q_j\frac{\partial q_j(x)}{\partial m_k}d\mu(x)}=0 \nonumber
\end{eqnarray}
Перепишем и продифференцируем рекуррентное соотношение:
\begin{eqnarray}
\label{Rec1}
\beta_{j+1}^{1/2}q_{j+1}(x)=(x-\alpha_j)q_j(x)-\beta_{j}^{1/2}q_{j-1}(x) \nonumber \\
\frac {\partial \beta_{j+1}^{1/2}} {\partial m_k} q_{j+1}(x)+
\beta_{j+1}^{1/2} \frac {\partial q_{j+1}} {\partial m_k}= \nonumber \\
-\frac {\alpha_j} {\partial m_k} q_j(x)+
(x-\alpha_j)\frac {q_j(x)} {\partial m_k}-
\frac {\beta_j^{1/2}} {\partial m_k}q_{j-1}(x)
-\beta_j^{1/2}\frac {q_{j-1}(x)} {\partial m_k}
\end{eqnarray}
Из рекуррентного соотношения:
\begin{eqnarray}
\beta_{j+2}^{1/2}q_{j+2}(x)=(x-\alpha_{j+1})q_{j+1}(x)-\beta_{j+1}^{1/2}q_{j}(x) \nonumber \\
\beta_{j+2}^{1/2}q_{j+2}(x)=(x-\alpha_{j+1}+\alpha_j-\alpha_j)q_{j+1}(x)-\beta_{j+1}^{1/2}q_{j}(x) \nonumber \\
(x-\alpha_j)q_{j+1}(x)=\beta_{j+2}^{1/2}q_{j+2}(x)+(\alpha_{j+2}-\alpha_{j+1})q_{j+1}(x)+\beta_{j+1}^{1/2}q_{j}(x) \nonumber
\end{eqnarray}
Домножим (~\ref{Rec1}) на $q_{j+1}(x)$:
\begin{eqnarray}
\frac {\partial \beta_{j+1}^{1/2}} {\partial m_k}
+\beta_{j+1}^{1/2} \frac {\partial q_{j+1}(x) } {\partial m_k}q_{j+1}(x)
 = (x-\alpha_j)q_{j+1}(x)\frac {q_j(x)} {\partial m_k} \nonumber
\end{eqnarray}
Подставляем выражение для $(x-\alpha_j)q_{j+1}(x)$:
\begin{eqnarray}
\frac {\partial \beta_{j+1}^{1/2}} {\partial m_k}
+\beta_{j+1}^{1/2} \frac {\partial q_{j+1}(x) } {\partial m_k}q_{j+1}(x)
 =\beta_{j+1}^{1/2} q_{j}(x)\frac {q_j(x)} {\partial m_k} \nonumber
\end{eqnarray}
Проинтегрируем полученное выражение и учтем (~\ref{Orth2})
\begin{eqnarray}
\frac {\partial \beta_{j+1}^{1/2}} {\partial m_k}-\frac{1}{2}\beta_{j+1}^{1/2}W_k(q_{j+1}^2)=-\frac{1}{2}W_k(q_k^2) \nonumber \\
\frac {\partial \beta_{j+1}^{1/2}} {\partial m_k}=\frac{1}{2}\beta_{j+1}^{1/2}W_k(q_{j+1}^2-q_k^2) \nonumber
\end{eqnarray}
Домножим на $2\beta^{1/2}$
\begin{equation}
\frac {\partial \beta_{j+1}} {\partial m_k}=\beta_{j+1}W_k(q_{j+1}^2-q_k^2)
\end{equation}
Домножим (~\ref{Rec1}) на $q_{j}(x)$
\begin{eqnarray}
\frac {\partial \beta_{j+1}^{1/2}} {\partial m_k}q_{j+1}(x)q_j(x)
+\beta_{j+1}^{1/2} \frac {\partial q_{j+1}(x) } {\partial m_k}q_{j}(x)
 = -\frac {\alpha_j} {\partial m_k} q_j(x)q_j(x)+
(x-\alpha_j)q_j(x) \frac {\partial q_j(x)} {\partial m_k} \nonumber
\end{eqnarray}
Учитывая, что
$(x-\alpha_j)q_j(x)=\beta_{j+1}^{1/2}q_{j+1}(x)+\beta_{j}^{1/2}q_{j-1}(x)$
перепишем:
\begin{eqnarray}
\frac {\partial \beta_{j+1}^{1/2}} {\partial m_k}q_{j+1}(x)q_j(x)
+\beta_{j+1}^{1/2} \frac {\partial q_{j+1}(x) } {\partial m_k}q_{j}(x)
 = -\frac {\alpha_j} {\partial m_k} q_j(x)q_j(x)+
\beta_{j}^{1/2}\frac {\partial q_j(x)} {\partial m_k}q_{j-1}(x) \nonumber
\end{eqnarray}
Интегрируя полученное выражение и учитывая (~\ref{Orth2}) получаем:
\begin{equation}
-\beta_{j+1}^{1/2}W_k(q_{j+1}q_j)=-\frac {\alpha_j} {\partial m_k}
-\beta_j^{1/2}W_k(q_jq_{j-1})
\end{equation}
\bf Теорема 1 \rm \\
При выполнении  (~\ref{Orthq}), (~\ref{Ord Mod moments}), (~\ref{Basisq})
частные производные для коэффициентов рекуррентного
соотношения выражаются как:
\begin{equation}
\frac{\partial \alpha_j} {\partial m_k}=\beta_{j+1}^{1/2}W_k(q_jq_{j+1})-\beta_{j}^{1/2}W_k(q_{j-1}q_{j}), \mbox {  для  } 2j+1 \leq N
\end{equation}
\begin{equation}
\frac{\partial \beta_j} {\partial m_k}=\beta_jW_k(q_j^2-q_{j-1}^2), \mbox{ для } 2j \leq N
\end{equation}

\subsection{Норма якобиана отображения $K_n$}

Якобиан $K^{'}_n[2n\times 2n]$ имеет следующий вид:
\begin{equation}
K^{'}_n=
\left(
\begin{array}{ccccccccccccc}
\displaystyle\frac{\partial \alpha_0} {\partial m_0} &
\displaystyle\frac{\partial \alpha_1} {\partial m_0} & \cdots &
\displaystyle\frac{\partial \alpha_{n-1}} {\partial m_0} &
\displaystyle\frac{\partial \beta_0} {\partial m_0} &
\displaystyle\frac{\partial \beta_1} {\partial m_0} & \cdots &
\displaystyle\frac{\partial \beta_{n-1}} {\partial m_0} \\

\displaystyle\frac{\partial \alpha_0} {\partial m_1} &
\displaystyle\frac{\partial \alpha_1} {\partial m_1} & \cdots &
\displaystyle\frac{\partial \alpha_{n-1}} {\partial m_1} &
\displaystyle\frac{\partial \beta_0} {\partial m_1} &
\displaystyle\frac{\partial \beta_1} {\partial m_1} & \cdots &
\displaystyle\frac{\partial \beta_{n-1}} {\partial m_1} \\
\cdots & \cdots & \cdots & \cdots & \cdots & \cdots & \cdots & \cdots & \\
\displaystyle\frac{\partial \alpha_0} {\partial m_{2n-1}} &
\displaystyle\frac{\partial \alpha_1} {\partial m_{2n-1}} & \cdots &
\displaystyle\frac{\partial \alpha_{n-1}} {\partial m_{2n-1}} &
\displaystyle\frac{\partial \beta_0} {\partial m_{2n-1}} &
\displaystyle\frac{\partial \beta_1} {\partial m_{2n-1}} & \cdots &
\displaystyle\frac{\partial \beta_{n-1}} {\partial m_{2n-1}} \\
\end{array}
\right)
\end{equation}

\begin{eqnarray}
\psi_{2j}(x)=\beta_j(q^2_j(x)-q^2_{j-1}(x)) \nonumber \\
\psi_{2j+1}(x)=\beta_{j+1}^{1/2}q_j(x)q_{j+1}(x)-\beta_j^{1/2}q_{j-1}(x)q_{j}(x) \nonumber \\
 j=0,\ldots,n-1 \nonumber
\end{eqnarray}

\begin{equation}
K^{'}_n=\Psi=(\psi_{i,j})_{i,j=0,2n-1},\mbox{   } \psi_{i,j}=W_j(\psi_i)
\end{equation}
Введем следующие обозначения:
\begin{eqnarray}
w_{\infty}(\psi_j)=\sum\limits_{k=0}^{N} {\mid W_k(\psi_j) \mid} \nonumber \\
w_{F}(\psi_j)=\sum\limits_{k=0}^{N} {W_k^2(\psi_j)} \nonumber
\end{eqnarray}
Нормы якобиана $K^{'}_n$ выражаются:
\begin{equation}
\parallel K^{'}_n \parallel _{\infty}=\parallel \Psi  \parallel _{\infty}=
\max\limits_{0\leq j \leq 2n-1} w_{\infty}(\psi_j)
\end{equation}
\begin{equation}
\parallel K^{'}_n \parallel _{F}=\parallel \Psi  \parallel _{F}=
\sqrt{ \sum\limits_{j=0}^{2n-1}{w_F(\psi_j)} }
\end{equation}
\bf Лемма 2. \rm \\
Для обычных моментов \\
1. Если $\psi_j$ - многочлен с чередующимися по знаку элементами,
то $ w_{\infty}(\psi_j)=\mid \psi(-1) \mid$ \\
2. Если $\psi_j$ - многочлен только с четными (или только нечетными)
степенями и чередующимся знаком, то $ w_{\infty}(\psi_j)=\mid \psi_j(i) \mid$ \\
3. В общем случае $w_F$ может быть выражено как $w_F(\psi_j)=\frac{1}{2\pi}\int\limits_{0}^{2\pi}{\mid \psi_j(e^{i\phi}) \mid ^{2}d\phi}$

