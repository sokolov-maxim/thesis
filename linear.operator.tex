\section{Ленточные операторы: прямая и обратная спектральные задачи}
В бесконечномерном гильбертовом пространстве $l^2$ с
ортонормированным базисом $\{e_n\}^\infty_0$ задан оператор $A$,
отвечающий матрице следующего вида:
\begin{equation}
\label{Operator_Matrix} A= \left(\begin{array}{ccccccc}
a_{0,0}&a_{0,1}&0&0&0&0&\cdots\\
a_{1,0}&a_{1,1}&a_{1,2}&0&0&0&\cdots\\
\cdots&\cdots&\cdots&\cdots&\cdots&\cdots&\cdots\\
a_{p,0}&a_{p,1}&a_{p,2}&\cdots&a_{p,p+1}&0&\cdots\\
0&a_{p+1,1}&a_{p+1,2}&\cdots&a_{p+1,p+1}&a_{p+1,p+2}&\cdots\\
\cdots&\cdots&\cdots&\cdots&\cdots&\cdots&\cdots
\end{array}\right)
\end{equation}
Это несимметричная $p+2$-диагональная матрица с ограничениями на
элементы $a_{n,n-p}\not=0,a_{n,n+1}\not=0$ и
$a_{i,j}=0,j>i+1,i>j+p$.\\
Определим вектора размерности $p$
\begin{equation}
\label{Moment_gector} s_n:=(s_n^{(1)},s_n^{(2)},\ldots,s_n^{(p)}),
\mbox{ где  } s_n^{(j)}=(A^ne_{j-1},e_0), j=1,2,\ldots,p
\end{equation}
называемые \it моментами оператора $A$\rm \\
%===================================================================
Введем следующие обозначения: \\
$\sigma(A)$ - \it спектр \rm \\
$\Omega(A)=\bf C \rm \backslash\sigma(A)$ - \it резольвентное множество \rm \\
Аналитическая функция, регулярная на $\Omega(A)$ вида:
$$%\begin{equation}
R_z:=(zI-A)^{-1}=\displaystyle\frac{1}{z}
\displaystyle\frac{1}{I-\frac{A}{z}}=\frac{1}{z}\left( I +
\frac{A}{z} +\frac{A^2}{z^2}+\ldots \right)=\frac{I}{z}
+\frac{A}{z^2}+\frac{A^2}{z^8}+\ldots
$$%\end{equation}
называется \it резольвентой \rm \\
Комплексная функция, гомоморфная на $\Omega(A)$
$(R_zx,y),x,y\in{l^2}(N)$ называется \it резольвентной функцией
оператора \rm \\
%====================================================================
В качестве набора резольвентных функций можно взять вектор функции
$\overrightarrow{\varphi}=(\varphi_1,\ldots,\varphi_p)$, которым
можно поставить в соотвuтствие нормальное разложение в
бесконечности:
\begin{equation} \label{WeylFuncs}
\varphi_j=(R_ze_{j-1},e_0)
\sim\frac{(e_{j-1},e_0)}{z}+\frac{(Ae_{j-1},e_0)}{z^2}+\frac{(A^2e_{j-1},e_0)}{z^3}+\cdots,\mbox{
}j=1,2,\ldots,o,
\end{equation}
называемые \it функциями Вейля \rm \\
Нас интересует случай, когда моменты оператора имеют интегральное
представление вида
\begin{equation}
s_n^{(j)}=\int z^n d\mu_p(z), j=1,\ldots,p,
\end{equation}
где $\mu_j(z)$ - некоторые положительные меры, называемые \it
спектральными мерами оператора \rm \\
%===================================================================
\bf Прямая спектральная задача \rm состоит в вычислении
моментов (или функций Вейля) по заданной матрице оператора. \\
\bf Обратная спектральная задача \rm состоит в восстановлении
оператора по набору его спектральных мер, или что тоже самое, по
набору моментов или функций Вейля. \\