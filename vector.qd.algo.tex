\subsection{Новая версия векторного алгоритма QD}
\subsubsection{Основные определения}
Для некоторого $p=nk+d$ доопределим  определитель Ганкеля
(~\ref{H}). Пусть $H_n^{k,d} (H_n=H_n^{0,0}) $ - соответствующий
определитель Ганкеля размерности $n \times n$:
$$%\begin{equation}
H_n^{k,d}= \left|
\begin{array}{cccccccccccccccccccccc}
s_k^{d+1} & \cdots & s_{k+n-1}^{d+1} \\
\cdots & \cdots & \cdots \\
\end{array}
\right|
$$%\end{equation}
Обозначим как $Q_n^{k,d}$ семейство векторно ортогональных
многочленов относительно функционалов
$L^{\nu}=(L_1^{\nu},L_2^{\nu},\ldots,L_p^{\nu})$ определяемых \it
сдвинутыми \rm моментами
$$ L_j^{\nu}(z^n)=s_{n+\nu}^{(j)}$$
Многочлены $Q_n^{k,d}$ имеют соответствующее выражение через
определители Ганкеля $H_n^{p,d}$:
\begin{equation}
\label{Q_from_H}
\begin{array}{cc}
Q_n^{k,d}(z)=H_n^{k,d}(z)/H_n^{k,d}\\
\left|\begin{array}{ccccc}
s_{k}^{(d+1)} & s_{k+1}^{(d+1)} & \cdots & s_{k+n}^{(d+1)}\\
\cdots & \cdots & \cdots & \cdots\\
1               & z               & \cdots & z^n
\end{array}\right|
\times {\left|\begin{array}{cccc}
s_{k}^{(d+1)} & s_{k+1}^{(d+1)} & \cdots & s_{k+n-1}^{(d+1)}\\
\cdots & \cdots & \cdots & \cdots\\
\cdots & \cdots & \cdots & \cdots\\
\end{array}\right|}^{-1}
\end{array}
\end{equation}
\subsubsection{Вывод алгоритма}
%====================================================
% Theorem 1
%=====================================================
\begin{teor}
Для некоторго $\nu=pk+d$ имеют место следующие соотношения
\begin{eqnarray}
\label{QDExAlpha} Q_n^{k,d+1}=Q_n^{k,d}-\alpha_n^{\nu}
Q_{n-1}^{k,d+1} \mbox{, где }
\alpha_n^{\nu}=\frac{H^{k,d}_{n+1}H_{n-1}^{k,d+1} }{H_{n}^{k,d}
H_{n}^{k,d+1}}
\end{eqnarray}
\begin{equation}
\label{QDExBeta} Q_{n+1}^{k,d} =zQ_n^{k+1,d}-\beta_{n+1}^{{\nu}}
Q_n^{k,d} \mbox{, где } \beta_{n+1}^{\nu}
=\frac{H^{k+1,d}_{n+1}H_{n}^{k,d} }{H_{n+1}^{k,d} H_{n}^{k+1,d}}
\end{equation}
\begin{equation}
\label{QDExGamma} Q_{n+1}^{k,d} =zQ_n^{k+1,d+1}-\gamma_{n}^{\nu}
Q_n^{k,d} \mbox{, где } \gamma_{n}^{\nu}=
\frac{H^{k+1,d}_{n+1}H_{n}^{k,d+1} }{H_{n+1}^{k,d}
H_{n}^{k+1,d+1}}
\end{equation}
\end{teor}
%========================================
\noindent{\bf Доказательство :} \\
%We have the following Sylvester's identity
%\begin{equation}
%H_{n}^{k+1,d+1}H_{n+2}^{k,d} = H_{n+1}^{k,d}
%H_{n+1}^{k+1,d+1}-H_{n+1}^{k+1,d}H_{n+1}^{k,d+1}
%\end{equation}
1. Раскладывая определитель $H_{n+1}^{k,d}(z)$ по минору
$H_{n}^{k+1,d}$ в соответствии с тождеством Сильвестра имеем:
\begin{eqnarray}
H_{n}^{k+1,d}\cdot H_{n+1}^{k,d} (z)=zH_{n+1}^{k,d}  \cdot
H_{n}^{k+1,d}(z) -H^{k+1,d}_{n+1} \cdot H_{n}^{k,d}  (z) \nonumber
\end{eqnarray}
Поделив на $H_{n}^{k+1,d}\cdot H_{n+1}^{k,d} $ получаем
соотношение (~\ref{QDExBeta})
\begin{eqnarray}
Q_{n+1}^{k,d} (z)=zQ_{n}^{k+1,d}(z)
-\frac{H^{k+1,d}_{n+1}H_{n}^{k,d} }{H_{n+1}^{k,d} H_{n}^{k+1,d}}
Q_{n}^{k,d}  (z) \nonumber
\end{eqnarray}
2. Раскладывая определитель $H_{n+1}^{k,d}(z)$ по минору
$H_{n}^{k+1,d+1}$ в соответствии с тождеством Сильвестра имеем:
\begin{eqnarray}
H_{n}^{k+1,d+1}\cdot H_{n+1}^{k,d} (z)=zH_{n+1}^{k,d}  \cdot
H_{n}^{k+1,d+1}(z) -H^{k+1,d}_{n+1} \cdot H_{n}^{k,d+1}  (z)
\nonumber
\end{eqnarray}
Поделив на $H_{n}^{k+1,d+1}\cdot H_{n+1}^{k,d}$ получаем
соотношение (~\ref{QDExGamma})
\begin{eqnarray}
Q_{n+1}^{k,d} (z)=zQ_{n}^{k+1,d+1}(z)
-\frac{H^{k+1,d}_{n+1}H_{n}^{k,d+1} }{H_{n+1}^{k,d}
H_{n}^{k+1,d+1}} Q_{n}^{k,d+1}  (z) \nonumber
\end{eqnarray}
3. Комбинируя (~\ref{QDExBeta}) и (~\ref{QDExGamma})
\begin{eqnarray}
Q_{n}^{k,d+1} (z)=zQ_{n-1}^{k+1,d+1}(z)
-\beta_{n}^{{\nu+1}}
Q_{n-1}^{k,d+1}  (z) \nonumber \\
Q_{n}^{k,d} (z)=zQ_{n-1}^{k+1,d+1}(z) - \gamma_{n}^{\nu}
Q_{n-1}^{k,d+1}  (z) \nonumber
\end{eqnarray}
получаем соотношение (~\ref{QDExAlpha})
\begin{eqnarray}
Q_n^{k,d+1}(z)=Q_n^{k,d}(z)-\frac{H^{k,d}_{n+1}H_{n-1}^{k,d+1}
}{H_{n}^{k,d} H_{n}^{k,d+1}} Q_{n-1}^{k,d+1}(z) \nonumber
\end{eqnarray}
Третье соотношение является зависимым от двух предыдущих, и как
следствие верно следующее соотношение: $
\alpha_n^{\nu}=\beta_n^{\nu+1}-\gamma_{n-1}^{\nu} $ \\
Коэффициенты $\alpha_n^{\nu}, \beta_n^{\nu}, \gamma_n^{\nu}$
образуют \it векторную QD таблицу \rm следующего вида:
\begin{equation}
\begin{array}{ccccccccccccccccc}
\beta_1^0 & \alpha_1^0 & \beta_2^0 & \alpha_2^0 & \beta_3^0 & \cdots \\
\beta_1^1 & \alpha_1^1 & \beta_2^1 & \alpha_2^1 & \beta_3^1 & \cdots \\
\beta_1^2 & \alpha_1^2 & \beta_2^2 & \alpha_2^2 & \beta_3^2 & \cdots \\
\cdots & \cdots & \cdots & \cdots & \cdots & \cdots  \\
\end{array}
\end{equation}

%=========================================
%  Theorem 2
%=========================================
\begin{teor}
Новая версия QD алгоритма выражается следующими соотношениями
коэффициентов $\alpha$ и $\beta$ при $\nu=pk+d$:
\begin{equation}
\label{QDExRec} \beta_{n+1}^{\nu+1}+\alpha_n^{\nu+p} =
\beta_{n+1}^{\nu}+\alpha_{n+1}^{\nu}
\end{equation}
\begin{equation}
\beta_{n}^{\nu+1}\alpha_n^{\nu+p} =
\beta_{n+1}^{\nu}\alpha_{n}^{\nu}
\end{equation}
при начальных условиях:
\begin{equation}
\beta_1^{\nu} = s^{d+1}_{k+1}/s_{k}^{d+1},\mbox{   }
\alpha_0^{\nu}=0
\end{equation}
\end{teor}
\bf Доказательство: \rm \\
Используя (~\ref{QDExBeta}) и (~\ref{QDExAlpha}) мы получаем:
$$%\begin{equation}
\begin{array}{lllllllll}
Q_{n+1}^{k,d} & =xQ_n^{k+1,d}-\beta_{n+1}^{{\nu}} Q_n^{k,d} \\
&
=x(Q_n^{k+1,d+1}+\alpha_n^{\nu+p}Q_{n-1}^{k+1,d+1})-\beta_{n+1}^{{\nu}}
(Q_n^{k,d+1}+\alpha_n^{\nu}Q_{n-1}^{k,d+1}) \\
&
=(Q_{n+1}^{k,d+1}+\beta_{n+1}^{\nu+1}Q_n^{k,d+1})+\alpha_n^{\nu+p}(Q_n^{k,d+1}+\beta_n^{\nu+1}Q_{n-1}^{k,d+1})-\beta_{n+1}^{{\nu}}
(Q_n^{k,d+1}+\alpha_n^{\nu}Q_{n-1}^{k,d+1}) \\
& =
Q_{n+1}^{k,d+1}+(\beta_{n+1}^{\nu+1}+\alpha_{n}^{\nu+p}-\beta_{n+1}^{\nu})Q_n^{k,d+1}+(\alpha_n^{\nu+p}\beta_{n}^{\nu+1}-\alpha_n^{\nu}\beta_{n+1}^{\nu})Q_{n-1}^{k,d+1}
\end{array}
$$%\end{equation}
Сравнивая с $$
Q_{n+1}^{k,d+1}=Q_{n+1}^{k,d}-\alpha_{n+1}^{\nu} Q_{n}^{k,d+1}
$$
получаем соотношения теоремы. \\
\begin{teor} Вектор  $\overrightarrow{f}$ формальных степенных рядов
допускает разложение в векторную непрерывную дробь тогда, и только
тогда, когда определители Ганкеля $H_n^{k,d}$ не равны нулю
\end{teor}
Критерий эквивалентен условию, что $(p+1)$ систем формальных
степенных рядов определяемые сдвинутыми моментами
$\overrightarrow{f}^{\nu}=(f_{\nu},f_{\nu+1},...,f_{\nu+p-1}),
\mbox{   } \nu=1,...,p+1$ регулярны.

\begin{teor}
Рекурретные коэффициенты векторных ортогональных многочленов $Q_n
= Q_n^{0,0}$
$$%\begin{equation}
Q_{n+1}(z)=(z-a_{n,n})Q_n(z)-a_{n,n-1}Q_{n-1}(z)-\ldots-a_{n,n-p}Q_{n-p}(z)
$$%\end{equation}
могут быть вычислены из элементов векторной QD
таблицы $\alpha, \beta$ следующим образом:
%
\begin{eqnarray}
a_{n,n}=\sum\limits_{i_1=-1}^{p-1}{u_{n,n-i_1}} \nonumber\\
a_{n,n-1}=\sum\limits_{i_1=0}^{p-1}{u_{n,n-i_1}}
\sum\limits_{i_2=0}^{i_1}{u_{n-1,n-i_2}} \nonumber\\
a_{n,n-2}=\sum\limits_{i_1=1}^{p-1}{u_{n,n-i_1}^{\nu }}
\sum\limits_{i_2=1}^{i_1}{u_{n-1,n-i_2}^{\nu }}
\sum\limits_{i_3=1}^{i_2}{u_{n-2,n-i_3}^{\nu }} \nonumber \\
\cdots \nonumber\\
a_{n,n-p}={u_{n,n-p+1}}{u_{n-1,n-p+1}}\ldots {u_{n-p,n-p+1}}
\nonumber
\end{eqnarray}
%
где
\begin{equation}
\left\{
\begin{array}{llllllll}
u_{n,n+1} = \beta_{n+1}^{0} \\ \\
u_{n,n-j} = \alpha_{n}^{j}, j=0,\ldots,p-1
\end{array}
\right.
\end{equation}
\end{teor}

\noindent{\bf Доказательство: }
Для $p=1$ случая $$Q_{n+1}(z)=(z-a_{n,n})Q_n(z)-a_{n,n-1}Q_{n-1}(z)$$\\
Соотношение $Q_n^{k,d+1}=Q_n^{k,d}-\alpha_n^{\nu} Q_{n-1}^{k,d+1}$
можно записать в виде $$ Q_n^{k+1,d}=Q_n^{k,d}-\alpha_n^{\nu}
Q_{n-1}^{k+1,d}$$ Стартуя с  $Q_{n+1}^{k,d} =
xQ_n^{k+1,d}-\beta_{n+1}^{{\nu}} Q_n^{k,d}$ можно записать
\begin{eqnarray*}
Q_{n+1}^{k,d} & = & x(Q_n^{k,d}-\alpha_n^{\nu}
Q_{n-1}^{k+1,d})-\beta_{n+1}^{{\nu}}
Q_n^{k,d} \nonumber \\
& = & (x-\beta_{n+1}^{{\nu}})Q_n^{k,d}-\alpha_n^{\nu}
xQ_{n-1}^{k+1,d} \nonumber \\
& = & (x-\beta_{n+1}^{{\nu}})Q_n^{k,d}-\alpha_n^{\nu}
(Q_n^{k,d}+\beta_n^{\nu}Q_{n-1}^{k,d})
\end{eqnarray*}
В итоге получаем
\begin{eqnarray*}
Q_{n+1}^{k,d}= (x-(\beta_{n+1}^{{\nu}}+\alpha_n^{\nu}))Q_n^{k,d}-
\alpha_n^{\nu}\beta_n^{\nu}Q_{n-1}^{k,d}) \\
a_{n,n} =\beta_{n+1}^{0}+\alpha_n^{0}, \mbox{    } a_{n,n-1} =
\alpha_n^{0}\beta_n^{0}
\end{eqnarray*}
Для $p=2$ случая $$Q_{n+1}(z)=(z-a_{n,n})Q_n(z)-a_{n,n-1}Q_{n-1}(z)-a_{n,n-2}Q_{n-2}(z)$$\\
имеем следующее
\begin{equation}
Q_{n+1}^{k,d} =
(x-(\beta_{n+1}^{{\nu}}+\alpha_n^{\nu+1}+\alpha_n^{\nu}))Q_n^{k,d}-\nonumber \\
-(\alpha_n^{\nu+1}\beta_{n}^{{\nu}}+\alpha_n^{\nu}(\beta_{n}^{{\nu}}+\alpha_{n-1}^{\nu+1}))Q_{n-1}^{k,d}-
\alpha_n^{\nu}\alpha_{n-1}^{\nu+1}\beta_{n-1}^{{\nu}}
Q_{n-2}^{k,d} \nonumber
\end{equation}
Откуда
\begin{eqnarray*}
 a_{n,n}
=\beta_{n+1}^{{\nu}}+\alpha_n^{\nu+1}+\alpha_n^{\nu} \nonumber
\\ a_{n,n-1} =
\alpha_n^{\nu+1}\beta_{n}^{{\nu}}+\alpha_n^{\nu}(\beta_{n}^{{\nu}}+\alpha_{n-1}^{\nu+1})
\nonumber \\ a_{n,n-2} =
\alpha_n^{\nu}\alpha_{n-1}^{\nu+1}\beta_{n-1}^{{\nu}}
\end{eqnarray*}
Далее по индукции получаем соотношение теоремы. \\
