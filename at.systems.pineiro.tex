\subsection{AT системы. Система Пинейро.}
\begin{defi}
Для некоторого набора мер $d\mu_j(x)=\rho_j (x)dx, j=1,\ldots,p$
имеющих общий носитель ${\Delta}$, система Марковских функций
$\overrightarrow{f}=(f_1,f_2,\ldots,f_p)$, где
$$
 f_j(z)
=\int_{\Delta_j}{\displaystyle\frac{d\mu_j(x)}{z-x}}
$$
называется AT системой, если функции
$$
\rho_1(x),\ldots,\rho_p,x \rho_1,\ldots, x\rho_p, x^2\rho_1,
\ldots
$$
также образуют систему Маркова
\end{defi}

\begin{defi}
Для некоторого набора мер $d\mu_j(x)=\rho_j (x)dx, j=1,\ldots,p$
имеющих общий носитель ${\Delta}$, система Марковских функций
$\overrightarrow{f}=(f_0,f_1,\ldots,f_p)$, где
$$
 f_j(z)
=\int_{\Delta_j}{\displaystyle\frac{d\mu_j(x)}{z-x}}
$$
называется MT системой, если функции
$$
\rho_1(x),\ldots,\rho_p,x \rho_1,\ldots, x\rho_p, x^2\rho_1,
\ldots
$$
образуют систему Чебышева на $\Delta$, т.е.
\end{defi}

\begin{defi}
AT система для которой
$$
d\mu_j(x)=x^{\alpha_j}(1-x)^{\alpha_0}dx, j=1,\ldots,p
$$
где $\alpha_j>-1, \alpha_i-\alpha_j \not \in \textbf{Z}$ \\
называется системой Пинейро.
\end{defi}


\begin{teor} \rm ~\cite{AptekaaKaliaJvaniseg} \textit{
Для системы Пинейро
$$
d\mu_j(x)=x^{\alpha_j}(1-x)^{\alpha_0}dx, j=1,\ldots,p
$$
где $\alpha_j>-1, \alpha_i-\alpha_j \not \in \textbf{Z}$ \\
известна формула Родригеса для соответствующих векторных
ортогональных многочленов со старшим коэффициентом единица
$$
Q_{\overrightarrow{n}}=\frac{(1-x)^{-\alpha_0}}{M_{\overrightarrow{n}}}
\prod_{j=1}^{p} { \left( x^{-\alpha_j} \frac{d^{n_j}} {dx^{n_j}}
x^{n_j+\alpha_j} \right) (1-x)^{n+\alpha_0}}
$$
где
$$
M_{\overrightarrow{n}}=(-1)^{n} \prod_{j=1}^{p} {
\frac{\Gamma(n+n_j+\alpha_j+\alpha_0+1)}{\Gamma(n+\alpha_j+\alpha_0+1)}}
$$}
\end{teor}

\begin{teor} \rm ~\cite{AptekaaKaliaJvaniseg} \textit{
Для систем Пинейро при $\Delta=[0,1], p=2$
$$
\begin{array}{llll}
d\mu_1(x)=x^{\alpha_1}(1-x)^{\alpha_0}dx \\
d\mu_2(x)=x^{\alpha_2}(1-x)^{\alpha_0}dx
\end{array}
$$
где $\alpha_j>-1, \alpha_i-\alpha_j \not \in \textbf{Z}$ \\
известны асимптотики для коэффициентов рекуррентного соотношения
$$
\begin{array}{llll}
\lim b_{n,n}=\displaystyle 3 \left( \frac{4}{27} \right) \\
\lim b_{n,n-1}=\displaystyle 3 \left(\frac{4}{27} \right)^{2} \\
\lim b_{n,n-2}=\displaystyle \left(\frac{4}{27} \right)^{3}
\end{array}
$$
В этом случае матрица оператора является компактным возмущением
оператора выраженного
следующей 3х диагональной маgрицей : $$%\begin{equation}
\left(
\begin{array}{cccccccc}
3\alpha^2 & 1 & 0 & 0 & 0 & \ldots \\
3\alpha^4 & 3\alpha^2 & 1 & 0 & 0 & \ldots \\
\alpha^6 & 3\alpha^6 & 6\alpha^2 & 2 & 0 & \ldots \\
0 & \alpha^6 & 3\alpha^4 & 3\alpha^2 & 1 & \ldots \\
\ldots & \ldots & \ldots & \ldots & \ldots & \ldots \\
\end{array}
\right) $$%\end{equation}
где $\alpha=\displaystyle\sqrt{\frac{4}{27}}$. Спектр оператора
определяется кривыми алгебраической функции $W(z):
(W+\alpha^2)^3-zW=0$ }\\
Известны прямые формулы для коэффициентов соответствующей
векторной непрерывной дроби Стилтъеса:
$$S(z)=
\displaystyle {(1,\cdots,1)\over (0,\cdots 0,z)+}\
{(1,\cdots,1,-a_1)\over (0,\cdots 0,1)+\cdots}\ \cdots
{(1,\cdots,1,-a_{p})\over (0,\cdots 0,1)+}\
{(1,\cdots,1,-a_{p+1})\over (0,\cdots 0,z)+\cdots }\
$$
где
$$
\begin{array}{lllllllllllllll}
a_{6k+1}=\displaystyle
\frac
{(2k+1+\alpha_1+\alpha_0)(2k+1+\alpha_2+\alpha_0)(k+1+\alpha_1)}
{(3k+1+\alpha_1+\alpha_0)(3k+2+\alpha_1+\alpha_0)(3k+1+\alpha_2+\alpha_0)}
\\
a_{6k+2}=\displaystyle\frac
{(2k+1+\alpha_2+\alpha_0)(2k+1+\alpha_0)(k+\alpha_2-\alpha_1)}
{(3k+1+\alpha_2+\alpha_0)(3k+2+\alpha_2+\alpha_0)(3k+2+\alpha_1+\alpha_0)}
\\
a_{6k+3}=\displaystyle\frac
{(2k+2+\alpha_1+\alpha_0)(2k+1+\alpha_0)(k+1+\alpha_1-\alpha_2)}
{(3k+2+\alpha_1+\alpha_0)(3k+3+\alpha_1+\alpha_0)(3k+2+\alpha_2+\alpha_0)}
\\
a_{6k+4}=\displaystyle\frac
{(2k+2+\alpha_2+\alpha_0)(2k+2+\alpha_1+\alpha_0)(k+1+\alpha_2)}
{(3k+2+\alpha_2+\alpha_0)(3k+3+\alpha_2+\alpha_0)(3k+3+\alpha_1+\alpha_0)}
\\
a_{6k+5}=\displaystyle\frac
{(2k+2+\alpha_2+\alpha_0)(2k+2+\alpha_0)(k+1)}
{(3k+3+\alpha_2+\alpha_0)(3k+3+\alpha_1+\alpha_0)(3k+4+\alpha_1+\alpha_0)}
\\
a_{6k}=\displaystyle\frac {(2k+1+\alpha_1+\alpha_0)(2k+\alpha_0)k}
{(3k+\alpha_2+\alpha_0)(3k+1+\alpha_2+\alpha_0)(3k+1+\alpha_1+\alpha_0)}
\end{array}
$$
\end{teor}
