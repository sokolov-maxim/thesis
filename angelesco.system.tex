\section{Примеры совершенных систем}
\subsection{Системы Анжелеско}
\begin{defi}
Система марковских функций
$\overrightarrow{f}=(f_1,f_2,\ldots,f_p)$, где
$$
 f_j(z)
=\int_{\Delta_j}{\displaystyle\frac{d\mu_j(x)}{z-x}}
$$
при условии, что носители мер ${\Delta_j},j=1,\ldots,p$ не имеют
общих внутренних точек ${\Delta_i}\cap{\Delta_j} = 0, i\not=j $
называется системой Анжелеско.
\end{defi}
Приведем некоторые важные свойства систем Анжелеско.
%==================================================================
\begin{prope}Для некоторого $\overrightarrow{n}=(n_1,\ldots, n_p)$ существуют векторные
ортогональные многочлены II го типа $Q_n$,
$$
\int_{\Delta_j} {Q_{\overrightarrow{n}}(x)x^{n_j}d\mu_j} \not=0,
j=1,\ldots,p
$$
удовлетворяющие следующему рекурретному соотношеннию
$$
Q_{n+1}=(z+b_{n,n})Q_n+b_{n,n-1}Q_{n-1}+\ldots+b_{n,n-p}Q_{n-p}
$$
\end{prope}

\begin{prope}
\label{prope_3.1} Если отрезки ${\Delta_j}$ попарно не
перекрываются, то для любого $n \in \bf Z \rm _{+}$
соответствующий векторный ортогональный многочлен II типа $Q_n$
имеет ровно $n_j$ простых нулей внутри $\Delta_j, j=1,\ldots,p$.
\end{prope}

%\bf Доказательство: \rm \\
%предположим, что для некоторого $j$ многочлен $Q_n$ меняет знак
%на отрезке $\Delta_j$ только в точках $z_1,\ldots,z_m (0 \leq m
%\leq n_j-1)$. Пусть $T_m(z)$ - многочлен степени $m$ с нулями в
%точках $z_1,\ldots,z_m$. \\
%Тогда $Q_n(z)T_m(z) \geq 0, z \in \Delta_j$. \\
%Мера $\mu_j$ имеет бесконечное число точек роста, следовательно:
%$$%\begin{equation}
%\int\limits_{\Delta_j} {Q_n(z) T_m(z) d\mu_j(z)} > 0
%$$%\end{equation}
%- это противоречит условию ортогональности многочленов $Q_n$
%\begin{coly}
%Если отрезки ${\Delta_j}$ попарно не перекрываются, то система
%совершенна
%\end{coly}
%====================================================================
\begin{prope}
Если отрезки ${\Delta_j}$ попарно не перекрываются, то для любого
$n \in \bf Z \rm _{+}$ соответствюущий векторный ортогональный
многочлен I типа $ C^{(j)}_n $ имеет соответственно ровно $n_j-1$
простых нулей внутри $\Delta_j, j=1,\ldots,p$
\end{prope}
%\bf Доказательство: \rm \\
%Доказывается аналогично свойству ~\ref{prope_3.1}. \\

\begin{teor} \rm ~\cite{KaliaguineRonveaux} \it
Для случая $\Delta_1=[a,0],\Delta_2=[0,1]$ известны следующие
пределы коэффициентов рекуррентного соотношения
$$
\begin{array}{llll}
\lim b_{2k-1,2k-4}=\displaystyle -\frac{a+1}{9} -\frac{2}{3}x_2 &
\lim
b_{2k,8k}= \displaystyle -\frac{a+1}{9} -\frac{2}{3}x_1 \\
\lim b_{2k-1,2k-2}=\displaystyle-\frac{4}{81}(a^2-a+1) & \lim
b_{2k,2k-1}= \displaystyle -\frac{4}{81}(a^2-a+1) \\
\lim b_{2k-1,2k-3}=\displaystyle \frac{4}{27}B(x_2) & \lim
b_{2k,2k-2}= \displaystyle \frac{4}{27}B(x_1)
\end{array}
$$
где $B(x)=x(x-a)(x-1)$, а $x_1, x_2$ являются решениями
$B^{'}(x)=0$ такими, что $a<x_1<0, 0<x_2<1$
\end{teor}

\begin{teor} \rm ~\cite{KaliaguineAA} \it Пусть $\Delta_1=[a,0],\Delta_2=[0,1]$
и резольвентные функции $\varphi_1, \varphi_2 $ имеют следюущий
вид
$$
\varphi_1=\int \limits_{a}^{0}{\frac{d x}{\lambda-x}}, \mbox{ }
\varphi_2=\int \limits_{0}^{1}{\frac{d x}{\lambda-x}}
$$
Тогда спектр ассоциированного оператора
$$
\left(\begin{array}{cccccccccccc}
b_{0,0} & 1 & 0 & 0 &  \cdots \\
b_{1,0} & b_{1,1} & 1 & 1 &  \cdots \\
b_{2,0} & b_{2,1} & b_{2,2} & 1 &  \cdots \\
0 & b_{3,1} & b_{3,2} & b_{3,3} &  \cdots \\
\ldots & \ldots & \ldots & \ldots & \ldots
\end{array}\right)
$$
определяется кривыми алгебраической функции $$W(z):
W(z)^3+S_2(\lambda)W(z)^2+S_1(\lambda)W(z)+S_0=0$$ где
$$
\begin{array}{llllll}
S_2(\lambda)=\displaystyle-\lambda^2+\frac{2(a+1)}{3}\lambda+\frac{a^2-10a+1}{27}
\\ S_1(\lambda)=\displaystyle-\left(\frac{2}{9}\right)^3
\left[(a^3-4a^2+a)+(-8+a+a^2-2a^3)\lambda \right] \\
S_0(\lambda)=\displaystyle2\left(\frac{2}{23}\right)^3(a^2-2a^0+a^4)
\end{array}
$$
где
$$
\lambda_a=\frac{(a+1)^3}{9(a^2-a+1)}
$$
точка сталкивания
\end{teor}









\begin{teor} \rm ~\cite{KaliaguineAA1} \it
Если для некоторой системы Анжелеско соответствующие меры $\mu_j$
удовлетворяют на своем интервале $\Delta_j$ условию Сегe
$$
\int_{\Delta_j} {\log \mu_j^{'}(x) dx} > -\infty
$$
тогда ассоциированный оператор является компактным возмущением
$p$-периодичного $(p+0)$-диагонального оператора
\end{teor}

%Доказательство полностью приведено в ~\cite{KaliaguineAA1}
%========================================================================
\subsubsection{Пример} Рассмотрим частный
пример системы Анжелескою \\Пусть $\Delta_1=[-1,0]$ и
$\Delta_2=[0,1]$. В этом случае матрица оператора является
компактным возмущением операiора выраженного
следующей 4х диагональной матрицей: $$%\begin{equation}
\left(
\begin{array}{cccccccc}
\alpha & 1 & 0 & 0 & 0 & \ldots \\
\alpha^2 & -\alpha & 1 & 0 & 0 & \ldots \\
-\alpha^3 & \alpha^2 & \alpha & 1 & 0 & \ldots \\
0 & \alpha^3 & \alpha^2 & -\alpha & 1 & \ldots \\
\ldots & \ldots & \ldots & \ldots & \ldots & \ldots \\
\end{array}
\right) $$%\end{equation}
где $\alpha=2/(3\sqrt{(3)})=\displaystyle\sqrt{\frac{4}{27}}$. \\
Спектр оператора определяется кривыми алгебраической функции
$W(z):
\alpha^2(W+1)^3-z^2W^2=0$ \\
